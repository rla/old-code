% Dokumendi klass, fondi suurus ja keel

\documentclass[12pt,estonian]{report}

% Kasutatavad paketid

\usepackage[estonian]{babel}
\usepackage[latin1]{inputenc}
\usepackage{setspace}
\usepackage{listings}
\usepackage{latexsym}
\usepackage{amsmath}
\usepackage{float}
\usepackage{amssymb}
\usepackage{geometry}

% Makrodes kasutatavad loendurid
% Lemmade, teoreemide ja algoritmide jaoks

\newcounter{lemma_counter}\setcounter{lemma_counter}{1}
\newcounter{theorem_counter}\setcounter{theorem_counter}{1}
\newcounter{algorithm_counter}\setcounter{algorithm_counter}{1}

% Kasutatavad makrod
% Lemma väljastamine

\newcommand{\lemma}[1]{
	\hspace{1cm}\textbf{Lemma \arabic{lemma_counter}\stepcounter{lemma_counter}. \textit{#1}}
}

% Teoreemi väljastamine

\newcommand{\theorem}[1]{
	\hspace{1cm}\textbf{Teoreem \arabic{theorem_counter}\stepcounter{theorem_counter}. \textit{#1}}
}

% Tõestuse alguse tähistamine kirjaga "Tõestus:"

\newcommand{\proof}[0]{
	\hspace{1cm}\textbf{Tõestus.}
}

% Tõestuse jagamine osadeks, osa "nimi"

\newcommand{\proofpart}[1]{
	\hspace{1cm}\textbf{#1. }
}

% Tõestuse lõpumärk

\newcommand{\proofend}[0]{\(\square\)}

% Protseduuri nimi erikirjas

\newcommand{\proc}[1]{\textsf{#1}}

% Horisontaalne joon

\newcommand{\horline}[0]{
	\vspace{1ex}\hrule width\columnwidth\vspace{1ex}
}

% Keskkond algoritmi väljastamiseks
% Kasuta kui \begin{algorithm}{OP}

\newenvironment{algorithm}[1]{
	\vspace{0.5cm}
	\horline
	\textbf{Algoritm \arabic{algorithm_counter}\stepcounter{algorithm_counter}}. \proc{#1}
}{
	\horline
	\vspace{0.5cm}
}

% Vastavalt algoritmi sisend/väljund ja keha

\newcommand{\ainput}[1]{\newline\emph{sisend:} #1}
\newcommand{\aoutput}[1]{\newline\emph{väljund:} #1}
\newcommand{\abody}[0]{\horline}

% "Listing" keskkonna parameetrid

\lstset{
	language=Prolog,
	basicstyle=\small\tt,
	commentstyle=\tt
}


% Vormistuslikud seaded
% Pealkirjad nummerdamata

%\setcounter{secnumdepth}{-2}

% Lehekülgede ääred

\geometry{verbose,a4paper,tmargin=2.5cm,bmargin=2.5cm,lmargin=4cm,rmargin=3cm}

% Mitu taset näidata sisukorras

\setcounter{tocdepth}{2}

% Jäta lõikude vahele rohkem ruumi

\setlength{\parskip}{\medskipamount}

% Ära taanda lõike

\setlength{\parindent}{0pt}

% Määra reavahe

\onehalfspacing

\begin{document}

% Siit algab tiitelleht

\thispagestyle{empty}

\begin{center}
{\large\sffamily{
T A R T U\ \ \ Ü L I K O O L\\
MATEMAATIKA-INFORMAATIKATEADUSKOND\\
Arvutiteaduse instituut\\
Informaatika eriala
}}

\vspace*{\stretch{5}}

{\textbf{\Large Raivo Laanemets}}

\vspace{4mm}

{\textbf{\LARGE DPLL protseduuri täiendamine lahendite täpseks loendamiseks}}

\vspace{5mm}

{\textbf{\Large Bakalaureusetöö (4 AP)}}

\end{center}

\vspace*{\stretch{4}}

{\hfill{\Large Juhendaja: Tõnu Tamme, MSc}}

\vspace*{\stretch{3}}

\begin{flushright}

\noindent Autor: ........................................... ``.....'' mai\hskip16pt 2007

\vspace{2mm}

\noindent Juhendaja: .................................... ``.....'' mai\hskip16pt 2007

\end{flushright}

\vspace*{\stretch{2}}

\begin{center}
{\large{TARTU 2007}}
\end{center}

% Siin lõpeb tiitelleht

\newpage

\tableofcontents{}

\chapter*{Sissejuhatus}
\addcontentsline{toc}{chapter}{\numberline{}Sissejuhatus}

Lausearvutusvalemi lahendite loendamisel on erinevaid rakendusi. Näiteks
loogikaskeemide omaduste väljaselgitamine ja testimismustrite genereerimine.
Lahendite loendamisele saab taandada tuletust Bayes'i võrgus ja Dedekindi
numbrite leidmist. Selles töös vaatleme lahendite täpset loendamist.
Koostatud on algoritme ka lahendite arvu umbkaudseks leidmiseks,
kasutades tõenäosuslikke ja randomiseeritud meetodeid.
Paraku on teada, et isegi umbkaudne loendamine garanteeritud väikese veaga on
arvutuslikult raske ülesanne. Täpse lahendite arvu leidmise probleem
on \#P-täielik \cite{key-8}.

Selles töös võtame vaatluse alla algoritmi, mis on saadud kehtestatavuse
algoritmi täiendades. Kehtestatavuse algoritm baseerub artiklites
\cite{key-1,key-2} kirjeldatud protseduuril. Nendes artikites on
kirjeldatud algoritmi pedikaatarvutuses kirja pandud teoreemide tõestamiseks
(predikaatarvutuse valemi $F$ samaselt tõesuse näitamine). Esitatud
protseduuri saab kirjeldada kolmest osast koosnevana:

\begin{enumerate}
\item Valemi $\neg F$ teisendamine prefikskujule $F'$.
\item Valemist $F'$ kvantorivabade konjuktiivsel normaalkujul lausearvutusvalemite $F'_{k}$
genereerimine.
\item Eelmises sammus saadud valemi $F'_{k}$ kehtestatavuse kontrollimine.
Kui $F'_{k}$ on kehtestatav, siis mine tagasi sammu (2), vastasel
korral väljasta {}``tõene'' ja lõpeta töö.
\end{enumerate}
Algoritm toetud faktile, et valem $F$ on samaselt tõene parajasti
siis, kui $\neg F$ on mittekehtestatav. Sammus (2) tehtavad teisendused
säilitavad kehtestatavust (s-ekvivalentsus). Sammu (3) sooritamiseks
rakendatakse valemile teatavaid süntaktilisi teisendusi kuni saadakse
tühivalem või valem, mis sisaldab tühiklauslit. Tühivalem on kehtestatav,
aga tühiklauslit sisaldav valem ei ole. Praktiliselt kõik lausearvutuse
kehtestatavuse täielikud aloritmid kasutavad nendes kahes artiklis välja toodud
teisendusreegleid.

Selles töös ei kasuta me kõiki artiklites \cite{key-1,key-2}
esitatud reegleid, vaid võtame aluseks nn. hargnemisreegli
(\emph{branching rule}) ja selle erikujud ning valemit lihtsustavad
reeglid. Need reeglid säilitavad kehtestatavust ning hilisemas töö osas
näitame, et nende reeglite jaoks kehtivad teatud võrrandid,
mis võimaldavad lahendite arvu loendamist.

Kõigepealt defineerime töös kasutatavad mõisted, seejärel kirjeldame
teisendusreeglid, tõestame nende korrektsuse ja anname näiteid nende
reeglite rakendamisest kehtestavate valemite tuletamiseks.
Pärast seda esitame kehtestatavuse otsinguprotseduuri
ja näitame selle korrektsust ning termineeruvust.
Lõpuks täiendame otsinguprotseduuri loendamiseks ning
tõestame ka selle korrektsuse ja termineeruvuse.


\section*{Definitsioonid}

\emph{Muutujaks} \emph{}nimetame suvalist sümbolit $x$ fikseeritud sümbolite
hulgast. \emph{Literaaliks} $l$ loeme muutujat $x$ ($l\equiv x$) või tema
eitust $-x$ ($l\equiv-x$). \emph{Klausliks} $C$ loeme literaalide disjunktsiooni
$C=l_{1}\vee..\vee l_{n}$. \emph{Konjuktiivsel normaalkujul valemiks} $F$
loeme klauslite konjuktsiooni $F=C_{1}\wedge..\wedge C_{k}$. Muutuja
$x$ \emph{väärtustuseks} nimetame funktsiooni $v$, $v(x)\in\{0,1\}$. Literaali
$l$ loeme tõeseks, kui $l\equiv x$ ja $v(x)=1$ või $l\equiv-x$
ja $v(x)=0$. Literaali $l$ loeme vääraks, kui $l\equiv x$ ja $v(x)=0$
või $l\equiv-x$ ja $v(x)=1$. Literaali $l\equiv x$ \emph{komplementaariks}
nimetame literaali $\bar{l}\equiv-x$ ja literaali $l'\equiv -x$ komplementaariks
literaali $\bar{l'}\equiv x$. Klauslit $C=l_{1}\vee..\vee l_{n}$
nimetame väärtustusel $v$ tõeseks ($v(C)=1$), kui $C$ sisaldab
vähemalt ühte tõest literaali, vastasel korral vääraks ($v(C)=0$).
Valemit $F=C_{1}\wedge..\wedge C_{k}$ loeme väärtustusel $v$ tõeseks
($v(F)=1$), kui ükski klauslitest $C_{1},..,C_{k}$ pole väär, vastasel
korral vääraks ($v(F)=0$). Töös eeldame, et ükski klausel ei sisalda
korraga literaali $l$ ja tema komplementaari $\bar{l}$. Samuti eeldame,
et ükski literaal ei esine samas klauslis mitu korda ja valem ei sisalda
ühtegi klauslit topelt. Esimesel juhul on võib klausli eemaldada, sest
selline klausel on alati tõene ja mingit infot ta ei anna. Teisel juhul
võib topelt esinevad literaalid või klauslid eemaldada, jättes alles
vastavalt ainult ühe literaali või klausli esinemise.

Konjuktiivsel normaalkujul valemi $F=C_{1}\wedge..\wedge C_{n}$
võib samastada klauslite hulgaga $\{ C_{1},..,C_{n}\}$. Klausli $C=l_{1}\vee..\vee l_{k}$
võib samastada literaalide hulgaga $\{ l_{1},..,l_{k}\}$. \emph{Tühivalemi}
$\emptyset$ loeme tõeseks. \emph{Tühiklausli} $\emptyset$ loeme vääraks.
Konjuktiivsel normaalkujul valemi, mis sisaldab tühiklauslit, loeme
samuti vääraks. Valemi ja klauslite väljendamine hulga kujul teeb
lihtsamaks ja loetavamaks algoritmide esitamise.

Valemist $F$ rääkides mingi konkreetse väärtustuse $v$ piires kasutame
tähistusviisi $F|_{A=1}$ näitamaks, et $v(A)=1$.


\chapter{Tuletamine}

Siin kirjeldatavad ja kasutatavad tuletusreeglid säilitavad valemi
kehtestatavust. Tuletamine toimub ülaltpoolt allapoole, mis tähendab,
kui reegli ülemises osas olev valem on tõene mingil väärtustusel,
siis on seda ka reegli alumises osas olev valem. Iga reegli jaoks
anname tema üldkuju, kommentaarid reegli kohta ja näite tema kasutamisest.
Reeglite korrektsuse tõestame töö järgmises osas.

Tuletussüsteemi ainsaks aksioomiks on tühivalem $\emptyset|_{x_1=\delta_1,...,x_l=\delta_l}$ koos mingi
kitsendava väärtustusega $v(x_i)=\delta_i$, $1\le i\le l$, $\delta_i \in \{0,1\}$. On selge, et tühivalem
on suvalisel väärtustusel kehtestatav.

Reeglite valikul ja kirjeldamisel on otseselt silmas peetud nende sobivust
valemi kehtestatavuse otsinguprotseduuri jaoks.

\section{Teisendusreeglid}


\subsection{Tõeste klauslite elimineerimine}

Tõeste klauslite elimineerimisreegel on valemit lihtsustav reegel
mingi väärtustuse piires. Näiteks vaatleme valemit

$F|_{A=1}=C_{1}\wedge C_{2}$, kus $C_{1}=A\vee B$ ja $C_{2}=C\vee B$.
Ilmselt on selge, et valemit $F$ saab lihtsustada, eemaldades sellest
tõesed klauslid. Antud juhul on tõene $C_{1}$ ning me saame lihtsustatud
valemiks $F'=C_{2}$. 

Reegli üldkuju positiivselt esineva muutuja jaoks \emph{(R1p)}:

\[
\frac{C_{1}\wedge\dots\wedge C_{i-1}\wedge C_{i+1}\wedge\dots\wedge C_{k}|_{x=1}}{C_{1}\wedge\dots\wedge C_{i-1}\wedge C_{i}\wedge C_{i+1}\wedge\dots\wedge C_{k}|_{x=1}}R1p,\]

kus $C_{i}=l_{i_{1}}\vee\dots\vee l_{i_{j-1}}\vee l_{i_{j}}\vee l_{i_{j+1}}\vee\dots\vee l_{i_{l}}$ ja $l_{i_{j}}\equiv x$.


Reegli üldkuju negatiivselt esineva muutuja jaoks \emph{(R1n)}:

\[
\frac{C_{1}\wedge\dots\wedge C_{i-1}\wedge C_{i+1}\wedge\dots\wedge C_{k}|_{x=0}}{C_{1}\wedge\dots\wedge C_{i-1}\wedge C_{i}\wedge C_{i+1}\wedge\dots\wedge C_{k}|_{x=0}}R1n,\]

kus $C_{i}=l_{i_{1}}\vee\dots\vee l_{i_{j-1}}\vee l_{i_{j}}\vee l_{i_{j+1}}\vee\dots\vee l_{i_{l}}$ ja $l_{i_{j}}\equiv -x$.


Elimineerimisreeglit on mõistlik kasutada vahetult pärast muutuja
väärtuse fikseerimist (näiteks vahetult pärast väärtustusreegli rakendamist).
See võimaldab eemaldada valemist klauslid, mis on juba tõeseks muutunud,
ning mis ei anna midagi juurde valemi kehtestatavusele. Kuivõrd valemis
võib literaal $(l\equiv x)$ sisalduda rohkem kui ühes klauslis, tuleb
reeglit rakendada järjest, kuni kõik sellised klauslid on elimineeritud.


\subsection{Väärade literaalide elimineerimine}

Eelmise reegli abil saime valemist eraldada klauslid, mis olid juba
tõesed, ning mis ei anna midagi juurde valemi muutujate sobivate väärtustuste
otsimisel. Samuti võime klauslitest kustutada literaalid, mille kohta
teame, et nad on seni leitud väärtustuses väärad. Võtame näiteks valemi
$F|_{B=0}=C_{1}\wedge C_{2}$, kus $C_{1}=A\vee B$ ja $C_{2}=C\vee B$.
Mõlemast klauslist $C_{1},C_{2}$ võime eemaldada literaali $B$.
Selle tulemusena saame valemi $F'|_{B=0}=C'_{1}\wedge C'_{2}$, kus
$C'_{1}=A$ ja $C'_{2}=B$.

Reegli üldkuju positiivselt esineva muutuja jaoks \emph{(R2p)}:

\[
\frac{C_{1}\wedge\dots\wedge C_{i}'\wedge..\wedge C_{k},C_{i}'=l_{i_{1}}\vee\dots\vee l_{i_{j-1}}\vee l_{i_{j+1}}\dots\vee l_{i_{l}}|_{x=0}}{C_{1}\wedge\dots\wedge C_{i}\wedge..\wedge C_{k},C_{i}=l_{i_{1}}\vee\dots\vee l_{i_{j-1}}\vee l_{i_{j}}\vee l_{i_{j+1}}\dots\vee l_{i_{l}},l_{i_{j}}\equiv x|_{x=0}}R2p\]


Reegli üldkuju negatiivselt esineva muutuja jaoks \emph{(R2n)}:

\[
\frac{C_{1}\wedge\dots\wedge C_{i}'\wedge..\wedge C_{k},C_{i}'=l_{i_{1}}\vee\dots\vee l_{i_{j-1}}\vee l_{i_{j+1}}\dots\vee l_{i_{l}}|_{x=1}}{C_{1}\wedge\dots\wedge C_{i}\wedge..\wedge C_{k},C_{i}=l_{i_{1}}\vee\dots\vee l_{i_{j-1}}\vee l_{i_{j}}\vee l_{i_{j+1}}\dots\vee l_{i_{l}},l_{i_{j}}\equiv-x|_{x=1}}R2n\]


Sarnaselt eespool kirjeldatud tõeste klauslite elimineerimise reegliga,
on mõistlik antud reeglit rakendada vahetult pärast mingi muutuja
väärtuse fikseerimist. Samuti võib muutuja esineda väärana rohkem
kui ühes klauslis. Seega tuleks ka seda reeglit rakendada järjest
nii kaua, kuni vastavaid klausleid enam valemis ei ole.


\subsection{Väärtustusreegel}

Väärtustusreegel fikseerib mingi valemis esineva muutuja $x$ väärtustuse. Väärtustusreegleid
on kaks: reegel (a), mis annab muutujale väärtuseks ``tõene'' ja
reegel (b), mis annab muutujale väärtuseks ``väär''. Reegli erikujuks
ühikklausli väärtustusreegel, mis kasutab valemis esinevat informatsiooni
muutuja väärtuse valimisel. 

% Hargnemisreegel (\emph{branching rule}) jaotab valemi kaheks komponendiks.
% Jaotamine tehakse mingi muutuja järgi, mis pole seni veel väärtustatud.
% Ühte komponenti võetakse muutuja tõese väärtusega ja teise komponenti
% väära väärtusega. Näiteks olgu antud valem $F=C_{1}\wedge C_{2}$,
% kus $C_{1}=A\vee B$ ja $C_{2}=C\vee B$. Valime jaotamise aluseks
% muutuja $A$. Siis saame kaks valemit $F|_{A=1}$ ja $F|_{A=0}$.
% {[}näidata tulemust pärast lihtsustuste sisseviimist] Sisuliselt võimaldab
% hargnemisreegel jagada kehtestatavuse otsingu kaheks: esimeses osas
% otsitakse mudelit, kus $v_{1}(A)=1$ ning teises osas mudelit, kus
% $v_{2}(A)=0$. On selge, et selline jaotus jagab valemi väärtustuste
% hulga kaheks. Seda omadust saame väga hästi ära kasutada valemi tõeväärtuste
% loendamisel. Siis esitame hargnemisreegli kahe eraldi tuletusreeglina.
% Esimeses reeglis (a) valitakse muutuja väärtuseks ``tõene'', teises
% (b) aga ``väär''.

Väärtustusreegli a üldkuju \emph{(R3p)}:


\[
\frac{C_{1}\wedge\dots\wedge C_{i}\wedge\dots\wedge C_{k}|_{x=1}}{C_{1}\wedge\dots\wedge C_{i}\wedge\dots\wedge C_{k}}R3p\]


Väärtustusreegli b üldkuju \emph{(R3n)}:

\[
\frac{C_{1}\wedge\dots\wedge C_{i}\wedge\dots\wedge C_{k}|_{x=0}}{C_{1}\wedge\dots\wedge C_{i}\wedge\dots\wedge C_{k}}R3n\]


Väärtustusreegli rakendamise järel on mõistlik rakendada järjest nii
tõeste klauslite elimineerimisreeglit kui ka väärade literaalide elimineerimisreeglit,
et eemaldada valemitest mittevajalikud komponendid.


\subsection{Ühikklausli väärtustamine}

Ühikklausli väärtustamine on väärtustusreegli erikuju.
Olgu antud valem $F=C_{1}\wedge C_{2}$, kus $C_{1}=A$ ja $C_{2}=B\vee\neg A$.
Rakendame nüüd hargnemisreeglit $A$ järgi. Saame kaks valemit $F|_{A=1}$
ja $F|_{A=0}$. On selge, et valem $F|_{A=0}$ on väär, sest klausel
$C_{1}$ on väär (ning saaksime pärast väära literaali $A$ elimineerimist
tühiklausli).

Reegli üldkuju positiivselt esineva muutuja jaoks \emph{(R4p)}:\[
\frac{C_{1}\wedge\dots\wedge C_{i}\wedge\dots\wedge C_{k},C_{i}=l_{i},l_{i}\equiv x|_{x=1}}{C_{1}\wedge\dots\wedge C_{i}\wedge\dots\wedge C_{k}}R4p\]


Reegli üldkuju negatiivselt esineva muutuja jaoks \emph{(R4n)}:

\[
\frac{C_{1}\wedge\dots\wedge C_{i}\wedge\dots\wedge C_{k},C_{i}=l_{i},l_{i}\equiv-x|_{x=0}}{C_{1}\wedge\dots\wedge C_{i}\wedge\dots\wedge C_{k}}R4n\]

\section{Näidistuletus}

Järgnevalt on esitatud valemi $F=C_{1}\wedge C_{2}\wedge C_{3}$,
kus $C_{1}=A\vee B\vee\neg C$, $C_{2}=\neg A\vee\neg B$ ja $C_{3}=C\vee\neg B$
tuletuspuu:

% Näidistuletus

$$\dfrac{\dfrac{\dfrac{\dfrac{\dfrac{\dfrac{\emptyset|_{A=1,B=0}}{(C\vee\neg B)|_{A=1,B=0}}R1n}{(\neg B)\wedge(C\vee\neg B)|_{A=1,B=0}}R1n}{(\neg B)\wedge(C\vee\neg B)|_{A=1}}R4n}{(\neg A\vee\neg B)\wedge(C\vee\neg B)|_{A=1}}R2n}{(A\vee B\vee\neg C)\wedge(\neg A\vee\neg B)\wedge(C\vee\neg B)|_{A=1}}R1p}{(A\vee B\vee\neg C)\wedge(\neg A\vee\neg B)\wedge(C\vee\neg B)}R3p$$


\section{Teisendusreeglite korrektsus}

Nüüd näitame ülalkirjeldatud reeglite korrektsuse, st. näitame, et
reeglid \emph{R1p, R1n, R2p, R2n, R3p, R3n, R4p} ja
\emph{R4n} säilitavad valemi kehtestatavuse. Kõik need tõestused on
sarnased: kõigepealt valime suvalise interpretatsiooni $v$, kus reegli
ülemisel pool olev valem on tõene, seejärel näitame, et selles interpretatsioonis
$v$ on ka reegli alumisel poolel olev valem tõene.

\lemma{Tõeste klauslite elimineerimine on korrektne}

Jagame tõestuse kahte ossa: kõigepealt näitame reegli \emph{R1p} korrektsuse,
siis teeme sedasama reegli \emph{R1n} jaoks.

\proofpart{R1p} Vaatleme valemit $F=C_{1}\wedge\dots\wedge C_{i-1}\wedge C_{i+1}\wedge\dots\wedge C_{k}$.
Näitame suvalise interpretatsiooni $v$ jaoks, kus $v(x)=1$, kui
$v(C_{1}\wedge\dots\wedge C_{i-1}\wedge C_{i+1}\wedge\dots\wedge C_{k})=1$,
siis ka $v(C_{1}\wedge\dots\wedge C_{i-1}\wedge C_{i}\wedge C_{i+1}\wedge\dots\wedge C_{k})=1$,
kus $C_{i}=l_{i_{1}}\vee\dots\vee l_{i_{l}}$ ja $l_{i_{j}}\equiv x$.

\proof

$v(C_{1}\wedge\dots\wedge C_{i-1}\wedge C_{i+1}\wedge\dots\wedge C_{k})=1$
ja $v(x)=1$  $\Rightarrow$\\
$v(C_{1})=\dots v(C_{i-1})=v(C_{i+1})=\dots=v(C_{k})=1$ ja $v(l_{i_{j}})=1$
$\Rightarrow$\\
$v(C_{1})=\dots v(C_{i-1})=v(C_{i+1})=\dots=v(C_{k})=1$ ja $v(C_{i})=1$
$\Rightarrow$\\
$v(C_{1}\wedge\dots\wedge C_{i-1}\wedge C_{i}\wedge C_{i+1}\wedge\dots\wedge C_{k})=1$
 \proofend

\proofpart{R1n} Vaatleme valemit $F=C_{1}\wedge\dots\wedge C_{i-1}\wedge C_{i+1}\wedge\dots\wedge C_{k}$.
Näitame suvalise interpretatsiooni $v$ jaoks, kus $v(x)=0$, kui
$v(C_{1}\wedge\dots\wedge C_{i-1}\wedge C_{i+1}\wedge\dots\wedge C_{k})=1$,
siis ka $v(C_{1}\wedge\dots\wedge C_{i-1}\wedge C_{i}\wedge C_{i+1}\wedge\dots\wedge C_{k})=1$,
kus $C_{i}=l_{i_{1}}\vee\dots\vee l_{i_{l}}$ ja $l_{i_{j}}\equiv-x$.

\proof

$v(C_{1}\wedge\dots\wedge C_{i-1}\wedge C_{i+1}\wedge\dots\wedge C_{k})=1$
ja $v(x)=0$ $\Rightarrow$\\
$v(C_{1})=\dots v(C_{i-1})=v(C_{i+1})=\dots=v(C_{k})=1$ ja $v(l_{i_{j}})=1$ $\Rightarrow$\\
$v(C_{1})=\dots v(C_{i-1})=v(C_{i+1})=\dots=v(C_{k})=1$ ja $v(C_{i})=1$
$\Rightarrow$\\
$v(C_{1}\wedge\dots\wedge C_{i-1}\wedge C_{i}\wedge C_{i+1}\wedge\dots\wedge C_{k})=1$
\proofend

\lemma{Väärade literaalide elimineerimine on korrektne}

Jagame taas tõestuse kahte ossa.

\proofpart{R2p} Olgu $F=C_{1}\wedge\dots\wedge C_{i}'\wedge..\wedge C_{k}$
ja $C_{i}'=l_{i_{1}}\vee\dots\vee l_{i_{j-1}}\vee l_{i_{j+1}}\dots\vee l_{i_{l}}$
ning $l_{i_{j-1}}\equiv x$. Näitame suvalise interpretatsiooni $v$
jaoks, kus $v(x)=0$, kui $v(C_{1}\wedge\dots\wedge C_{i}'\wedge..\wedge C_{k})=1$,
siis ka $v(C_{1}\wedge\dots\wedge C_{i}\wedge..\wedge C_{k})=1$,
kus $C_{i}=l_{i_{1}}\vee\dots\vee l_{i_{j-1}}\vee l_{i_{j}}\vee l_{i_{j+1}}\dots\vee l_{i_{l}}$.

\proof

$v(C_{1}\wedge\dots\wedge C_{i}'\wedge..\wedge C_{k})=1$ ja $v(x)=0$
$\Rightarrow$\\
$v(C_{1})=\dots v(C_{i}')=\dots=v(C_{k})=1$ ja $v(l_{i_{j}})=0$
$\Rightarrow$\\
$v(C_{1})=\dots v(C_{i}')=\dots=v(C_{k})=1$ ja $v(C_{i})=v(C_{i}'\vee l_{i_{j}})=1$
$\Rightarrow$\\
$v(C_{1}\wedge\dots\wedge C_{i}\wedge..\wedge C_{k})=1$ \proofend

\proofpart{R2n} Olgu $F=C_{1}\wedge\dots\wedge C_{i}'\wedge..\wedge C_{k}$
ja $C_{i}'=l_{i_{1}}\vee\dots\vee l_{i_{j-1}}\vee l_{i_{j+1}}\dots\vee l_{i_{l}}$
ning $l_{i_{j-1}}\equiv-x$. Näitame suvalise interpretatsiooni $v$
jaoks, kus $v(x)=1$, kui $v(C_{1}\wedge\dots\wedge C_{i}'\wedge..\wedge C_{k})=1$,
siis ka $v(C_{1}\wedge\dots\wedge C_{i}\wedge..\wedge C_{k})=1$,
kus $C_{i}=l_{i_{1}}\vee\dots\vee l_{i_{j-1}}\vee l_{i_{j}}\vee l_{i_{j+1}}\dots\vee l_{i_{l}}$.

\proof

$v(C_{1}\wedge\dots\wedge C_{i}'\wedge..\wedge C_{k})=1$ ja $v(x)=1$
$\Rightarrow$\\
$v(C_{1})=\dots v(C_{i}')=\dots=v(C_{k})=1$ ja $v(l_{i_{j}})=0$
$\Rightarrow$\\
$v(C_{1})=\dots v(C_{i}')=\dots=v(C_{k})=1$ ja $v(C_{i})=v(C_{i}'\vee l_{i_{j}})=1$
$\Rightarrow$\\
$v(C_{1}\wedge\dots\wedge C_{i}\wedge..\wedge C_{k})=1$ \proofend

\lemma{Väärtustusreeglid on korrektsed}

\proof

Reeglid \emph{R3p, R3n} on triviaalselt korrektsed.
\proofend

\lemma{Ühikklausli väärtustamine on korrektne}

\proof

Tegemist on väärtustusreeglite erijuhuga. \proofend

\chapter{Otsing}

Praktilistes rakendustes huvitab meid mingi valemi $F$ muutujate
väärtustus $v$, millel ta tõene on. Selleks tuleb koostada protseduur,
mis konstrueerib valemi tuletuspuid. Seda ülesannet võime vaadelda
kui otsingut olekute ruumis. Sellisel juhul peab otsinguprotseduur
rakendama tuletusreegeid tagurpidi, st. altpoolt ülespoole. See tähendab,
et igal sammul tuleks proovida rakendada reeglit, mida rakendada saab.
Samme tuleks teha nii kaua, kuni oleme saanud tühivalemi $\emptyset$.
Korjates kokku väärtustusreeglite rakendamisel saadud muutujate väärtused,
võime lõpuks õelda, millisel muutujate väärtustusel oli antud valem
tõene. Järgnevalt proovime leida näites kasutatud valemi $F$ muutujate
väärtustuse, millel see valem on tõene.

Olgu antud valem $F=C_{1}\wedge C_{2}\wedge C_{3}$, kus $C_{1}=A\vee B\vee\neg C$,
$C_{2}=\neg A\vee\neg B$ ja $C_{3}=C\vee\neg B$. Rakendame teisendusi
sellele valemile, et leida kehtestav väärtustus.

\begin{enumerate}
\item Väärtustusreegli kasutamine $A$ järgi, saame valemi $F|_{A=1}$.
Kõigepealt rakendame lihtsustavaid teisendusi valemile $F|_{A=1}$:

\begin{enumerate}
\item Elimineerime tõese klausli $C_{1}$.
\item Elimineerime väära literaali $\neg A$ klauslist $C_{2}$, saame valemi
$F'|_{A=1}={C'}_{2}\wedge C_{3}$, ${C'}_{2}=\neg B$.
\end{enumerate}
\item Valemile $F'|_{A=1}$ rakendame nüüd ühe literaaliga klausli elimineerimist
klausli $C'_{2}$ järgi. Saame valemi $F''|_{A=1,B=0}$. Kohe saame
ka valemit lihtsustada, eemaldades klausli $C_{3}.$ Kuna tulemuseks
on tühivalem, mis on tõene valem, siis võime otsingu lõpetada. Otsingu
tuleusena oleme saanud osalise väärtustuse $v_{o}$, kus $v_{o}(A)=1$
ja $v_{o}(B)=0$. On lihtne kontrollida, et esialgne valem on antud
väärtususes tõene sõltumata $C$ väärtusest.
\end{enumerate}
Nüüd oleme valmis formuleerima otsingualgoritmi valemi kehtestatavuse
kindlakstegemiseks.


\section{Otsinguprotseduur}

Ülalkirjeldatud teisendusreeglite rakendamiseks saame kirja panna
otsinguprotseduuri. Protseduuri paneme kirja kui rekursiivse otsingu,
mille eesmärgiks on leida etteantud valemi muutujate tõene väärtustus.
Protseduur \proc{OP} (otsinguprotseduur) saab sisendina valemi $F$.
Protseduuri esimeses sammus kontrollime valemi triviaalset tõesust/väärust
vastavalt sellele, kas valem on tühivalem või sisaldab tühiklauslit.
Kui valemi tõesuse üle ei saa veel otsustada, siis rakendame hargnemisreeglit.
Kõigepealt üritame kasutada reegli erikuju, ühe literaaliga klausli
elimineerimist. Kui ühe literaaliga klauslit valemis ei esine, rakendame
üldist hargnemisreeglit mingi valemi muutuja $x$ järgi, kusjuures
muutuja valime protseduuri \proc{VM} (vali muutuja) järgi. Kuivõrd
üldine väärtustusreegel jagab otsingu kaheks, siis jätkame kõigepealt
esimese haru järgi \emph{(R3p)}. Kui see ei anna tulemust, siis jätkame teise haru \emph{(R3n)}
järgi. Enne kumbagi harusse suundumist rakendame valemile lihtsustavaid
teisendusi: tõeste klauslite elimineerimist ja väärade literaalide
elimineerimist. Kõik algoritmis rakendatavad teisendused sooritatakse
altpoolt ülespoole.

\begin{algorithm}{OP}
\ainput{konjuktiivsel normaalkujul valem $F$.}
\aoutput{tulemus, kas valem $F$ on kehtestatav või mitte.}
\abody

\begin{enumerate}
\item Kui $F=\emptyset$, siis tagasta ``kehtestatav'' ja lõpeta töö.
\item Kui $\emptyset\in F$, siis tagasta ``mittekehtestatav'' ja lõpeta
töö.
\item Kui leidub ühikliteraaliga klausel $C_{i}=l$ valemis $F$, siis

\begin{enumerate}
\item Kui $l\equiv x$, siis rakenda valemile $F$ teisendust \emph{R4p},
saades valemi $F|_{x=1}$. Sellele valemile rakenda järjest lihtsustavaid
teisendusi \emph{R1p} ja \emph{R2n}, kuni neid rohkem rakendada ei
saa. Lihtsustamiste tulemusena saadi valem $F'$. Tagasta \mbox{\proc{OP}$(F')$}
tulemus.
\item Kui $l\equiv-x$, siis rakenda valemile $F$ teisendust \emph{R4n},
saades valemi $F|_{x=0}$. Sellele valemile rakenda järjest lihtsustavaid
teisendusi \emph{R1n} ja \emph{R2p}, kuni neid rohkem rakendada ei
saa. Lihtsustamiste tulemusena saadi valem $F'$. Tagasta \mbox{\proc{OP}$(F')$}
tulemus. 
\end{enumerate}
\item Kui kolmandas sammus ei leidunud ühe literaaliga klauslit, siis rakendame
väärtustusreeglit ühte \emph{R3p, R3n} (proovime mõlemat). Need reeglid
eeldavad ühe muutuja $x$ valikut. Muutuja valik toimub taas protseduuriga
\mbox{$x=$\proc{VM}$(F)$}. Rakendades ühte neist reeglitest, saame valemi $F'$.
Tagasta \mbox{\proc{OP}$(F')$} tulemus.
\end{enumerate}
\end{algorithm}

Hetkel teostame muutuja valiku sammudel (3) ja (4) juhuslikul. Selleks
võib olla näiteks kõige ``vasakpoolsem'' muutuja. Artiklis \cite{key-10}
on põhjalikult vaadeldud muutuja valiku põhimõtteid DPLL protseduuris
ja sarnastes algoritmides. Teatud valemite klassi jaoks võib sobiva
põhimõtte järgi valitud muutuja anda head heuristilist väärtust, mis
vähendab protseduuri keskmist tööaega.

\begin{algorithm}{VM}
\ainput{konjuktiivsel normaalkujul valem $F$.}
\aoutput{mingi valemis $F$ esinev muutujasümbol.}
\abody

\begin{enumerate}
\item Tagasta juhuslikult valitud muutuja valemist $F$.
\end{enumerate}
\end{algorithm}


\section{Otsinguprotseduuri korrektsus}

Nüüd näitame, et ülalkirjeldatud algoritm \proc{OP} on korrektne
kehtestatavuse suhtes.

\theorem{kui otsinguprotseduur \textsf{OP} tagastab konjuktiivsel normaalkujul valemi $F$ korral
``kehtestatav'', siis on valem $F$ kehtestatav.}

\proof otsinguprotseduur koostab valemi $F$ tuletuspuu ja kasutab
ainult tuletusreegleid \emph{R1p, R1n, R2p, R2n, R3p, R3n, R4p} ja \emph{R4n}.
Vastavalt lemmadele (1), (2), (3) ja (4)
säilitavad need reeglid valemi \mbox{kehtestatavuse. \proofend}


\section{Otsinguprotseduuri termineeruvus}

\theorem{otsinguprotseduur \proc{OP} termineerub}

\proof On selge, et otsinguprotseduur \proc{OP} lõpetab töö, kui
valem on tühivalem või sisaldab tühiklauslit (vastavalt sammus (1) ja
(2)). Termineeruvuse näitamiseks piisab näidata, et protseduuri \proc{OP}
rakendamisel $n$-muutujaga valemile toimuvad protseduuri rekursiivsed rakendamised
$k$-muutujalistele valemitele, kus $k<n$. Sellisel juhul väheneb iga rekursiivse
kutsega rangelt valemi muutujate arv. On selge, et lõpliku $n$ korral
ei saa teha mittelõplikku arvu rekursiivseid pöördumisi, jõudmata
$0$-muutujaga valemini, milleks saab olla vaid valem sammus (1)
või valemini, mis sisaldab tühiklauslit (samm (2)).

Kõigepealt näitame, et sammus (3) saadav valem $F$' sisaldab $k<n$
muutujat, kui $F$ sisaldas $n$ muutujat. On selge, et $F$ sisaldab
vähemalt ühte muutujat $x$. Vaatleme juhtu (3.a), kus $C_{i}\in F$,
$C_{i}=l$, $l\equiv x$. Kõik klauslid, mis sisaldavad literaali
$l$, on tõesed ja me eemaldame nad järjestikuse tõeste klauslite
eemaldamise reegli rakendamisega. Selle tulemusena ei ole saadud valemis
ühtegi muutuja $x$ positiivset esinemist. Teisest küljest, kõikidest
klauslitest, milles sisaldaldub $x$ negatiivselt ($\overline{l}\equiv x$),
kustutame kõik $\overline{l}$ esinemised järjestikuse väärade literaalide
elimineerimise reegliga. Seega ei sisalda valem $F'$ enam muutujat
$x$, st. tema muutujate arv on vähenenud ühe võrra. Juhu (3.b) jaoks
on tõestuse käik analoogiline.

Sammus (4) eeldame, et \proc{VM} termineerub, muidu on tõestuse käik
sarnane, kasutades lihtsustavate reeglite järjestrakendamisel muutuja
$x$ kadumist esialgsest valemist. \proofend


\chapter{Lahendite loendamine}

Tähistagu $\mu_{n}(F)$ $n$ väärtustamata muutujaga lausearvutusvalemi $F$ muutujate
tõeste väärtustuste arvu. Meid huvitab efektiivne $\mu_{n}$ arvutamise
protseduur suvalise etteantud valemi $F$ jaoks. Artiklis \cite{key-4}
on näidatud, et selleks saab edukalt kasutada DPLL protseduuri. Meie
formuleerime algoritmi (loendamisprotseduur) siin kirjeldatud loogilise
raamistiku piires ning tõestame selle korrektsuse ning termineeruvuse.

Kõigepealt vaatleme triviaalseid juhte: tühivalemit ja tühiklauslit
sisaldavat valemit. Parameeter $n$ tähistab antud juhul fiktiivseid
muutujaid, mis on küll olemas, aga ei esine valemis. On selge, et
tühivalemi $F=\emptyset$ korral $\mu_{n}(F)={2}^{n}$, sest nendele
$n$ muutujale võib anda suvalise väärtustuse. Seevastu tühiklauslit sisaldava valemi
korral on meil tegemist samaselt väära valemiga ja mingi fiktiivsete
muutujate väärtustus ei saa valemit tõeseks muuta. Seega kui $\emptyset\in F$,
siis $\mu_{n}(F)=0$.

Loendamisprotseduuri ehitame üles eespool kirjeldatud tuletusreegleid
kasutades. Kõigepealt näitame, et tõeste klauslite elimineerimisreeglid
\emph{(R1p, R1n)} säilitavad valemi $F$ lahendite arvu.

\lemma{Tõeste klauslite elimineerimisreegel säilitab
konjuktiivsel normaalkujul valemi $F$ lahendite arvu.}

\proof

Olgu valem $F=C_{1}\wedge\dots\wedge C_{i-1}\wedge C_{i+1}\wedge\dots\wedge C_{k}$.
Pärast tõeste klauslite elimineerimisreegli ülalt alla rakendamist
saame valemi $F'=C_{1}\wedge\dots\wedge C_{i-1}\wedge C_{i}\wedge C_{i+1}\wedge\dots\wedge C_{k}$,
kus $C_{i}=l_{i_{1}}\vee\dots\vee l_{i_{l}}$ ja $l_{i_{j}}\equiv x$.
Lisaks on meil kitsendus $v(x)=\delta$, $\delta\in\{0,1\}$ mingi
muutuja $x$ jaoks. On selge, et $\mu_{n}(F)=\mu_{n}(F')$ (1) parajasti
siis, kui iga $F$ lahend on $F'$ lahendiks ja iga $F'$ lahend on
ka $F$ lahendiks. Vastavalt lemmale (1) teame, et iga $F$ lahend
on ka $F'$ lahend. Siin näitame, et iga $F'$ lahend on ka $F$ lahend.

Olgu $v$ suvaline interpretatsioon, kus $v(F')=1$. Siis

$v(C_{1}\wedge\dots\wedge C_{i-1}\wedge C_{i}\wedge C_{i+1}\wedge\dots\wedge C_{k})=1$ $\Rightarrow$\\
$v(C_{1})=\dots v(C_{i-1})=v(C_{i})=v(C_{i+1})=\dots=v(C_{k})=1$ $\Rightarrow$\\
 $v(C_{1}\wedge\dots\wedge C_{i-1}\wedge C_{i+1}\wedge\dots\wedge C_{k})$. \proofend

\lemma{Väärade literaalide elimineerimisreegel säilitab
konjuktiivsel normaalkujul valemi $F$ lahendite arvu.}

Olgu valem $F=C_{1}\wedge\dots\wedge C_{i-1}\wedge C_{i}'\wedge C_{i+1}\wedge\dots\wedge C_{k}$, kus
$C_{i}'=l_{i_{1}}\vee\dots\vee l_{i_{j-1}}\vee l_{i_{j+1}}\dots\vee l_{i_{l}}$. Sellele valemile
rakendame reeglit \emph{R2p} või \emph{R2n} ja saame valemi $F'$. Sarnaselt lemmale (5)
näitame ka siin, et iga $F$ lahend on $F'$ lahend ja vastupidi.

\proofpart{R2p} Pärast reegli \emph{R2p} ülalt alla rakendamist saame valemi
$F'=C_{1}\wedge\dots\wedge C_{i-1}\wedge C_{i}\wedge C_{i+1}\wedge\dots\wedge C_{k}$, kus
$C_{i}=l_{i_{1}}\vee\dots\vee l_{i_{j-1}}\vee l_{i_{j}}\vee l_{i_{j+1}}\dots\vee l_{i_{l}}$ ja
$l_{i_j}\equiv x$.

\proof

Vastavat lemmale (2, \emph{R2p}) on iga $F$ lahend, kus $v(x)=0$, $F'$ lahend. Näitame,
et iga $F'$ lahend, kus $v(x)=0$, on ka $F$ lahend.

$v(C_{1}\wedge\dots\wedge C_{i-1}\wedge C_{i}\wedge C_{i+1}\wedge\dots\wedge C_{k})=1$ ja $v(x)=0$ $\Rightarrow$\\
$v(C_{1})=\dots v(C_{i-1})=v(C_{i})=v(C_{i+1})=\dots=v(C_{k})=1$ ja $v(l_{i_{j}})=0$ $\Rightarrow$\\
$v(C_{1})=\dots v(C_{i-1})=v(C_{i+1})=\dots=v(C_{k})=1$ ja $v(C_i\vee l_{i_{j}})=1$ $\Rightarrow$\\
$v(C_{1})=\dots v(C_{i-1})=v(C_{i+1})=\dots=v(C_{k})=1$ ja $v(C_{i}')=1$ $\Rightarrow$\\
$v(C_{1})=\dots v(C_{i-1})=C_{i}'=v(C_{i+1})=\dots=v(C_{k})=1$. \proofend

\proofpart{R2n} Pärast reegli \emph{R2n} ülalt alla rakendamist saame valemi
$F'=C_{1}\wedge\dots\wedge C_{i-1}\wedge C_{i}\wedge C_{i+1}\wedge\dots\wedge C_{k}$, kus
$C_{i}=l_{i_{1}}\vee\dots\vee l_{i_{j-1}}\vee l_{i_{j}}\vee l_{i_{j+1}}\dots\vee l_{i_{l}}$ ja
$l_{i_j}\equiv -x$.

\proof

Vastavat lemmale (2, \emph{R2n}) on iga $F$ lahend, kus $v(x)=1$, $F'$ lahend. Näitame,
et iga $F'$ lahend, kus $v(x)=1$, on ka $F$ lahend.

$v(C_{1}\wedge\dots\wedge C_{i-1}\wedge C_{i}\wedge C_{i+1}\wedge\dots\wedge C_{k})=1$ ja $v(x)=1$ $\Rightarrow$\\
$v(C_{1})=\dots v(C_{i-1})=v(C_{i})=v(C_{i+1})=\dots=v(C_{k})=1$ ja $v(l_{i_{j}})=0$ $\Rightarrow$\\
$v(C_{1})=\dots v(C_{i-1})=v(C_{i+1})=\dots=v(C_{k})=1$ ja $v(C_i\vee l_{i_{j}})=1$ $\Rightarrow$\\
$v(C_{1})=\dots v(C_{i-1})=v(C_{i+1})=\dots=v(C_{k})=1$ ja $v(C_{i}')=1$ $\Rightarrow$\\
$v(C_{1})=\dots v(C_{i-1})=C_{i}'=v(C_{i+1})=\dots=v(C_{k})=1$. \proofend

Eespool vaadeldud reeglid \emph{R1p, R1n, R2p, R2n} sobivad hästi ka lahendite
loendamise protseduuri jaoks, sest nad aitavad oluliselt lihtsustada valemit, parandades
sellega algoritmi kiirusomadusi. Lisaks säilitavad nad lahendite arvu, mis on vajalik
loendamisalgoritmi korrektsuseks.

Me saame kasutada ka reegleid \emph{R3p} ja \emph{R3n}, aga mitte viisil,
nagu me tegime seda otsinguprotseduuris \proc{OP}.
Olgu antud mingi $n$ väärtustamata muutujaga valem $F$ ja mingi muutuja $x$
(esineb valemis $F$ või on fiktiivne). Seejärel rakendame reegleid \emph{R3p} ja \emph{R3n} alt üles muutuja $x$
järgi valemile $F$ ning saame tulemuseks vastavalt valemid $F|_{x=1}$ ja $F|_{x=0}$. Me
näitame, et $\mu_n(F)=\mu_{n-1}(F|_{x=1})+\mu_{n-1}(F|_{x=0})$.

\lemma{$n$ väärtustamata muutujaga valemi $F$ ja valemis esineva muutuja $x$ jaoks kehtib võrratus
$\mu_n(F)=\mu_{n-1}(F|_{x=1})+\mu_{n-1}(F|_{x=0})$.}

\proof
Väärtustuste hulk jaguneb kaheks mittelõikuvaks hulgaks, mõlemas summa
pooles väärtustame ühe muutuja ära. \proofend

Järgnevad tõestused väärtustusreeglite erikujude \emph{R4p} ja \emph{R4n} jaoks.

\lemma{(R4p) $n$ väärtustamata muutujaga valemi $F$ jaoks, mis sisaldab ühikklauslit $C_i$, $C_i=l$, $l\equiv x$, kehtib
võrratus $\mu_n(F)=\mu_{n-1}(F|_{x=1})$.}

\proof
Vastavalt lemmale (7) kehtib võrratus \mbox{$\mu_n(F)=\mu_{n-1}(F|_{x=1})+\mu_{n-1}(F|_{x=0})$ (1)}.
Kuivõrd $F|_{x=0}$ korral $v(l)=0$ ja $v(C_i)=0$ ning $v(F|_{x=0})=0$, st. $F|_{x=0}$ on samaselt
väär, siis võrratuse (1) paremas pooles liige $\mu_{n-1}(F|_{x=0})=0$. \proofend

\lemma{(R4n) $n$ väärtustamata muutujaga valemi $F$ jaoks, mis sisaldab ühikklauslit $C_i$, $C_i=l$, $l\equiv -x$, kehtib
võrratus $\mu_n(F)=\mu_{n-1}(F|_{x=0})$.}

\proof
Sarnane lemma (8) tõestusega. Vastavalt lemmale (7) kehtib võrratus \mbox{$\mu_n(F)=\mu_{n-1}(F|_{x=1})+\mu_{n-1}(F|_{x=0})$ (1)}.
Kuivõrd $F|_{x=1}$ korral $v(l)=0$ ja $v(C_i)=0$ ning $v(F|_{x=1})=0$, st. $F|_{x=1}$ on samaselt
väär, siis võrratuse (1) paremas pooles liige $\mu_{n-1}(F|_{x=1})=0$. \proofend

Kasutades eespool defineeritud tuletusreegleid ja nende jaoks kehtivaid võrratusi lahendite arvu kohta,
saame formuleerida järgmise algoritmi lahendite loendamise jaoks.

\section{Loendamisprotseduur}

\begin{algorithm}{LP}
\ainput{konjuktiivsel normaalkujul valem $F$ ja muutujate arv $n$.}
\aoutput{valemi $F$ muutujate väärtustuste arv, millel valem $F$ tõene on.}
\abody

\begin{enumerate}
\item Kui $F=\emptyset$, siis tagasta $2^{n}$.
\item Kui $\emptyset\in F$, siis tagasta $0$.
\item Kui leidub ühikliteraaliga klausel $C_{i}=l$ valemis $F$, siis

\begin{enumerate}
\item Kui $l\equiv x$, siis rakenda valemile teisendust \emph{R4p}, saades
valemi $F|_{x=1}$. Sellele valemile rakenda järjest lihtsustavaid
teisendusi \emph{R1p} ja \emph{R2n}, kuni neid rohkem rakendada ei
saa. Lihtsustamiste tulemusena saadi valem $F'$. Tagasta \mbox{\proc{LP}$(F',n-1)$}
tulemus.
\item Kui $l\equiv-x$, siis rakenda valemile teisendust \emph{R4n}, saades
valemi $F|_{x=0}$. Sellele valemile rakenda järjest lihtsustavaid
teisendusi \emph{R1n} ja \emph{R2p}, kuni neid rohkem rakendada ei
saa. Lihtsustamiste tulemusena saadi valem $F'$. Tagasta \mbox{\proc{LP}$(F',n-1)$}
tulemus.
\end{enumerate}
\item Kui kolmandas sammus ei leidunud ühe literaaliga klauslit, siis valime
kõigepealt muutuja $x$, \mbox{$x=$\proc{VM}$(F)$}. Seejärel rakenda
reegleid \emph{R3p} ja \emph{R3n}, saades vastavad valemid $F|_{x=1}$
ja $F|_{x=0}$. Valemile $F|_{x=1}$ rakenda lihtsustavaid teisendusi
\emph{R1p} ja \emph{R2n}, kuni neid enam rakendada ei saa. Selle tulemusena
saame valemi $F_{p}$. Valemile $F|_{x=0}$ rakenda aga lihtsustavaid
teisendusi \emph{R1n} ja \emph{R2p}, kuni neid enam rakendada ei saa.
Selle tulemusena saame valemi $F_{n}$. Tagasta \mbox{\proc{LP}$(F_p, n-1)$}
$+$ \mbox{\proc{LP}$(F_n, n-1)$}.
\end{enumerate}
\end{algorithm}

\section{Loendamisprotseduuri korrektsus}

\theorem{Kui \proc{LP} tagastab $n$ väärtustamata muutujaga kojuktiivsel normaalkujul valemi $F$ korral täisarvu $t$,
siis on valemi $F$ nendel $n$ muutujal $t$ erinevat väärtustust, mille korral valem $F$ on tõene.}

\proof
Kasutame induktsiooni valemis $F$ esinevate mittefiktiivsete muutujate arvu järgi.

\emph{Baas}. Baasjuhtum jaguneb kaheks.
\begin{enumerate}
\item Kui valem $F$ on tühivalem, siis ei ole temas ühtegi mittefiktiivset muutujat.
Sellele vastab algoritmi \proc{LP} samm (1). Siis on meil vaja väärtustada $n$
fiktiivset muutujat. Nendel muutujatel on $2^n$ erinevat väärtustust. Seega
tagastab algoritm \proc{LP} sammus (1) õige tulemuse.
\item Kui valem $F$ sisaldab tühiklauslit, siis võib temas küll veel mittefiktiivseid
muutujaid. Sellele vastab algoritmi \proc{LP} samm (2). Kuivõrd tühiklauslit 
sisaldav valem $F$ on samaselt väär, siis on tema lahendite arv 0.
\end{enumerate}

\emph{Samm}. Sammu sooritamisel eeldame, et iga $0\le k<m$ korral tagastab \proc{LP}
$n$ väärtustamata muutujaga valemi $F$ korral, milles on $k$ mittefiktiivset muutujat, korrektse tulemuse
$t$.

Olgu valemis $F$, mis ei ole tühivalem ega sisalda tühiklauslit, $m$ mittefiktiivset muutujat. Rakendame protseduuri \proc{LP}
valemile $F$. Ilmselt sooritatakse kõigepealt samm (3) või (4).

\begin{enumerate}
\item Sammu (3) rakendamise korral on meil $C_i\in F$, kus $C_i=l$ ja $l\equiv x$ või $l\equiv -x$.
Rakendades järjest lihtsustavaid teisendusi, eemaldatakse valemist muutuja $x$ positiivsed ning
negatiivsed esinemised. On selge, et pärast lihtsustamist saadud valemis $F'$ on $m'$, $m'<m$ mittefiktiivset
muutujat. Tagastame \mbox{\proc{LP}$(F',n-1)$}, mis lemma (8) (juhul $l\equiv x$) või
lemma (9) (juhul $l\equiv -x$) ning sammu eelduse tõttu on korrektne tulemus.
\item Sammu (4) rakendamise korral valib protseduur \proc{VM} valemist $F$ ühe
mittefiktiivse muutuja. Rakendades kõigepealt väärtustusreegleid \emph{R3p} ja \emph{R3n}
ning seejärel lihtsustavaid teisendusi, saame vastavad valemid $F_p$ ja $F_n$. Sarnaselt
eelmise juhuga, eemaldatakse lihtsustamise käigus valemitest vähemalt
üks mittefiktiivne muutuja. Tagastame
\mbox{\proc{LP}$(F_p, n-1)$} $+$ \mbox{\proc{LP}$(F_n, n-1)$}, mis lemma (7) ja sammu
eelduse tõttu on korrektne tulemus. \proofend
\end{enumerate}

\section{Loendamisprotseduuri termineeruvus}

\theorem{Loendamisprotseduur \proc{LP} termineerub}

\proof
Täiesti sarnane teoreemi (2) tõestusega (\proc{OP} termineeruvus). Igal algoritmi \proc{LP} rekursiivsel
pöördumisel termineerub \proc{LP} sammus (1) või (2) või vähendatakse sisendvalemis esinevate
muutujate arvu vähemalt ühe võrra, rakendades järjest lihtsustusreegleid sammus (3) või (4). \proofend


 \chapter*{Kokkuvõte}
 \addcontentsline{toc}{chapter}{\numberline{}Kokkuvõte}

Töös vaatlesime lausearvutusvalemi lahendite loendamise probleemi ning selle lahendamiseks
kasutatavat algoritmi, mis on saadud valemi kehtestatavuse algoritmi sobivalt täiendades.
Kõigepealt koostasime tuletusreeglid kehtestatavate valemite tuletamiseks ning näitasime
nende reeglite korrektsuse kehtestatavuse suhtes. Pärast seda
oli juba lihtne ehitada üles kehtestatavuse kindlakstegemise algoritmi, mis
kasutas otsingut ja eespool kirjeldatud tuletusreegleid. Otsinguprotseduuri omadustest
tõestasime korrektsuse kehtestatavuse suhtes ja termineeruvuse.

Lõpuks täiendasime kehtestatavuse algoritmi lahendite loendamise jaoks. Samal
ajal näitasime, kuidas käitub valemi lahendite arv, rakendades sellele mõnda
kirjeldatud teisendusreeglitest. See võimaldas anda tõestuse
loendamisprotseduuri korrektsusele.

Töö sisaldab lisadena nii otsinguprotseduuri kui loendamisalgoritmi realisatsioone
keeles Prolog. Prolog võimaldas kasutatavad sümbolteisendused lakooniliselt
ja selgelt arvutiprogrammina realiseerida.

% Resümee

\chapter*{Using DPLL procedure to count models}
\addcontentsline{toc}{chapter}{\numberline{}Using DPLL procedure to count models}


{\textbf{\Large Bachelor thesis (6 ECTS credits)}}

\vspace{1cm}

{\textbf{\Large Raivo Laanemets}}

\vspace{1cm}

{\textbf{\Large Abstract}}

\vspace{1cm}

In this thesis we present logical inference system for satisfiable
boolean formulas. The rules of this system resemble the steps of well-know
DPLL algorithm for the boolean satisfiability problem. The proof of correctness
of this system is given. Then we build up SAT procedure that
uses exactly these rules. The correctness and termination of this
procedure is given. Finally we show how to extend the given SAT procedure
to effectively count the satisfying assignments. At the very end, implementations
of both SAT and counting procedure in Prolog are given.

% Resümee lõpp

% Kirjandus

\addcontentsline{toc}{chapter}{Kirjandus}
\begin{thebibliography}{20}

\bibitem[Davis60]{key-1}M.Davis, H.Putnam, \emph{A Computing Procedure for
Quantification Theory},
J. Assoc. Comput. Mach., 7:201-215, 1960.

\bibitem[Davis62]{key-2}M.Davis, G.Logemann, D.Loveland, \emph{A Machine Program
for Theorem-Proving},
Communications of the ACM, 4:394-397, 1962.

\bibitem[Birnbaum99]{key-4}E.Birnbaum, E.L.Lozinskii, \emph{The Good Old Davis-Putnam
Procedure Helps Counting Models},
Journal of Artifical Intelligence Research 10, 1999, p. 457-477.

\bibitem[Gomes06]{key-5}C.P.Gomes, A.Sabharwal, B.Selman, \emph{Model Counting:
A New Strategy for Obtaining Good Bounds},
AAAI 2006.

\bibitem[Sang05]{key-6}T.Sang, P.Beame, H.Kautz, \emph{Solving Bayesian Networks
by Weighted Model Counting},
AAAI 2005: 475-482.

\bibitem[Roth96]{key-8}D.Roth, \emph{On the hardness of approximate reasoning},
Artifical Intelligence, 82(1/2):273-302, 1996.

\bibitem[Moskewicz01]{key-9}M.W.Moskewicz, C.F.Madigan, Y.Zhao, L.Zhang, S.Malik,
\emph{Chaff: Engineering an Efficent SAT Solver},
In Proceedings of the 38th Design Automation Conference (DAC'01), June 2001.

\bibitem[Lagoudakis01]{key-10}M.G.Lagoudakis, M.L.Littman, \emph{Learning to Select
Branching Rules in the DPLL Procedure for Satisfiability},
Electronic Notes in Discrete Mathematics (ENDM), Vol. 9.
\end{thebibliography}

% Siit algavad tööga kaasa pandud lisad

\chapter*{Lisa 1\\
	Otsinguprotseduuri realisatsioon}

% Otsinguprotseduuri kood. Enne koodi lisamist siia tuleb
% see teisendada iso-8859-15 kodeeringusse.
% Teisendamine;
% iconv -f utf-8 -t iso-8859-15 sisendfail > väljundfail

% Määra reavaheks 1.0

\setstretch{1.0}

% Lisa kood
\lstinputlisting{op_iso.pl}

% Vii reavahe tagasi 1.5 peale

\onehalfspacing

\chapter*{Lisa 2\\
	Loendamisprotseduuri realisatsioon}

\setstretch{1.0}
\lstinputlisting{lp_iso.pl}
\onehalfspacing



\end{document}
