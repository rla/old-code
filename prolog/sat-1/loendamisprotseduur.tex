%% LyX 1.4.4 created this file.  For more info, see http://www.lyx.org/.
%% Do not edit unless you really know what you are doing.
\documentclass[estonian]{article}
\usepackage[latin1]{inputenc}

\makeatletter

%%%%%%%%%%%%%%%%%%%%%%%%%%%%%% LyX specific LaTeX commands.
%% Bold symbol macro for standard LaTeX users
\providecommand{\boldsymbol}[1]{\mbox{\boldmath $#1$}}


\usepackage{babel}
\makeatother
\begin{document}
\begin{enumerate}
\item Kui $F=\emptyset$, siis tagasta $2^{n}$.
\item Kui $\emptyset\in F$, siis tagasta $0$.
\item Kui leidub ühikliteraaliga klausel $C_{i}=l$ valemis $F$, siis

\begin{enumerate}
\item Kui $l\equiv x$, siis rakenda valemile teisendust \emph{R4p}, saades
valemi $F|_{x=1}$. Sellele valemile rakenda järjest lihtsustavaid
teisendusi \emph{R1p} ja \emph{R2n}, kuni neid rohkem rakendada ei
saa. Lihtsustamiste tulemusena saadi valem $F'$. Tagasta \mbox{\proc{LP}$(F',n-1)$}
tulemus.
\item Kui $l\equiv-x$, siis rakenda valemile teisendust \emph{R4n}, saades
valemi $F|_{x=0}$. Sellele valemile rakenda järjest lihtsustavaid
teisendusi \emph{R1n} ja \emph{R2p}, kuni neid rohkem rakendada ei
saa. Lihtsustamiste tulemusena saadi valem $F'$. Tagasta \mbox{\proc{LP}$(F',n-1)$}
tulemus.
\end{enumerate}
\item Kui kolmandas sammus ei leidunud ühe literaaliga klauslit, siis valime
kõigepealt muutuja $x$, \mbox{$x=$\prox{VM}$(F)$}. Seejärel rakenda
reegleid \emph{R3p} ja \emph{R3n}, saades vastavad valemid $F|_{x=1}$
ja $F|_{x=0}$. Valemile $F|_{x=1}$ rakenda lihtsustavaid teisendusi
\emph{R1p} ja \emph{R2n}, kuni neid enam rakendada ei saa. Selle tulemusena
saame valemi $F_{p}$. Valemile $F|_{x=0}$ rakenda aga lihtsustavaid
teisendusi \emph{R1n} ja \emph{R2p}, kuni neid enam rakendada ei saa.
Selle tulemusena saame valemi $F_{n}$. Tagasta \mbox{\proc{LP}$(F_p, n-1)$}
$+$ \mbox{\proc{LP}$(F_n, n-1)$}.
\end{enumerate}

\end{document}
