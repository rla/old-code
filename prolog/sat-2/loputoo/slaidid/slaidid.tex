\documentclass{beamer}
\usepackage[latin1]{inputenc}

\mode<presentation>
{
	\usetheme{default}
	\useoutertheme[]{shadow}
	\usecolortheme{beaver}
}

\setbeamerfont{title}{size=\huge}
\setbeamerfont{frametitle}{size=\large}

\setbeamertemplate{footline}{}

\title{DPLL protseduur lahendite loendamiseks}
\subtitle{Bakalaureusetöö, 4AP}
\author{
	\textit{Autor:} Raivo Laanemets \\
	\textit{Juhedaja:} Tõnu Tamme \\
    Tartu Ülikool \\
    Arvutiteaduse instituut
}
\date{17. jaanuar 2008}

\begin{document}

\frame{\titlepage}

\section{Sissejuhatus}
\subsection{Kehtestatavusprobleem}
\frame
{
	\frametitle{Kehtestatavusprobleem}
	
	
	\begin{itemize}
		\item Antud $n$ muutuja Boole'i funktsioon $f(x_1,\dots,x_n)$.
		\item Leidub selline muutujate $x_1,\dots,x_n$ väärtustus, et
		$f(x_1,\dots,x_n)=1$? ($x_1,\dots,x_n$ on lahend).
		\item Mõnikord soovime teada ka konkreetset $x_1,\dots,x_n$ väärtustust,
		millel $f$ tõene on.
	\end{itemize}
}
\subsection{Loendamisprobleem}
\frame
{
	\frametitle{Loendamisprobleem}
	
	\begin{itemize}
		\item Antud $n$ muutuja Boole'i funktsioon $f(x_1,\dots,x_n)$.
		\item Mitu muutujate $x_1,\dots,x_n$ väärtustust leidub, nii et
		$f(x_1,\dots,x_n)=1$? ($f$ lahendite arv)
	\end{itemize}
}
\subsection{Keerukus}
\frame
{
	\frametitle{Keerukus}
	
	\begin{itemize}
		\item Kehtestatavusprobleem on \emph{NP-täielik} (S.Cook, 1971).
		\item Klassi \emph{NP-täielik} kuuluvad ka:
		sõltumatu tippude hulk, klikkide leidmine, tippude hulga kate (graafiteooria
		probleemid), samuti \emph{Sudoku}, $\dots$
	\end{itemize}
}
\frame
{
	\frametitle{Keerukus}
	
	\begin{itemize}
		\item Lahendite loendamine on kehtestatavusele vastav loendamisprobleem ja on
		\emph{\#P-täielik} (D.Roth, 1996).
		\item Klassi \emph{\#P-täielik} kuuluvad ka:
		tuletus Bayes'i võrgus, Dedekindi arvude leidmine, $\dots$
	\end{itemize}
}
\section{DPLL algoritm}
\subsection{DP algoritm}
\frame
{
	\frametitle{DP algoritm}
	
	\begin{itemize}
		\item M.Davis, H.Putnam, \emph{A Computing Procedure for Quantification Theory}, 1960
		\item Predikaatloogikas kirja pandud teoreemide tõestamiseks.
		\item Sisendiks predikaatloogika valem $F$.
		\item Algoritm asendas $F$ tõesuse kontrolli $\neg F$ mittekehtestatavuse kontrolliga.
	\end{itemize}
}
\frame
{
	\frametitle{DP algoritm}
	
	Algoritm koosnes neljast sammust:
	\begin{enumerate}
		\item Valemi $F$ eituse $\neg F$ leidmine.
		\item Valemi $\neg F$ kirjutamine prefikskujule $G_p$ (kvantorid ''eespool``, maatriks KNK-l).
		\item $G_p$ eksistentsikvantorite eemaldamine skolemiseerimise teel, saadi valem $G_s$.
		\item Eespool saadud valemist kvantori- ja muutujavabade lausearvutusvalemite
		$G_{s,1},G_{s,2},\dots$ genereerimine kuni nendest $n$ esimese valemi
		konjuktsioon $G_{s,1}\wedge\dots\wedge G_{s,n}$ on mittekehtestatav. 
	\end{enumerate}
}
\frame
{
	\frametitle{DP algoritm}
	
	Sammus (4) tehti järgmist:
	\begin{itemize}
		\item Leidub klausel $C=l$ - eemalda kõik klauslid,
		kus esineb $l$ ja kõikidest klauslitest $\bar{l}$ (ühikliteraal).
		\item $l$ esineb mingis klauslis, aga $\bar{l}$ mitte üheski - eemalda
		kõik klauslid, kus $l$ esineb (puhas literaal).
		\item Kui $W=(A\vee l)\wedge(B\vee \bar{l})\wedge R$, $A$,
   		$B$ ja $R$ literaali $l$ ega $\bar{l}$ ei sisalda,
   		kirjuta $W$ kujule $W'=(A\vee B\vee)\wedge R$ (muutujate elimineerimine).
	\end{itemize}
}
\frame
{
	\frametitle{DP algoritm}
	
	\begin{itemize}
		\item Reegleid sammus 4 rakendati kuni saadi tühi klausel
		(klausel on tõene parajasti siis, kui vähemalt üks tema literaalidest on tõene)
		\item Sarnaneb väga resolutiooniga (Robinson, 1965).
		\item DP algoritm oli korrektne ja täielik ($F$ tõene - algoritm tuvastas selle).
	\end{itemize}
}
\subsection{DPLL algoritm}
\frame
{
	\frametitle{DPLL algoritm}
	
	\begin{itemize}
		\item M.Davis, G.Logemann, D.Loveland, \emph{A Machine Program for Theorem-Proving}, 1962
		\item Asendas elimineerimise jaotamisega:
		\item Kui $W$ on kujul $W=(A\vee l)\wedge(B\vee \bar{l})\wedge R$, $A$,
		$B$ ja $R$ literaali $l$ ega $\bar{l}$ ei sisalda, jaota $W$ valemiteks
		$W_1=A\wedge R$ ja $W_2=B\wedge R$.
		\item $W$ on mittekehtestatav parajasti siis kui $W_1$ ja $W_2$ seda on.
	\end{itemize}
}
\subsection{DPLL algoritm}
\frame
{
	\frametitle{DPLL algoritm}
	
	\begin{itemize}
		\item Valem tõene, kui ''kadusid ära`` kõik klauslid.
		\item Otseselt muutujate väärtustust ei anna.
		\item Ei paku mugavat viisi lahendite loendamiseks.
		\item PL teoreemitõestamiseks ebaefektiivne võrreldes
		resolutsiooni ja unifitseerimisega.
	\end{itemize}
}
\subsection{DPLL SAT}
\frame
{
	\frametitle{DPLL SAT}
	
	\begin{itemize}
		\item Kasutatav lausearvutusloogika valemite kehtestatavuse jaoks.
		\item Annab kehtestatava väärtuse kui see leidub.
		\item Jaotamine asendatud väärtustamisega.
		\item Kasutatab Boole-Shannoni dekompositsiooni.
		\item Töötab muutujate osalise väärtustusega.
	\end{itemize}
}
\frame
{
	\frametitle{Boole-Shannoni dekompositsioon}
	
	\begin{itemize}
		\item Universaalne vahend $n$ muutujaga ($n > 0$) valemite
		jagamiseks kaheks $n-1$ valemiks.
		\item Valida valemist $F$ muutuja $x$, saadakse kaks valemit $F_x$ ja $F_{\bar{x}}$,
		esimeses võta $x=1$, teises $x=0$.
		\item Võimaldab ära kasutada DPLL lihtsustusreegleid.
		\begin{itemize}
			\item $F_x$ - eemalda kõik $x$ sisaldavad klauslid, klauslitest eemalda $\neg x$.
			\item $F_{\bar{x}}$ - eemalda kõik $\neg x$ sisaldavad klauslid, klauslitest eemalda $x$.
		\end{itemize}
		\item Vali $F_x$ või $F_{\bar{x}}$ ja jätka dekompositsiooni rakendamist.
	\end{itemize}
}
\frame
{
	\frametitle{Boole-Shannoni dekompositsioon}
	
	\begin{itemize}
		\item Kui valem sisaldab ühikklauslit $C=x$ ($C=\neg x$),
		siis teha dekompositsioon $x$ järgi ja ignoreerida $F_{\bar{x}}$ ($F_x$).
		\item Esimesel juhul on samaselt väär $F_{\bar{x}}$, teisel $F_x$.
	\end{itemize}
}
\frame
{
	\frametitle{Boole-Shannoni dekompositsioon}
	
	\begin{itemize}
		\item Dekompositsiooni saab rakendada ainult lõplik arv kordi, sest muutujad saavad otsa.
		\item Võimalik kaks olukorda vahetult pärast lihtsustamist:
		\begin{enumerate}
			\item[*a] $F$ ei sisalda ühtegi klauslit - leitud kehtestatav väärtustus.
			\item[*b] $F$ mingi klausel muutus tühjaks - $F$ on väär.
		\end{enumerate}
		\item Juhul (*a) on annavad muutujad, mille järgi dekompositsiooni rakendati, kehtestatava väärtustuse.
	\end{itemize}
}
\frame
{
	\frametitle{Boole-Shannoni dekompositsioon}
	
	DPLLSAT($F$):
	\begin{enumerate}
		\item \textbf{if} $F$ tühi, tagasta \textbf{true}.
		\item \textbf{else if} $F$ sisaldab tühiklauslit, tagasta \textbf{false}.
		\item \textbf{else if} $F$ sisaldab ühikklauslit $C=x$, tagasta DPLLSAT($F_x$).
		\item \textbf{else} Vali muutuja $x$, tagasta $F_x\vee F_{\bar{x}}$.
	\end{enumerate}
}
\section{Lahendite loendamine}
\frame
{
	\frametitle{Lahendite loendamine}
	
	\begin{itemize}
		\item E.Birnbaum, E.L.Lozinskii, \emph{The Good Old Davis-Putnam Procedure Helps Counting Models}, 1999
		\item DPLL protseduuri kasutamine loendamiseks.
		\item Kasutab ära, et valem muutub tõeseks enne kõikide muutujate väärtustamist.
		
	\end{itemize}
}
\end{document}
