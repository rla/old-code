\chapter*{Kokkuvõte}
\addcontentsline{toc}{chapter}{Kokkuvõte}

Käesolevas bakalaureusetöös uurisime lausearvutusvalemi kehtestatavate
lahendite loendamise probleemi. Kuna antud probleem on tihedalt seotud
kehtestatavusprobleemiga ja kuivõrd töös vaadeldud algoritm
\textit{DPLL} on aluseks nii loendamise kui kehtestatavuse lahendamiseks, oli
mõistlik neid probleeme koos vaadelda.

Töös esimeses osas esitasime \textit{DPLL} algoritmi üldisel kujul, nagu teda
kasutatakse tänapäeval levinud kõrge efektiivsusega kehtestatavusalgoritmides.
Natuke vaatlesime ka algoritmi originaalkuju ning mehaanilist teoreemide tõestamist -
originaalülesannet, mille lahendamiseks \textit{DPLL} algoritm mõeldud oli.

Töö teises osas andsime algoritmi realisatsiooni tehisintellekti
programmeerimiskeeles Prolog, mis tänu sisseehitatud tagurdusalgoritmile ja
hea sobivuse poolest sümbolarvutuse ülesannete jaoks, sobis suurepäraselt
käesoleva probleemi lahendusalgoritmi implementeerimiseks.

Realiseeritud algoritm on täielik ning seda saab kasutada ka samal otstarbel,
milleks originaalalgoritmi, s.o. tõestusprotseduurides. Implementatsioon on
hõlpsasti integreeritav teiste Prologi programmidega, ning tänu ülevaatlikule
koodile ning realisatsiooniga kaasa tulevale silumisvahendile, kasutatav ka
hariduslikel eesmärkidel \textit{DPLL}-tüüpi algoritmide tutvustamisel.

Paraku ei võimaldanud töö piiratud maht esitada \textit{DPLL} algoritmi mitmeid
edasiarendusi, mis tõstavad lahendusprotseduuri efektiivsust veelgi. Vaatluse
alt jäi välja konfliktide analüüs ja muutuja valiku heuristika, mille abil saab
algoritmi häälestada teatud valdkonnast (näiteks riistvara verifitseerimisega
seotud ülesanded), kus valemid on teatud struktuuriga pärit valemite kiiremaks
lahendamiseks, samuti mitmed võtted lahendite ligikaudseks loendamiseks. Selles
valdkonnas on viimasel ajal toimunud eriti suur areng. Töös pole kajastatud ka
olulisi lahendite loendamise rakendusi, näiteks kaalutud loendamist, mis vastab
tuletusele Bayes'i võrgus, samuti Dedekindi arvude leidmist.
