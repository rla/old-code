\chapter*{Sissejuhatus}
\addcontentsline{toc}{chapter}{Sissejuhatus}

Lausearvutusvalemi lahendite loendamise probleem on tihedalt seotud
kehtestatavuse probleemiga ja omab seetõttu suurt teoreetilist ning
praktilist tähtsust. Mitmed olulised probleemid on teisendatavad
nendele probleemidele. Seetõttu on tähtis nii loendamise kui kehtestatavuse
algoritmide efektiivsus, st. töökiirus. Selles vallas on viimasel ajal
toimunud palju uuringuid ning erinevate lahendusalgoritmide kohta on kirjutatud
kümneid artikleid.

Valemi kehtestatavus kui otsustusprobleem on \textit{NP-täielik} \cite{cook71},
mis paigutab ta samasse keerukuslassi kümnete teiste oluliste probleemidega.

Lahendite loendamise probleem on kehtestatavusele vastav loendamisprobleem.
Tema kuulub keerukusklassi \textit{\#P-täielik} \cite{roth96}. Samuti on
teada, et lahendite arvu isegi ligikaudne leidmine väikese garanteeritud veaga
on arvutuslikult raske ülesanne. Klassis \textit{\#P-täielik} asuvad ka mitmed
tehisintellekti valdkonnast tuntud ülesanded, näiteks tuletus Bayes'i võrgus ja
teised tuletusprobleemid mittetäieliku teadmusega (tõenäosuslikes)
süsteemides.

Kehtestatavuse jaoks on teada palju algoritme. Üheks neist on \textit{DPLL}
\emph{(Davis-Putnam-Logemann-Loveland)} protseduur \cite{davis60,davis62},
mille edasiarendatud variandid on kõrge efektiivsusega lahendusprogrammides
kasutusel ka tänapäeval, rohkem kui 40 aastat pärast originaalalgoritmi
publitseerimist.

Hiljuti selgus, et \textit{DPLL} algoritmi saab küllaltki edukalt täiendada ka
lahendite loendamise jaoks \cite{birnbaum99}.

\section*{Töö ülesehitus ja eesmärgid}

Käesolevas töös vaadeldakse originaalsest \textit{DPLL} algoritmist
saadud lausearvutusvalemite lahendite loendamisalgoritmi.

Töö alguses esitatakse probleemi formaalne kirjeldus ja töös
läbivalt kasutatavate mõistete definitsioonid.

Esimeses osas esitatakse algoritmide kirjeldus ning pseudokood,
selgitatakse lahti kasutatud ideed ja antakse
põhjendus, miks ühe või teise põhimõtte tarvitamine on lubatav ja mõistlik.

Teises osas antakse lihtne realisatsioon
programmerimiskeeles Prolog. Põhiliselt sisaldab see peatükk kasutatud
andmestruktuuride ja protseduuride kirjeldusi ning näiteid nende kasutamisest.

Tööd jääb lõpetama lisaosana kaasa pandud programmi dokumentatsioon, s.o.
koostatud protseduuride sisend/väljundargumentide kirjeldused ning lühike
ülevaade sellest, mida konktreetne protseduur teeb. Programmi masinloetava
lähetkoodi võib leida kaasapandud optiliselt andmekandjalt.

\section*{Definitsioonid}

\emph{Muutujaks} nimetame suvalist sümbolit $x$
fikseeritud sümbolite hulgast. \emph{Literaaliks} $l$ loeme muutujat $x$ ($l\equiv x$) või tema
eitust $-x$ ($l\equiv-x$). \emph{Klausliks} $C$ loeme literaalide disjunktsiooni
$C=l_{1}\vee\dots\vee l_{n}$. \emph{Konjuktiivsel normaalkujul valemiks} (KNK)
$F$ loeme klauslite konjuktsiooni $F=C_{1}\wedge\dots\wedge C_{k}$. Muutuja
$x$ \emph{väärtustuseks} nimetame funktsiooni $v$, $v(x)\in\{0,1\}$. Literaal
$l$ on tõene, kui $l\equiv x$ ja $v(x)=1$ või $l\equiv\neg x$
ja $v(x)=0$. Literaali $l$ on väär, kui $l\equiv x$ ja $v(x)=0$
või $l\equiv-x$ ja $v(x)=1$. Literaali $l\equiv x$ \emph{komplementaariks}
on literaal $\bar{l}\equiv\neg x$ ja literaali $l'\equiv\neg x$ komplementaariks
on $\bar{l'}\equiv x$. Klauslit $C=l_{1}\vee\dots\vee l_{n}$
nimetame väärtustusel $v$ tõeseks ($v(C)=1$), kui $C$ sisaldab
vähemalt ühte tõest literaali, vastasel korral vääraks ($v(C)=0$).
Valemit $F=C_{1}\wedge\dots\wedge C_{k}$ loeme väärtustusel $v$ tõeseks
($v(F)=1$), kui ükski klauslitest $C_{1},\dots,C_{k}$ pole väär, vastasel
korral vääraks ($v(F)=0$).

\emph{n-muutuja Boole'i funktsiooniks} $f$ nimetame
funktsiooni signatuuriga \mbox{$f:\{0,1\}^n\to \{0,1\}$}. On selge, et suvalist
Boole'i funktsiooni saab esitada mingi talle vastava lausearvutuse valemiga.

\textbf{Kehtestatavusprobleem}

Olgu antud $n$-muutuja Boole'i funktsioon $f(x_1,\dots,x_n)$.
Kehtestatavusprobleem küsib, kas leidub\footnotemark[1] selline muutujate
$x_1,\dots,x_n$ väärtustus, kus $f(x_1,\dots,x_n)=1$.

\textbf{Loendamisprobleem}

Olgu antud $n$-muutuja Boole'i funktsioon $f(x_1,\dots,x_n)$. Lahendite
loendamise probleem on loendamisprobleem, mitu erinevat $x_1,\dots,x_n$
väärtustust leidub, kus $f(x_1,\dots,x_n)=1$.

\footnotetext[1]{Praktilistes rakendustes omavad sageli tähtsust ka
tegelikud muutujate väärtused.}
