\chapter{Programmi kasutamine}

Programmi kasutamiseks on vajalik arvutisse paigaldatud Prologi interpretaator.
Programm on küll kirjutatud Swi-Prologi realisatsioonile, kuid on lihtsalt
porditav ka teistele populaarsetele Prologi implementatsioonidele.

\section{Põhiprotseduurid}

\subsection{Lahendite loendamine}

Loendamiseks saab kasutada protseduuri \texttt{dpll\_loenda/2}, mis asub failis
\textit{dpll\_loenda.pl}, ning mis võtab sisendargumendiks konjuktiivsel
normaalkujul listiesituses valemi, kus muutujad on kodeeritud täisarvudena.
Lahendite arv antakse väljundmuutujas, mis on
protseduuri teiseks argumendiks.

\textbf{Näide kasutamisest:}

\begin{verbatim}
?- dpll_loenda([[-1, 2, -3], [-2, -4], [4]], C).
C = 3
\end{verbatim}

\subsection{Kehtestava väärtustuse leidmine}

Valemi kehtestatavuse leidmise protseduur asub failis \textit{dpll.pl}.
Põhiprotseduuriks on \texttt{dpll/2}, mis võtab sisendiks valemi samal
kujul nagu loendamiseks kasutatav protseduur. Väljundiks antakse muutujate
väärtustuse termi muutujate osa (andmestruktuuri kirjeldus on esitatud eespool).

Kui mingi muutuja väärtustus pole vajalik kehtestatavuse jaoks, siis esineb
termis sellel kohal Prologi muutuja, st. muutuja väärtust ei kitsendata.

Tagurdamisel genereeritakse automaatselt järgmine väärtustus. Kui valem ei
olnud kehtestatav, siis protseduuri kutse lõpetab negatiivse tulemusega.

\textbf{Näited päringutest:}

\begin{verbatim}
?- dpll([[-1, 2, -3], [-2, -4], [4]], C).
C = v(1, 0, 0, 1) ;
C = v(0, 0, 1, 1) ;
C = v(0, 0, 0, 1) ;

?- dpll([[-1, 2, -3], [-2, -4]], C).
C = v(1, 1, _G778, 0)

?- dpll([[-1], [1]], C).
No

\end{verbatim}

Teine võimalus on kasutada integratsioonimoodulit \texttt{dpllp/2} (failis
\textit{dpllp.pl}), mis võimaldab muutujatena tarvitada Prologi muutujaid
(\texttt{dpllk/2}). Sellisel viisil saab siduda lahendaja teiste Prologi
programmidega. Lisaks sisaldab moodul võimalust kasutada mittekonjuktiivsel
normaalkujul olevaid valemeid (\texttt{dpllp/2}). Kasutatavad operaatorid on
prioriteedi kasvamise järjekorras: \texttt{<=>,=>,v,\&,-}.

\textbf{Näited päringutest:}

\begin{verbatim}
?- dpllk([[-A, B, -C], [-B, -D], [D]]).
A = 1,
B = 0,
C = 0,
D = 1

?- dpllp((-A v B v -C) & (B => -D) & D).

A = 1,
B = 0,
C = 0,
D = 1
\end{verbatim}

\section{cnf failide sisselugemine}

Suuremate valemite (rohkem kui kümned muutujad) hoidmiseks on populaarsed
\texttt{cnf} formaadis failid. \texttt{cnf} formaadis faile saab
sisse lugeda päringuga \texttt{loe\_cnf}, mille esimeseks argumendiks on loetava
faili nimi ning teiseks argumendiks saadav valem.

\textbf{Näide kasutamisest:}

\begin{verbatim}
?- loe_cnf('../testid/ais/ais6.cnf', F), dpll_loenda(F, C).
F = [[1, 2, 3, 4, 5, 6], [-1, -2], [-1, -3],
  [-1, -4], [-1, -5], [-1, -6], [-2, -3], [-2|...], [...|...]|...],
C = 24
\end{verbatim}

\section{Silumisvahend}

Programmi koostamisel oli oluliseks abivahendiks silumismoodul \texttt{silur}.
Silumismoodulit toetavad kõik eespool esitatud lahendus- ja
loendamisprotseduurid. Kui silumine on sisse lülitatud (seda saab teha
päringuga \texttt{silur:sisse.}), siis kirjutab programm kasutajale väljundisse
infot teostatavate tegevuste kohta lahendussammude kaupa. Vaikimisi on
silumisvahend välja lülitatud.

\textbf{Näide siluri väljundist:}

\begin{verbatim}
?- dpll([[-1, 2, -3], [-2, -4], [4]], C).
0 - Valem: [[2, -1, -3], [-2, -4], [4]]
0 > Tõeseks valitud ühikliteraal 4
|:
1 - Valem: [[2, -1, -3], [-2]]
1 > Tõeseks valitud ühikliteraal -2
|:
2 - Valem: [[-1, -3]]
2 > Tõeseks valitud literaal 1
|:
3 - Valem: [[-3]]
3 > Tõeseks valitud ühikliteraal -3
|:
4 - Valem: []
C = v(1, 0, 0, 1)
Yes
\end{verbatim}
