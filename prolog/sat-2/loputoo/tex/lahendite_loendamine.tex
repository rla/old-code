% Dokumendi klass, fondi suurus ja keel

\documentclass[12pt,estonian]{report}

% Kasutatavad paketid

\usepackage[estonian]{babel}
\usepackage[latin1]{inputenc}
\usepackage{setspace}
\usepackage{listings}
\usepackage{latexsym}
\usepackage{amsmath}
\usepackage{float}
\usepackage{amssymb}
\usepackage{geometry}
\usepackage{verbatim}
\usepackage{algorithmic}

% Kasutatav font "Computer Modern"

\fontfamily{cm}

% Makrodes kasutatavad loendurid
% Lemmade, teoreemide ja algoritmide jaoks

\newcounter{lemma_counter}\setcounter{lemma_counter}{1}
\newcounter{theorem_counter}\setcounter{theorem_counter}{1}
\newcounter{algorithm_counter}\setcounter{algorithm_counter}{1}
\newcounter{example_counter}\setcounter{example_counter}{1}

% Kasutatavad makrod
% Lemma väljastamine

\newcommand{\lemma}[1]{
	\hspace{1cm}\textbf{Lemma \arabic{lemma_counter}\stepcounter{lemma_counter}. \textit{#1}}
}

% Teoreemi väljastamine

\newcommand{\theorem}[1]{
	\hspace{1cm}\textbf{Teoreem
	\arabic{theorem_counter}\stepcounter{theorem_counter}. \textit{#1}} }

% Tõestuse alguse tähistamine kirjaga "Tõestus:"

\newcommand{\proof}[0]{
	\hspace{1cm}\textbf{Tõestus.}
}

% Tõestuse jagamine osadeks, osa "nimi"

\newcommand{\proofpart}[1]{
	\hspace{1cm}\textbf{#1. }
}

% Tõestuse lõpumärk

\newcommand{\proofend}[0]{\(\square\)}

% Protseduuri nimi erikirjas

\newcommand{\proc}[1]{\textsf{#1}}

% Horisontaalne joon

\newcommand{\horline}[0]{
	\vspace{1ex}\hrule width\columnwidth\vspace{1ex}
}

% Keskkond algoritmi väljastamiseks
% Kasuta kui \begin{algorithm}{OP}

\newenvironment{algorithm}[1]{
	\vspace{0.5cm}
	\textbf{\large{Algoritm
	\arabic{algorithm_counter}}\stepcounter{algorithm_counter}}. \proc{#1}
	\horline
	}{
	\vspace{0.5cm}
	\horline
	\vspace{0.5cm}
}

\newcommand{\ainputoutput}[2]{
	\hskip 2cm\llap{\textbf{Sisend:}} #1\\
	\null\hskip 2cm\llap{\textbf{Väljund:}} #2
}

\newcommand{\abody}[0]{\horline\vspace{0.5cm}}

% "Listing" keskkonna parameetrid

\lstset{
	language=Prolog,
	basicstyle=\footnotesize\tt,
	commentstyle=\tt
}

\newcommand{\programm}[1]{
	\setstretch{1.0}
	\lstinputlisting[captionpos=t,frame=single]{#1}
	\onehalfspacing
}

% Reanumbri väljastamine 'listing' keskkonnas

\renewcommand*\thelstnumber{\oldstylenums{\the\value{lstnumber}}:}


% Vormistuslikud seaded
% Pealkirjad nummerdamata

%\setcounter{secnumdepth}{-2}

% Lehekülgede ääred

\geometry{verbose,a4paper,tmargin=3.5cm,bmargin=2.5cm,lmargin=3cm,rmargin=2cm}

% Mitu taset näidata sisukorras

\setcounter{tocdepth}{2}

% Jäta lõikude vahele rohkem ruumi

\setlength{\parskip}{\medskipamount}

% Ära taanda lõike

\setlength{\parindent}{0pt}

% Määra reavahe

\onehalfspacing

\begin{document}
\thispagestyle{empty}

\begin{center}
{\large\sffamily{
T A R T U\ \ \ Ü L I K O O L\\
MATEMAATIKA-INFORMAATIKATEADUSKOND\\
Arvutiteaduse instituut\\
Informaatika eriala
}}

\vspace*{\stretch{5}}

{\textbf{\Large Raivo Laanemets}}

\vspace{4mm}

{\textbf{\LARGE DPLL protseduur lahendite loendamiseks}}

\vspace{5mm}

{\textbf{\Large Bakalaureusetöö (4 AP)}}

\end{center}

\vspace*{\stretch{4}}

{\hfill{\Large Juhendaja: Tõnu Tamme, MSc}}

\vspace*{\stretch{3}}

\begin{flushright}

\noindent Autor: ........................................... ``.....''
jaanuar\hskip16pt 2008

\vspace{2mm}

\noindent Juhendaja: .................................... ``.....''
jaanuar\hskip16pt 2008

\end{flushright}

\vspace*{\stretch{2}}

\begin{center}
{\large{TARTU 2008}}
\end{center}

% Siin lõpeb tiitelleht

\newpage

\tableofcontents{}
\chapter*{Sissejuhatus}
\addcontentsline{toc}{chapter}{Sissejuhatus}

Lausearvutusvalemi lahendite loendamise probleem on tihedalt seotud
kehtestatavuse probleemiga ja omab seetõttu suurt teoreetilist ning
praktilist tähtsust. Mitmed olulised probleemid on teisendatavad
nendele probleemidele. Seetõttu on tähtis nii loendamise kui kehtestatavuse
algoritmide efektiivsus, st. töökiirus. Selles vallas on viimasel ajal
toimunud palju uuringuid ning erinevate lahendusalgoritmide kohta on kirjutatud
kümneid artikleid.

Valemi kehtestatavus kui otsustusprobleem on \textit{NP-täielik} \cite{cook71},
mis paigutab ta samasse keerukuslassi kümnete teiste oluliste probleemidega.

Lahendite loendamise probleem on kehtestatavusele vastav loendamisprobleem.
Tema kuulub keerukusklassi \textit{\#P-täielik} \cite{roth96}. Samuti on
teada, et lahendite arvu isegi ligikaudne leidmine väikese garanteeritud veaga
on arvutuslikult raske ülesanne. Klassis \textit{\#P-täielik} asuvad ka mitmed
tehisintellekti valdkonnast tuntud ülesanded, näiteks tuletus Bayes'i võrgus ja
teised tuletusprobleemid mittetäieliku teadmusega (tõenäosuslikes)
süsteemides.

Kehtestatavuse jaoks on teada palju algoritme. Üheks neist on \textit{DPLL}
\emph{(Davis-Putnam-Logemann-Loveland)} protseduur \cite{davis60,davis62},
mille edasiarendatud variandid on kõrge efektiivsusega lahendusprogrammides
kasutusel ka tänapäeval, rohkem kui 40 aastat pärast originaalalgoritmi
publitseerimist.

Hiljuti selgus, et \textit{DPLL} algoritmi saab küllaltki edukalt täiendada ka
lahendite loendamise jaoks \cite{birnbaum99}.

\section*{Töö ülesehitus ja eesmärgid}

Käesolevas töös vaadeldakse originaalsest \textit{DPLL} algoritmist
saadud lausearvutusvalemite lahendite loendamisalgoritmi.

Töö alguses esitatakse probleemi formaalne kirjeldus ja töös
läbivalt kasutatavate mõistete definitsioonid.

Esimeses osas esitatakse algoritmide kirjeldus ning pseudokood,
selgitatakse lahti kasutatud ideed ja antakse
põhjendus, miks ühe või teise põhimõtte tarvitamine on lubatav ja mõistlik.

Teises osas antakse lihtne realisatsioon
programmerimiskeeles Prolog. Põhiliselt sisaldab see peatükk kasutatud
andmestruktuuride ja protseduuride kirjeldusi ning näiteid nende kasutamisest.

Tööd jääb lõpetama lisaosana kaasa pandud programmi dokumentatsioon, s.o.
koostatud protseduuride sisend/väljundargumentide kirjeldused ning lühike
ülevaade sellest, mida konktreetne protseduur teeb. Programmi masinloetava
lähetkoodi võib leida kaasapandud optiliselt andmekandjalt.

\section*{Definitsioonid}

\emph{Muutujaks} nimetame suvalist sümbolit $x$
fikseeritud sümbolite hulgast. \emph{Literaaliks} $l$ loeme muutujat $x$ ($l\equiv x$) või tema
eitust $-x$ ($l\equiv-x$). \emph{Klausliks} $C$ loeme literaalide disjunktsiooni
$C=l_{1}\vee\dots\vee l_{n}$. \emph{Konjuktiivsel normaalkujul valemiks} (KNK)
$F$ loeme klauslite konjuktsiooni $F=C_{1}\wedge\dots\wedge C_{k}$. Muutuja
$x$ \emph{väärtustuseks} nimetame funktsiooni $v$, $v(x)\in\{0,1\}$. Literaal
$l$ on tõene, kui $l\equiv x$ ja $v(x)=1$ või $l\equiv\neg x$
ja $v(x)=0$. Literaali $l$ on väär, kui $l\equiv x$ ja $v(x)=0$
või $l\equiv-x$ ja $v(x)=1$. Literaali $l\equiv x$ \emph{komplementaariks}
on literaal $\bar{l}\equiv\neg x$ ja literaali $l'\equiv\neg x$ komplementaariks
on $\bar{l'}\equiv x$. Klauslit $C=l_{1}\vee\dots\vee l_{n}$
nimetame väärtustusel $v$ tõeseks ($v(C)=1$), kui $C$ sisaldab
vähemalt ühte tõest literaali, vastasel korral vääraks ($v(C)=0$).
Valemit $F=C_{1}\wedge\dots\wedge C_{k}$ loeme väärtustusel $v$ tõeseks
($v(F)=1$), kui ükski klauslitest $C_{1},\dots,C_{k}$ pole väär, vastasel
korral vääraks ($v(F)=0$).

\emph{n-muutuja Boole'i funktsiooniks} $f$ nimetame
funktsiooni signatuuriga \mbox{$f:\{0,1\}^n\to \{0,1\}$}. On selge, et suvalist
Boole'i funktsiooni saab esitada mingi talle vastava lausearvutuse valemiga.

\textbf{Kehtestatavusprobleem}

Olgu antud $n$-muutuja Boole'i funktsioon $f(x_1,\dots,x_n)$.
Kehtestatavusprobleem küsib, kas leidub\footnotemark[1] selline muutujate
$x_1,\dots,x_n$ väärtustus, kus $f(x_1,\dots,x_n)=1$.

\textbf{Loendamisprobleem}

Olgu antud $n$-muutuja Boole'i funktsioon $f(x_1,\dots,x_n)$. Lahendite
loendamise probleem on loendamisprobleem, mitu erinevat $x_1,\dots,x_n$
väärtustust leidub, kus $f(x_1,\dots,x_n)=1$.

\footnotetext[1]{Praktilistes rakendustes omavad sageli tähtsust ka
tegelikud muutujate väärtused.}

\chapter{DPLL algoritm}

\textit{DP (Davis-Putnam)} algoritm \cite{davis60} oli üks esimesi
süstemaatilisi algoritme esimest järku predikaatarvutuse valemitega kirja pandud teoreemide
tõestamiseks arvuti abil. Olgugi, et algoritm on küllaltki ebaefektiivne võrreldes
paar aastat hiljem (1965) avaldatud Robinsoni algoritmiga, on tarvitatud sellest
algoritmist pärit meetodeid hiljem kehtestatavuse algoritmide arendamiseks.
\textit{DP} kasutas loogilise tõesuse kontrollimise asemel valemi eituse
mittekehtestatavuse väljaselgitamist, mis on esimesega samaväärne probleem.

Töö piiratud maht võimaldab anda ainult väga põgusa ülevaate
originaalalgoritmist. Algoritm koosnes neljast osast:

\begin{enumerate}
  \item Esialgse valemi $F$ eituse $\neg F$ leidmine.
  \item Valemi $\neg F$ kirjutamine prenex-kujule\footnotemark[2] $G_p$
  (prefikskuju leidmine).
  \item $G_p$ eksistentsikvantorite eemaldamine
  vastavate muutujate funktsionaalsümbolitega asendamise teel
  (skolemiseerimine\footnotemark[2]). Saadakse valem $G_s$. Tähtis on teada,
  et $G_p$ ja $G_s$ ei ole loogiliselt ekvivalentsed, kuid on mittekehtestatavuse suhtes ekvivalentsed, st. $G_p$ on mittekehtestatav parajasti siis, kui $G_s$ on mittekehtestatav.
  \item Valemist $G_s$ muutuja- ja kvantorivabade lausearvutusvalemite
  $G_{s,1},G_{s,2},\dots$ genereerimine kuni nendest $n$ esimese valemi
  konjuktsioon $G_{s,1}\wedge\dots\wedge G_{s,n}$ on mittekehtestatav. 
\end{enumerate}

\footnotetext[2]{Prenex-kuju/skolemiseerimine jt.
valdkonnaga seotud mõisted seletatakse lahti allikas
\cite{schoning04}.}

Algoritm töötas nii kaua, kuni sammus (4) oli genereeritud piisaval hulgal
kvantori- ja muutujavabu valemeid. On ära näidatud, et kui esialgne valem $F$
on loogiliselt tõene, siis leidub lõplik kogus valemeid $G_{s,1},G_{s,2},\dots$
ja programm lõpetab oma töö, vastasel korral jääbki ta valemeid genereerima.
Kuna predikaatloogika on poollahenduv, siis see ongi parim tulemus, mille me
võime saada.

Reeglid sammus (4) valemite konjuktsiooni $W=G_{s,1}\wedge\dots\wedge G_{s,n}$
kehtestatavuse kindlakstegemiseks olid järgmised:

\begin{enumerate}
  \item \emph{Ühikklauslite elimineerimine}
  	\begin{enumerate}
        \item Kui valem $W$ sisaldab ühikklauslit $C_i$ literaaliga $l$ ja
        samuti ühikklauslit literaaliga $\bar{l}$, siis sisaldab $W$ vastuolu
        ning on mittekehtestatav.
        \item Kui (a) ei ole rakendatav ja valem $W$ sisaldab ühikklauslit $C_i$
        literaaliga $l$, siis kustutada valemist $W$ kõik klauslid, mis sisaldavad literaali $l$ ja
        igast klauslist, mis sisaldab $l$ komplementaari $\bar{l}$,
        kustutada $\bar{l}$. Saadav valem $W'$ on mittekehtestatav parajasti
        siis, kui seda on $W$.
   \end{enumerate}
   \item \emph{Puhta literaali elimineerimine}\\
        Kui muutuja $x$ esineb valemi $W$ klauslites ainult
        positiivse või negatiivse literaalina, siis võib kõik muutujat $x$
        sisaldavad klauslid eemaldada, saades valemi $W'$, mis on mittekehtestatav parajasti siis,
        kui seda on $W$.
   \item \emph{Muutujate elimineerimine}\\
   		Kui valem $W$ on kujul $W=(A\vee l)\wedge(B\vee \bar{l})\wedge R$, kus $A$,
   		$B$ ja $R$ literaali $l$ ega tema komplementaari $\bar{l}$ ei sisalda,
   		siis võib valemi $W$ ümber kirjutada kujule $W'=(A\vee B\vee)\wedge R$.
   		Valem $W$ on mittekehtestatav parajasti siis, kui $W'$ on mittekehtestatav.
\end{enumerate}

Hiljem publitseeritud artiklis \cite{davis62} asendatakse
muutujate elimineerimisreegel teistsugusel kujul oleva nn.
\emph{jaotamisreegliga}:

Kui valem $W$ on kujul $W=(A\vee l)\wedge(B\vee \bar{l})\wedge R$, kus $A$,
$B$ ja $R$ literaali $l$ ega tema komplementaari $\bar{l}$ ei sisalda,
siis võib valemi $W$ jaotada valemiteks $W_1=A\wedge R$ ja $W_2=B\wedge R$ ning
esialgne valem on mittekehtestatav parajasti siis kui mõlemad $W_1$ ja $W_2$ on
mittekehtestatavad.

Koos selle reegliga moodustab \textit{DP} algoritm algoritmi \textit{DPLL}.

Hetkel levinud \textit{DPLL}-tüüpi kehtestatavuse algoritmid kasutavad
originaalsest algoritmist erinevat lähenemist. Eelkõige on algoritmid
realiseeritud tagurdusalgoritmidena, mis töötavad muutujate osalise
väärtustusega, samm-sammult valides muutujatele väärtused, lihtsustades
valemit pärast iga väärtustamist, kuni saadakse selline lõpptulemus, mille 
korral valemit võib lugeda seni leitud väärtustuse piires tõeseks.


\section{Boole-Shannoni dekompositsioon}

Boole-Shannoni dekompositsioon (lahutus) vastab \textit{DPLL} algoritmis
jaotamisreeglile. Sisuliselt on siin tegemist universaalse vahendiga
$n$-muutuja ($n>0$) Boole'i funktsiooni avaldamiseks kahe, või erijuhul ühe,
$n-1$ muutuja Boole'i funktsiooni kaudu.

Olgu antud antud $n$-muutuja Boole'i funktsioon $f(x_1,\dots,x_n)$. Sellest
funktsioonist võime saada kaks $n-1$-muutja funktsiooni, asendades muutuja
$x_i$ väärtusega 1 esimeses ja 0 teises. Nendeks funktsioonideks on:

$$f_{\bar{x_1}}(x_1,\dots,x_{i-1},x_{i+1}\dots,x_n)=f(x_1,\dots,x_{i-1},0,x_{i+1}\dots,x_n)$$

ja

$$f_{x_1}(x_1,\dots,x_{i-1},x_{i+1}\dots,x_n)=f(x_1,\dots,x_{i-1},1,x_{i+1}\dots,x_n).$$

Sealjuures kehtib ka samaväärsus

$$f=f_{\bar{x_1}}\vee f_{x_1},$$

mille korrektsuse tõestuse võib leida allikast \cite{tombak07}. Põhimõtteliselt
on võimalik kehtestatavuse algoritm koostada ainult dekompositsiooni kasutades,
sest rakendades seda rekursiivselt järjest igale alamvalemile, jõuame ükskord
olukorrani, kus kõik muutujad on asendatud väärtustega 1 või 0, ning saame
valemi väärtuse välja arvutada. Paraku oleks see väga ebaefektiivne algoritm,
võrreldav tõeväärtustabeli koostamisega.

Vaatleme dekompositsiooni rakendust KNK valemile $F=(x\vee\neg y\vee z) \wedge
(\neg x\vee y\vee\neg z) \wedge (x\vee\neg y\vee\neg z) \wedge (\neg x\vee\neg
y\vee\neg z)$ muutuja $y$ järgi. Saame valemid:

$$F_y=(x\vee 0\vee z) \wedge
(\neg x\vee 1\vee\neg z) \wedge (x\vee 0\vee\neg z) \wedge (\neg x\vee 0\vee\neg
z)$$
ja
$$F_{\bar{y}}=(x\vee 1\vee z) \wedge
(\neg x\vee 0\vee\neg z) \wedge (x\vee 1\vee\neg z) \wedge (\neg x\vee 1\vee\neg
z).$$

Oh lihtne märgata, et saadud valemeid oleks võimalik oluliselt lihtsustada.
Üldiselt saame lihtsustamise põhimõtted kirja panna lihtsa eeskirjana, mis
lihtsustab dekompositsiooni $F \Leftrightarrow F_x \vee F_{\bar{x}}$ valemid
$F_x$ ja $F_{\bar{x}}$:

\begin{enumerate}
  \item Valemist $F_x$ eemalda kõik klauslid, kus muutuja $x$ esineb
  positiivselt ja kõikidest $F_x$ klauslitest, kus $x$ esineb negatiivselt,
  eemalda $x$.
  \item Valemist $F_{\bar{x}}$ eemalda kõik klauslid, kus muutuja $x$ esineb
  negatiivselt ja kõikidest $F_{\bar{x}}$ klauslitest, kus $x$ esineb positiivselt,
  eemalda $x$.
\end{enumerate}

See algoritm võimaldab lihtsustada $F_x$ ja $F_{\bar{x}}$ n.ö. vahepealse
väljarvutamiseta, kuigi asendades $x$ väärtuse valemisse sisse, saame valemi
lihtsustada ka tuntud lausearvutuse samaväärsusi $x\vee 0 \equiv x$ ja $x\vee 1
\equiv 1$ kasutades.

Võttes eespool toodud näite uuesti ette, rakendame lihtsustusi leitud valemitele
$F_y$ ja $F_{\bar{y}}$, saades semantiliselt samaväärsed, kuigi süntaktiliselt
märksa lihtsamad KNK valemid:

$$F_y=(x\vee z)\wedge (x\vee\neg z)\wedge(\neg x\vee\neg z)$$
ja
$$F_{\bar{y}}=(\neg x\vee\neg z).$$

Dekompositsiooni ei saa rakendada kuitahes palju arv kordi, sest
valemi muutujate arv väheneb iga korraga ühe võrra (kuigi lihtsustuste
tõttu võivad kaduda valemist ka need muutujad, mille järgi ei ole lahutust
rakendatud). On võimalik kahe erineva olukorra tekkimine:

\begin{enumerate}
  \item[(*a)] Valemist eemaldatakse kõik klauslid - sellisel juhul ei saa valem
  enam vääraks muutuda, sest ei ole võimalik, et mõni valemi klausel oleks
  väär. Sellisel juhul on valem tõene.
  \item[(*b)] Valemisse juhtub sisse jääma klausel, millest on eemaldatud kõik
  muutujad - selline valem on väär, kuivõrd temas leidub väär klausel.
\end{enumerate}

Rakendades dekompositsiooni näites saadud valemile $F'_{\bar{y}}$ edasi muutuja
$x$ järgi, saame omakorda valemid $F_{\bar{y},x}=\neg z$ ja
$F_{\bar{y},\bar{x}}$, millest viimane ei sisalda ühtegi klauslit ja on
seetõttu tõene ning seega oleme leidnud kehtestava väärtustuse
$\{y=0,x=0\}$.

Pannes kokku nii väärtustamise kui valemite lihtsustamise, saab koostada
järgmise algoritmi (1):

\newpage

\begin{algorithm}{SAT($F$)}
\ainputoutput{konjuktiivsel normaalkujul valem $F$.}{tulemus, kas valem $F$ on
kehtestatav.}
\abody

\begin{algorithmic}[1]
\IF {$F$ ei sisalda ühtegi klauslit}
	\RETURN \TRUE
\ELSIF {$F$ sisaldab tühja klauslit}
	\RETURN \FALSE
\ELSE
	\STATE Vali valemist $F$ muutuja $x$.
	\STATE Rakenda valemile $F$ dekompositsiooni $x$ järgi, saades valemid $F_x$
	ja $F_{\bar{x}}$.
	\STATE Lihtsusta valemeid $F_x$ ja $F_{\bar{x}}$.
	\RETURN \proc{SAT($F_x$)}$\vee$\proc{SAT($F_{\bar{x}}$)}.
\ENDIF
\end{algorithmic}
\end{algorithm}

Esitatud rekursiivsest pseudokoodist on näha, et ridadel (1-2) kontrollitakse
eespool antud lõpptingimuse (*a) täidetust, kus valemi $F$ kõik klauslid on
tõeseks muutunud ja lihtsustamiste tulemusena ei ole valemisse $F$ enam
ühtegi klauslit jäänud. Sellisel juhul tagastab protseduur \texttt{true},
st. etteantud valem on kehtestatav.

Ridadel (3-4) kontrollitakse seevastu lõpptingimuse (*b) täidetust, st.
inspekteeritavasse valemisse on sattunud väär klausel. Sellisel juhul ei ole
enam mõtet midagi edasi arvutada, sest valemit ei õnnestu ühelgi viisil
tõeseks muuta. Protseduur tagastab \texttt{false}, st. etteantud valem $F$ on
mittekehtestatav.

Kui protseduuri kutsungist ei väljutud ridade (1-4) kaudu, siis rakendadakse
valemile Boole-Snannoni dekompositsiooni (6-7), vahetult mille järel toimub
lihtsustamine (8). See annab kaks valemit $F_x$ ja $F_{\bar{x}}$, milles on
vähem muutujaid, kui esialgses valemis $F$. Real (9) rakendatakse mõlemale
alamvalemile algoritmi \proc{SAT} rekursiivselt.

Nagu eespool mitu korda mainitud, realiseeritakse sellised algoritmid
tagurdusalgoritmidega. Tagurduse valikupunktiks on siin rida (9). Siin on näha,
et valik tuleb teha kahe valemi $F_x$ ja $F_{\bar{x}}$ vahel, millele algoritmi
rekursiivselt edasi rakendada. Kui me oskaks kohe valida õige valemi, oleks
sellise algoritmi keskmine keerukus lineaarne muutujate arvu suhtes.
Paraku me ei oska õiget valemit valida ja selletõttu tuleb halvimal juhul mõlemad valemid läbi
proovida, aga see muudab algoritmi tööaja juba eksponentsiaalseks. On näidatud,
et leidub isegi terve valemite klass, mille jaoks see algoritm töötab alati
eksponentsiaalses ajas \cite{tombak07}.

\subsection{Ühikklauslite elimineerimine}

Boole-Shannoni dekompositsioonil on erikuju. Olgu $F$ konjuktiivsel
normaalkujul olev valem. Kui $F$ sisaldab sellist klauslit $C$, mis koosneb
ainult ühest literaalist, st. $C=l$ ja $l\equiv x$, siis võib sooritada $F$
lahutuse ($F_x$, $F_{\bar{x}}$) muutuja $x$ järgi ja ignoreerida valemit
$F_{\bar{x}}$, sest see on samaselt väär. Seda põhjendab asjaolu, et
väärtustusel $v(x)=0$, $C=l=0$. See aga tähendab omakorda, et valem
$F_{\bar{x}}$ sisaldab väära klauslit. Täpselt sarnane olukord kehtib ka juhul
kui $l\equiv\neg x$, aga siis võime vaatluse alt välja jätta valemi $F_x$.

Ühikklauslite elimineerimine kui Boole-Shannoni lahutuse erikuju vastab
originaalses \textit{DPLL} algoritmis ühikklauslite elimineerimisreegli juhule
(b). Selle reegli saame lisada otse algoritmi (1), saades juba märksa
efektiivsema, kuigi siiski veel eksponentsiaalses ajas töötava, algoritmi:

\begin{algorithm}{DPLLSAT($F$)}
\ainputoutput{konjuktiivsel normaalkujul valem $F$.}{tulemus, kas valem $F$ on
kehtestatav.}
\abody

\begin{algorithmic}[1]
\IF {$F$ ei sisalda ühtegi klauslit}
	\RETURN \TRUE
\ELSIF {$F$ sisaldab tühja klauslit}
	\RETURN \FALSE
\ELSIF {$F$ sisaldab ühikklauslit $C_i=l$}
	\STATE Rakenda valemile $F$ dekompositsiooni $x$ järgi, saades valemid
	$F_x$ ja $F_{\bar{x}}$.
	\IF {$l\equiv x$}
		\STATE Lihtsusta valemit $F_x$.
		\RETURN \proc{DPLLSAT($F_x$)}.
	\ELSE
		\STATE Lihtsusta valemit $F_{\bar{x}}$.
		\RETURN \proc{DPLLSAT($F_{\bar{x}}$)}
	\ENDIF
\ELSE
	\STATE Vali valemist $F$ muutuja $x$.
	\STATE Rakenda valemile $F$ dekompositsiooni $x$ järgi, saades valemid $F_x$
	ja $F_{\bar{x}}$.
	\STATE Lihtsusta valemeid $F_x$ ja $F_{\bar{x}}$.
	\RETURN \proc{DPLLSAT($F_x$)}$\vee$\proc{DPLLSAT($F_{\bar{x}}$)}.
\ENDIF
\end{algorithmic}
\end{algorithm}

\subsection{Puhta literaali elimineerimine}

Puhta literaali elimineerimine on lisaks ühikklausli elimineerimisele veel üks
täiendav heuristiline reegel kehtestatavuse leidmiseks. Olgugi et see esines
originaalses \proc{DPLL} algoritmis, on temast hilisemates variantides
loobutud eelkõige selle tõttu, et praktikas on osutunud tema kasutamine
ebaefektiivse realisatsiooni tõttu vähetõhusaks.

Puhta literaali $l$ elimineerimine kujub endast valemi $F=C_1\wedge\dots\wedge
C_k$ klauslitest nende eemaldamist, kus $l$ esineb. Sealjuures ei tohi
üheski klauslis $C_1\wedge\dots\wedge C_k$ esineda $l$ komplementaari $\bar{l}$.
Reegli õigustuse võib leida allikast \cite{tombak07}. Puhta literaali reeglile
vastab Boole-Shannoni dekompositsiooni rakendamine valemile $F$ literaalile $l$
vastava muutuja $x$ järgi. Siis ütleb reegel, et me võime ignoreerida valemit
$F_{\bar{x}}$, kui $l\equiv x$ ja valemit $F_x$ juhul $l\equiv -x$.

Paraku on puhta literaali eemaldamisel mittekasulik omadus
valemi lahendite arvu suhtes. Nimelt kaotame selle kasutamisel osa lahenditest,
sest võib juhtuda, et valem võib olla tõene ka juhul, kui puhas literaal $l$ on
väär, st. esialgne valem ja pärast teisendust saadud valem ei ole loogiliselt
samaväärsed. Näitena sobib valem $F=x\vee y$, millest saaksime rangelt
puhta literaali elimineerimist tarvitades lahendi $x=1, y=1\}$, aga sealjuures
ignoreeriksime täielikult lahendeid $\{x=0, y=1\}$ ja $\{x=1, y=0\}$.

See halb omadus on põhjuseks, miks puhta literaali elimineerimise reegel on
käesolevas töös vaadeldud \textit{DPLL} algoritmi variandist välja jäetud.
Teiseks põhjuseks on valemist puhta literaali otsimise arvutuslik keerukus
valemi andmestruktuurist, mis on esitatud töö teises pooles, sest andmestruktuur on
optimeeritud ühikliteraali leidmiseks ning valemi kiireks lihtsustamiseks
vahetult pärast muutuja valikut.


\section{Lahendite loendamine}

Seni oleme töös vaadelnud ainult kehtestatavuse algoritmi. Selgub, et
lahendite loendamiseks tuleb seda (2) ainult vähe muuta.
Loendamine kasutab ära asjaolu, et ülalkirjeldatud algoritm üritab leida
kehtestatavat väärtustust üksikute muutujate väärtustamise kaupa, kusjuures
alati lõpetab protseduur ühes kahest olukorrast (*a, *b), mida on töös eespool
juba kirjeldatud.

On ilmselt selge, et valemi lahendite arv (kujutades ette protseduuri
\proc{SAT} või \proc{DPLLSAT} rakendamist valemile $F$ ja vahetult olukorra (*a)
või (*b) esinemist) on neil juhtudel järgmine:

\begin{itemize}
  \item[(*a):] Valemis $F$ oli esialgu $n$ muutujat, nendest
  on ära väärtustatud $k$ tükki. See tähendab, et ülejäänud $t=n-k$ muutujale võib anda
  suvalise väärtustuse, ilma et peaks kartma, et valem $F$ vääraks muutuks. $t$
  muutujale saab kokku anda $2^t$ erinevat väärtustust, seega on $F$
  lahendite arv $2^{n-k}$.
  \item[(*b):] Kuna valem $F$ on väär, siis on tema lahendite arv 0.
\end{itemize}

Vastavalt nendele juhtudele on vaja modifitseerida algoritmi \proc{SAT} või
\proc{DPLLSAT} ainult niipalju, et anda sisendparameetrina kaasa valemi $F$
esialgne lahendite arv $n$ ja seni väärtustatud muutjate arv $k$. Protseduuri
rekursiivses sammus (\proc{SAT}, rida 9; \proc{DPLLSAT}, rida 18) tuleb
loomulikult igal sammul väärtustatud muutujate arvule juurde liita 1 . Saadav
algoritm on esitatud järgnevalt:

\begin{algorithm}{DPLL\_LOENDA($F,N,K$)}
\ainputoutput{knk valem $F$, $F$ muutjate arv $N$, väärtustatud muutujate arv
$K$.}{valemi $F$ lahendite arv $S$.}
\abody

\begin{algorithmic}[1]
\IF {$F$ ei sisalda ühtegi klauslit}
	\RETURN $2^{N-K}$
\ELSIF {$F$ sisaldab tühja klauslit}
	\RETURN 0
\ELSIF {$F$ sisaldab ühikklauslit $C_i=l$}
	\STATE Rakenda valemile $F$ dekompositsiooni $x$ järgi, saades valemid
	$F_x$ ja $F_{\bar{x}}$.
	\IF {$l\equiv x$}
		\STATE Lihtsusta valemit $F_x$.
		\RETURN \proc{DPLL\_LOENDA($F_x$)}.
	\ELSE
		\STATE Lihtsusta valemit $F_{\bar{x}}$.
		\RETURN \proc{DPLL\_LOENDA($F_{\bar{x}}$)}
	\ENDIF
\ELSE
	\STATE Vali valemist $F$ muutuja $x$.
	\STATE Rakenda valemile $F$ dekompositsiooni $x$ järgi, saades valemid $F_x$
	ja $F_{\bar{x}}$.
	\STATE Lihtsusta valemeid $F_x$ ja $F_{\bar{x}}$.
	\RETURN \proc{DPLL\_LOENDA($F_x$)}$ + $\proc{DPLL\_LOENDA($F_{\bar{x}}$)}.
\ENDIF
\end{algorithmic}
\end{algorithm}

Sarnaselt algoritmile \proc{DPLLSAT}, saab antud protseduuri kirja panna
tagurdusalgoritmina, mida on tehtud ka töö teises pooles realiseeritud
programmis. Selleks piisab võtta kasutusele globaalne muutuja, mille
väärtust suurendada iga kord $2^{n-k}$ võrra ($n$ - muutujate koguarv, $k$ -
väärtustatud muutujate arv), kui algoritm leiab kehtestava väärtustuse.
Seejärel tuleb lasta algoritmil tagurdada kuni viimase rekursioonitasemeni,
milles pole läbi vaadatud mõlemat valemit. (rida 18 algoritmis
\proc{DPLL\_LOENDA}) ning pärast protseduuri rakendamise lõppu lugeda selle
muutuja väärtus, mis annabki lahendite arvu.


\chapter{Loendamisalgoritmi realisatsioon}

Selles töö osas anname eespool kirjeldatud algoritmi realisatsiooni
keeles Prolog. Prolog sai valitud tema mitmete heade omaduste
tõttu, mida saab edukalt ära kasutada sellist tüüpi ülesannete lahendamisel.
Prolog baseerub ise tagurdusalgoritmil ja see vähendab algoritmi
kirjapanekul tööd oluliselt, sest programmeerija ise ei pea
tegelema selle keerulise osaga, mida on tagurdamisel muutujate
väärtuste efektiivne taastamine. Lisaks on Prolog tuntud kui
sümbolmanipuleerimise keel, peale selle ei ole vaja Prologis
programmeerijal muretseda korrektse mäluhalduse pärast, sest Prolog
sisaldab automaatset mälukoristust. Need omadused teevad Prologist keele,
mis on vägagi sobiv keeruliste algoritmide
realiseerimiseks või prototüüpimiseks. Samuti on oluline mainida, et
Prolog on populaarne keel tehisintellekti valdkonda kuuluvate ülesannete
lahendamiseks. See muudab parajasti lihtsaks antud algoritmi integreerimise ja
laiendamise teiste tehisintellekti probleemide jaoks.

Käesoleva töö praktilise osa täielikuks mõistmiseks on tarvis osata lugeda
Prologi programme. Väga heaks sissejuhatavaks materjaliks (millest ka täiesti
piisab töö mõistmiseks), on allikas \cite{tamme03}.

Realisatsiooni kirjeldust alustame programmi üldise struktuuri esitamisega,
kusjuures ei ole isegi olulised kasutatavad andmestruktuurid, mille
kirjelduse anname alles mõnevõrra hiljem.

Selles töös esitatud viis \textit{DPLL} kehtestatavus- ja loendamisalgoritmi
jaoks ei ole loomulikult ainuõige võimalus programmi realiseerimiseks. Samas on
see lähenemine töö autori arvates sobiv Prolog-tüüpi keeles algoritmi
implementeerimiseks. Imperatiivses programmeerimiskeeles, näiteks C
keeles, on märksa levinum kasutada lähenemist, mille pseudokood on toodud
artiklis \cite{silva96}.

\newpage

\section{Realisatsiooni ülesehitus}

Algoritmi realiseerimisel sai koostatud lihtne tagurdusel põhinev
programm. Kasutatud andmestruktuurid seletatakse põhjalikumalt lahti selleks
pühendatud töö sektsioonis pärast algoritmi kirjeldust.

\subsection{Muutujate kodeerimine}

Programmis esitame lahendatava valemi muutujad täisarvudena 1,2,\ldots. See on
levinud viis, kuna võimaldab mugavat indekseerimist lineaarsetes andmestruktuurides,
näiteks massiivides. Samuti saame kompaktselt ja elegantselt esitada literaale.
Muutuja positiivse esinemise tähistamiseks tuleb kasutada lihtsalt
muutujale vastavat täisarvu, negatiivse esinemise tarvis aga vastandarvu.

See kodeerimismeetod pakub triviaalsed lahendused operatsioonidele nagu
literaali komplementaari leidmine jt. Olgugi et Prolog kui predikaatarvutust
põhjana kasutav keel otseselt massiive andmestruktuuridena ei paku, leidub
alternatiivne lähenemine termide inspekteerimise vahendite näol. 

\subsection{Põhiprotseduur}

Eespool esitatud algoritm \proc{DPLLSAT} on kirja pandud
kahe protseduuri järjest rakendamisel saadava rekursiivse programmi abil:

\begin{lstlisting}[numbers=left,xleftmargin=1cm,basicstyle=\tt]
dpll(L, F, S, Ct, D, Ys):-
	kontrolli(O, L, F, S, Ct, Ys),
	samm(O, F, S, Ct, D).
\end{lstlisting}

Selles protseduuris on muutujate kirjeldused järgmised (kõik muutujad on
sisendparameetrid):

\begin{itemize}
  \setlength{\itemsep}{1mm}
  \item [\texttt{L} -] viimati tõeseks valitud literaal;
  \item [\texttt{F} -] lahendatav valem;
  \item [\texttt{S} -] hetkel kehtiv muutujate ja klauslite väärtustus;
  \item [\texttt{Ct} -] hetkel tõeste klauslite arv;
  \item [\texttt{D} -] seni sooritatud lahendussammude arv, võrdne seni
  väärtustatud muutujate arvuga;
  \item [\texttt{Ys} -] ühikliteraalide vahepinu (sisaldab literaale, mis on
  programmi töö käigus ühikliteraalideks muutunud ja tuleks deskompositsiooni
  rakendades esmajärjekorras ära kasutada);
  \item [\texttt{O} -] protseduuri \texttt{kontrolli/6} rakendamisel saadav
  väärtus, mille põhjal tehakse järgmine lahendussamm.
\end{itemize}

Protseduur \texttt{kontrolli/6} sisuliselt vaatab järgi, kas hetkel kehtiva
väärtustuse piires on valem \texttt{F} juba tõeseks muutunud või mitte.
Põhimõtteliselt on võimalikud kolm olukorda:

\begin{enumerate}
  \item Valem \texttt{F} on tõeseks muutunud, st. kõik tema klauslid sisaldavad
  vähemalt ühte tõest literaali. Sellisel juhul on leitud kehtestav väärtustus
  ja programm lõpetab oma töö ära. \texttt{kontrolli/6} väljund on siis
  \texttt{O = sat}.
  \item Valemis \texttt{F} leiduvad ühikklauslid ning sooritada tuleb
  dekompositsioon nende järgi. Sellele olukorrale vastab \texttt{kontrolli/6}
  väljund \texttt{O = yhik(Ls)}, kus \texttt{Ls} on list literaalidest, mis
  ühikklauslitena esinevad. \texttt{kontrolli/6} on koostatud niimoodi, et see
  list ei saa kunagi tühi olla. Kui protseduuri \texttt{dpll/6} kutsudes
  sisaldas loend \texttt{Ys} mõnd elementi, siis täitub alati
  \texttt{kontrolli/6} praegune tulemus ja \texttt{Ls} sisaldab vähemalt neid
  elemente, mida sisaldab \texttt{Ys}.
  \item Valem \texttt{F} ei ole veel tõene, aga temas puuduvad hetkel
  ühikklauslid, mistõttu tuleb valida suvaline muutuja dekompositsiooni
  rakendamiseks. Siin on erijuhuks olukord, kus kõik muutujad on juba varem ära
  valitud ja antud sammus ei ole see enam võimalik. Programm tagurdab
  automaatselt viimase valikupunktini, milleks on eelmine
  rekursioonisamm, kus valiti küll muutuja, aga pole veel läbi proovitud tema
  mõlemat väärtustust. Kui selliseid valikupunkte ei ole, siis lõpetab programm
  töö negatiivse tulemusega. Olukorrale, kus järgmises sammus tuleks teha
  muutuja valik, vastab \texttt{kontrolli/6} väljund \texttt{O = vali}.
\end{enumerate}

Järgnevalt esitame protseduuri \texttt{samm/5} ülalkirjeldatud juhtudele. On
selge, et ei ole tähtis definitsioonide järjekorral, sest nad on üksteist
välistavad \texttt{samm/5} esimese sisendargumendi järgi (mis samal ajal on ka
\texttt{kontrolli/6} väljundiks).

\textbf{Juht 1:}

Kuivõrd on leitud kehtestav väärtustus, siis ei pea enam midagi edasi arvutama
ja samal ajal ei ole tarvis hoolida muutujate \texttt{F, S, Ct, D} väärtustest,
mida võime ignoreerida. Seega sobib järgmine definitsioon:

\begin{lstlisting}[numbers=left,xleftmargin=1cm,basicstyle=\tt]
samm(sat, _, _, _, _).
\end{lstlisting}

\textbf{Juht 2:}

Valemis esinesid ühikklauslid, võtame esimese vastava literaali tõeseks ja
rakendame väärtustusprotseduuri \texttt{propageeri/5}, mille väljundina saame
tekkinud väärade klauslite arvu \texttt{Cf} ja tõeste klauslite arvu
\texttt{Ct1}. Loomulikult on tarvis kontrollida ega vääri klausleid
ei saadud, mida tehakse real (3). Kui saadi, lõpetatakse selles harus
arvutamine ära ja tagurdatakse, muidu arvutatakse tõeste klauslite koguarv
\texttt{Ct2} ja rekursioonisügavus \texttt{D1} ning sooritatakse
päring \texttt{dpll/6}.

\begin{lstlisting}[numbers=left,xleftmargin=1cm,basicstyle=\tt]
samm(yhik([Y|Ys]), F, S, Ct, D):-
	propageeri(Y, F, S, Cf, Ct1),
	(Cf > 0 ->
		fail
		;
		Ct2 is Ct + Ct1,
		D1 is D + 1,
		dpll(Y, F, S, Ct2, D1, Ys)
	).
\end{lstlisting}

\textbf{Juht 3:}

Viimases võimalikus olukorras toimetetakse sarnaselt eelmise juhuga. Erinev on
vaid see, et siin tuleb muutuja valida eraldi päringuga \texttt{vali\_muutuja(F,
S, L)}. See protseduur teostab valiku valemi \texttt{F} ja hetkel kehtiva
väärtustuse \texttt{S} põhjal. Tema väljundiks on literaal \texttt{L}, mis
tähistab, kas vastav muutuja tuleb valida tõeseks või vääraks. Ülejäänud osas
toimitakse samamoodi kui juhul (2), ainult et ühikliteraalide vahepinu võetakse
võrdseks tühja listiga. 

\begin{lstlisting}[numbers=left,xleftmargin=1cm,basicstyle=\tt]
samm(vali, F, S, Ct, D):-
	vali_muutuja(F, S, L),
	propageeri(L, F, S, Cf, Ct1),
	(Cf > 0 ->
		fail
		;
		Ct2 is Ct + Ct1,
		D1 is D + 1,
		dpll(L, F, S, Ct2, D1, [])
	).
\end{lstlisting}

Kehtestatavusprogramm koos silumismooduli kutsetega, mis selguse huvides jäeti
välja siin esitatud kirjeldusest, on realiseeritud programmimoodulis
\texttt{dpll} ja asub failis \textit{dpll.pl}.

\section{Lahendite loendamine}

Lahendite loendamise jaoks tuleb programmi täiustada
tõeste väärtustuste loenduriga (lisa-sisendparameeter predikaatidele
\texttt{dpll} ja \texttt{samm}) ja asendada sammu sooritamise protseduuri
\texttt{samm} definitsioon ülalkirjeldatud juhule (1) järgmisega:

\begin{lstlisting}[numbers=left,xleftmargin=1cm,basicstyle=\tt]
samm(sat, F, _, _, D, Lo):-
	muutujate_arv(F, N),
	nb_getval(Lo, C),
	C1 is C + integer(2^(N-D)),
	nb_setval(Lo, C1),
	fail.
\end{lstlisting}

Protseduuris tähistab \texttt{Lo} loendurit, millest küsitakse kõigepealt seni
saadud lahendite arv \texttt{C} (rida 3), millele liidetakse juurde hetkel
leitud lahendite arv (rida 4). Saadud tulemus \texttt{C1} salvestatakse tagasi
loendurisse \texttt{Lo} (rida 5). Protseduuri lõpus teostatakse
pseudopäring \texttt{fail} (rida 6), mis programselt kutsub esile tagurdamise.
Loenduri kui globaalse muutuja kasutamiseks tarvitatakse ekstraloogilisi predikaate
\texttt{nb\_getval/2} ning \texttt{nb\_setval/2}, mille tulemus ei sõltu
üldisest Prologi virtuaalmasina käitumisest tagurdusoperatsiooni käigus.

\section{Põhiandmestruktuurid}

Eespool esitatud programmi selgitusest on välja jäetud valemi andmestruktuuri
kirjeldus (\texttt{F} struktuur protseduurides \texttt{dpll} ja \texttt{samm}),
samuti hetkel lahendatava valemi lausearvutusmuutujate väärtuse talletamine
(\texttt{S} struktuur). Programmi üldisel kirjeldamisel on nad ebaolulised,
kuid siiski vajalikud kasutatud abipredikaatide \texttt{kontrolli/6},
\texttt{propageeri/5} ja \texttt{muutujate\_arv/2} implementeerimisel.

\subsection{Valem}

Valemit käsitletakse programmis kui staatilist struktuuri. Eespool kirjeldatud
lihtsustusteisendused on lahendatud klauslite vaheväärtuste hoidmisega ning
efektiivse muutujate-klauslite indekseerimispõhimõttega.

Valemit säilitatakse termis kujul
{\tt valem(n,m,vc(C1,...,Cn),cv(V1,...,Vm))}, kus {\tt n} on valemi muutujate
arv, {\tt m} on valemi klauslite arv ning termid {\tt vc} ja {\tt cv} on kasutusel järgmisel
otstarbel:

\begin{itemize}
	\item Termis {\tt vc(C1,...,Cn)} hoitakse muutuja indeksi {\tt i} järgi
	klausleid, kuhu antud muutuja kuulub. Iga {\tt Ci} on omakorda term {\tt
	c(Cip, Cin)}, kus {\tt Cip} tähistab nende klauslite listi, kus muutuja
	indeksiga {\tt i} positiivselt esineb ja {\tt Cin} nende klauslite listi, kus
	see muutuja negatiivselt esineb.
	\item Termis {\tt cv(V1,...,Vm)} seevastu hoitakse klausli indeksi {\tt j}
	järgi muutujaid, mis sellesse klauslisse kuuluvad. Siin tuleb sammuti vahet
	teha positiivse ja negatiivse esinemise vahel. Selletõttu oleks sobiv {\tt Vj}
	struktuur {\tt v(Vjp, Vjn)}, kus list {\tt Vjp}({\tt Vjn}) tähistab muutujate
	indekseid, mis esinevad klauslis indeksiga {\tt j} positiivselt(negatiivselt).
\end{itemize}

\textbf{Näide valemi teisendamisest sisekujule:}

\begin{verbatim}
?- teisenda([[-1, 2, -3], [-2, -4], [4]], T).
T = valem(
    4,
    3,
    vc(c([], [1]), c([1], [2]), c([], [1]), c([3], [2])),
    cv(v([2], [1, 3]), v([], [2, 4]), v([4], []))
)
\end{verbatim}

Selline andmestruktuur võimaldab leida konstantse ajaga kõik klauslid, mis
sisaldavad etteantud literaali, samuti mingi kindla klausli (teades klausli
indeksit) kõik literaalid, kuna termi argumendi inspekteerimine protseduuri
\texttt{arg/3} abil on konstantse keerukusega \cite{swi}.

\subsection{Väärtustus}

Programmis hetkel kehtivat väärtustust hoiame struktuuris \\
{\tt olek(v(X1,...,Xn), c(Y1,...,Ym))}, kus
{\tt X1,...,Xn} on vastavate muutujate väärtused ja {\tt Y1,...,Ym} klauslite
väärtused. Tõest väärtust tähistame täisarvuga 1 ja väära täisarvuga 0. Kui
muutujal väärtust ei ole, siis on vastav koht termis lihtsalt tühi. Kuna
korraga saab ühel kohal asuda ainult üks väärtus, siis garanteerib selline
andmestruktuur muutujate oleku ühesuse. Klauslite väärtustamist kasutatakse ära
programmi efektiivsuse tõstmiseks, sest siis ei ole tarvis pidevalt uuesti
välja arvutada klauslite väärtuseid.

\textbf{Näide tühja väärtustuse koostamisest valemi jaoks, milles on 4 muutujat
ja 3 klauslit:}

\begin{verbatim}
?- tyhi_olek(4, 3, S).
S = olek(v(_G256, _G259, _G262, _G265), v(_G276, _G279, _G282))
\end{verbatim}

Kasutades algoritmis ära teadmist viimati väärtustatud muutuja kohta
(\texttt{L} - viimati tõeseks valitud literaal protseduurides
\texttt{kontrolli/6} ja \texttt{propageeri/5}), on selge, et saab praktiliselt
konstantses ajas teha järgmisi operatsioone:

\begin{enumerate}
  \item Ühikklauslite leidmine - vaadata läbi ainult need klauslid, mis
  sisaldasid literaali \texttt{L} komplementaari \texttt{-L}, sest ainult
  sellistes klauslites vähenes esinevate muutujate arv.
  \item Tõeste klauslite leidmine - tõeseks märkida kõik
  klauslid, milles esines \texttt{L}.
  \item Väärade klauslite leidmine - vääraks saavad muutuda ainult klauslid,
  mis sisaldasid \texttt{L} komplementaari \texttt{-L}.
\end{enumerate}

\chapter{Programmi kasutamine}

Programmi kasutamiseks on vajalik arvutisse paigaldatud Prologi interpretaator.
Programm on küll kirjutatud Swi-Prologi realisatsioonile, kuid on lihtsalt
porditav ka teistele populaarsetele Prologi implementatsioonidele.

\section{Põhiprotseduurid}

\subsection{Lahendite loendamine}

Loendamiseks saab kasutada protseduuri \texttt{dpll\_loenda/2}, mis asub failis
\textit{dpll\_loenda.pl}, ning mis võtab sisendargumendiks konjuktiivsel
normaalkujul listiesituses valemi, kus muutujad on kodeeritud täisarvudena.
Lahendite arv antakse väljundmuutujas, mis on
protseduuri teiseks argumendiks.

\textbf{Näide kasutamisest:}

\begin{verbatim}
?- dpll_loenda([[-1, 2, -3], [-2, -4], [4]], C).
C = 3
\end{verbatim}

\subsection{Kehtestava väärtustuse leidmine}

Valemi kehtestatavuse leidmise protseduur asub failis \textit{dpll.pl}.
Põhiprotseduuriks on \texttt{dpll/2}, mis võtab sisendiks valemi samal
kujul nagu loendamiseks kasutatav protseduur. Väljundiks antakse muutujate
väärtustuse termi muutujate osa (andmestruktuuri kirjeldus on esitatud eespool).

Kui mingi muutuja väärtustus pole vajalik kehtestatavuse jaoks, siis esineb
termis sellel kohal Prologi muutuja, st. muutuja väärtust ei kitsendata.

Tagurdamisel genereeritakse automaatselt järgmine väärtustus. Kui valem ei
olnud kehtestatav, siis protseduuri kutse lõpetab negatiivse tulemusega.

\textbf{Näited päringutest:}

\begin{verbatim}
?- dpll([[-1, 2, -3], [-2, -4], [4]], C).
C = v(1, 0, 0, 1) ;
C = v(0, 0, 1, 1) ;
C = v(0, 0, 0, 1) ;

?- dpll([[-1, 2, -3], [-2, -4]], C).
C = v(1, 1, _G778, 0)

?- dpll([[-1], [1]], C).
No

\end{verbatim}

Teine võimalus on kasutada integratsioonimoodulit \texttt{dpllp/2} (failis
\textit{dpllp.pl}), mis võimaldab muutujatena tarvitada Prologi muutujaid
(\texttt{dpllk/2}). Sellisel viisil saab siduda lahendaja teiste Prologi
programmidega. Lisaks sisaldab moodul võimalust kasutada mittekonjuktiivsel
normaalkujul olevaid valemeid (\texttt{dpllp/2}). Kasutatavad operaatorid on
prioriteedi kasvamise järjekorras: \texttt{<=>,=>,v,\&,-}.

\textbf{Näited päringutest:}

\begin{verbatim}
?- dpllk([[-A, B, -C], [-B, -D], [D]]).
A = 1,
B = 0,
C = 0,
D = 1

?- dpllp((-A v B v -C) & (B => -D) & D).

A = 1,
B = 0,
C = 0,
D = 1
\end{verbatim}

\section{cnf failide sisselugemine}

Suuremate valemite (rohkem kui kümned muutujad) hoidmiseks on populaarsed
\texttt{cnf} formaadis failid. \texttt{cnf} formaadis faile saab
sisse lugeda päringuga \texttt{loe\_cnf}, mille esimeseks argumendiks on loetava
faili nimi ning teiseks argumendiks saadav valem.

\textbf{Näide kasutamisest:}

\begin{verbatim}
?- loe_cnf('../testid/ais/ais6.cnf', F), dpll_loenda(F, C).
F = [[1, 2, 3, 4, 5, 6], [-1, -2], [-1, -3],
  [-1, -4], [-1, -5], [-1, -6], [-2, -3], [-2|...], [...|...]|...],
C = 24
\end{verbatim}

\section{Silumisvahend}

Programmi koostamisel oli oluliseks abivahendiks silumismoodul \texttt{silur}.
Silumismoodulit toetavad kõik eespool esitatud lahendus- ja
loendamisprotseduurid. Kui silumine on sisse lülitatud (seda saab teha
päringuga \texttt{silur:sisse.}), siis kirjutab programm kasutajale väljundisse
infot teostatavate tegevuste kohta lahendussammude kaupa. Vaikimisi on
silumisvahend välja lülitatud.

\textbf{Näide siluri väljundist:}

\begin{verbatim}
?- dpll([[-1, 2, -3], [-2, -4], [4]], C).
0 - Valem: [[2, -1, -3], [-2, -4], [4]]
0 > Tõeseks valitud ühikliteraal 4
|:
1 - Valem: [[2, -1, -3], [-2]]
1 > Tõeseks valitud ühikliteraal -2
|:
2 - Valem: [[-1, -3]]
2 > Tõeseks valitud literaal 1
|:
3 - Valem: [[-3]]
3 > Tõeseks valitud ühikliteraal -3
|:
4 - Valem: []
C = v(1, 0, 0, 1)
Yes
\end{verbatim}

\chapter*{Kokkuvõte}
\addcontentsline{toc}{chapter}{Kokkuvõte}

Käesolevas bakalaureusetöös uurisime lausearvutusvalemi kehtestatavate
lahendite loendamise probleemi. Kuna antud probleem on tihedalt seotud
kehtestatavusprobleemiga ja kuivõrd töös vaadeldud algoritm
\textit{DPLL} on aluseks nii loendamise kui kehtestatavuse lahendamiseks, oli
mõistlik neid probleeme koos vaadelda.

Töös esimeses osas esitasime \textit{DPLL} algoritmi üldisel kujul, nagu teda
kasutatakse tänapäeval levinud kõrge efektiivsusega kehtestatavusalgoritmides.
Natuke vaatlesime ka algoritmi originaalkuju ning mehaanilist teoreemide tõestamist -
originaalülesannet, mille lahendamiseks \textit{DPLL} algoritm mõeldud oli.

Töö teises osas andsime algoritmi realisatsiooni tehisintellekti
programmeerimiskeeles Prolog, mis tänu sisseehitatud tagurdusalgoritmile ja
hea sobivuse poolest sümbolarvutuse ülesannete jaoks, sobis suurepäraselt
käesoleva probleemi lahendusalgoritmi implementeerimiseks.

Realiseeritud algoritm on täielik ning seda saab kasutada ka samal otstarbel,
milleks originaalalgoritmi, s.o. tõestusprotseduurides. Implementatsioon on
hõlpsasti integreeritav teiste Prologi programmidega, ning tänu ülevaatlikule
koodile ning realisatsiooniga kaasa tulevale silumisvahendile, kasutatav ka
hariduslikel eesmärkidel \textit{DPLL}-tüüpi algoritmide tutvustamisel.

Paraku ei võimaldanud töö piiratud maht esitada \textit{DPLL} algoritmi mitmeid
edasiarendusi, mis tõstavad lahendusprotseduuri efektiivsust veelgi. Vaatluse
alt jäi välja konfliktide analüüs ja muutuja valiku heuristika, mille abil saab
algoritmi häälestada teatud valdkonnast (näiteks riistvara verifitseerimisega
seotud ülesanded), kus valemid on teatud struktuuriga pärit valemite kiiremaks
lahendamiseks, samuti mitmed võtted lahendite ligikaudseks loendamiseks. Selles
valdkonnas on viimasel ajal toimunud eriti suur areng. Töös pole kajastatud ka
olulisi lahendite loendamise rakendusi, näiteks kaalutud loendamist, mis vastab
tuletusele Bayes'i võrgus, samuti Dedekindi arvude leidmist.

\chapter*{Using DPLL procedure to count models}
\addcontentsline{toc}{chapter}{Using DPLL procedure to count models}

{\textbf{\Large Bachelor thesis (6 ECTS credits)}}

\vspace{1cm}

{\textbf{\Large Raivo Laanemets}}

\vspace{1cm}

{\textbf{\Large Abstract}}

\vspace{1cm}

In this thesis we study the problem of model counting of boolean formulas.
The model counting problem is closely related to boolean satisfiability
problem and it turns out that \textit{DPLL}-like backtracking
satisfiability algorithms are also suitable for the counting problem. We look at
the ideas behind \textit{DPLL} algorithm and implement it in Prolog programming
language. It appears that Prolog is suitable for solving these kind of problems
because of built-in backtracking and general suitability for symbol
manipulations. The developed implementation can also be integrated with other
Prolog programs.



\addcontentsline{toc}{chapter}{Kirjandus}
\begin{thebibliography}{20}

\bibitem[Bir99]{birnbaum99}E.Birnbaum, E.L.Lozinskii, \emph{The Good Old
Davis-Putnam Procedure Helps Counting Models},
Journal of Artifical Intelligence Research 10, 1999, p. 457-477.

\bibitem[Co71]{cook71}S.Cook, \emph{The complexity of theorem-proving
procedures}, Proceedings of the 3rd ACM Symposium on Theory of Computing,
p. 151-158, ACM Press May 1971.

\bibitem[Dav60]{davis60}M.Davis, H.Putnam, \emph{A Computing Procedure for
Quantification Theory},
J. Assoc. Comput. Mach., 7:201-215, 1960.

\bibitem[Dav62]{davis62}M.Davis, G.Logemann, D.Loveland, \emph{A Machine
Program for Theorem-Proving},
Communications of the ACM, 4:394-397, 1962.

\bibitem[Ro96]{roth96}D.Roth, \emph{On the hardness of approximate reasoning},
Artifical Intelligence, 82(1/2):273-302, 1996.

\bibitem[Schö94]{schoning04}U.Schöning, \emph{Logic for Computer Scientists},
Progress in Computer Science and Applied Logic, Birkhäuser 1994.

\bibitem[Si96]{silva96}J.P.M.Silva, K.A.Sakallah, \emph{GRASP - A New Search
Algorithm for Satisfiability}, Proceedings of the International Conference on
Computer-Aided Design, November 1996.

\bibitem[Swi]{swi}J.Wielemaker, \emph{SWI-Prolog 5.6.47 Reference
Manual}, University of Amsterdam 2007.

\bibitem[Tam03]{tamme03}T.Tamme, \emph{Loogilise programmeerimise meetod},
Tartu, 2003.

\bibitem[Tom07]{tombak07}M.Tombak, \emph{Keerukusteooria}, Tartu 2007.

\end{thebibliography}
\chapter*{Lisa 1: programmi dokumentatsioon}
\addcontentsline{toc}{chapter}{Lisa 1: programmi dokumentatsioon}
\setcounter{chapter}{1}

Programmi moodulite dokumentatsiooni koostamiseks on kasutatud Prologi
programmeerijate hulgas laialt levinud stiili, kus predikaatide argumentide
ülesmärkimisel tähistatakse prefikssümboliga {\tt '+'} sisendargumente,
sümboliga {\tt '-'} väljundargumente ning sümboliga {\tt '?'} argumente, mis
võivad olla nii sisendiks kui ka väljundiks. Samuti on üles märgitud ka
argumentide tüübid. Selleks on lisatud argumendi nimele sobiv tüübi nimi kujul
{\tt :tüübinimi}. Tüübinimed on kasutusel ainult dokumentatsiooni
lugemise lihtsustamiseks ning nad ei oma mingit formaalset
tähendust. Ära on toodud ainult moodulitest välja eksporditud predikaatide
dokumentatsioon.

\section{Moodul ''dpllp``}

Moodul DPLL algoritmi kasutamiseks läbi Prolog muutujate.

\begin{itemize}
\item {\tt dpllp(+{\it F})} - Kehtestatavuse lahendaja, mille sisendiks
on konjuktiivsel normaalkujul listiesituses
antud valem. Valemi muutujad, mis on vajalikud
valemi tõesuseks, väärtustatakse väärtusega 1 (tõene)
või 0 (väär).
Näide kasutamisest:

\begin{verbatim}
?- dpllp((A v B) => C).
C = 1 ;
A = 0,
B = 0,
C = 0
?-
\end{verbatim}


\end{itemize}

\section{Moodul ''dpll``}

DPLL kehtestatavusalgoritmi peamoodul.

\begin{itemize}
\item {\tt dpll(+{\it F}:valem, -{\it V})} - DPLL algoritmi realisatsioon. Valem {\it F} on konjuktiivsel
normaalkujul valem, mis on antud listiesituses ja mille
muutujad on kodeeritud täisarvudena. Väljund {\it V} on muutujate
väärtustus, mis on antud andmestruktuuriga {\tt olek}.
Näide kasutamisest:

\begin{verbatim}
?- dpll([[1, 2, 3], [-1]], V1).
V1 = v(0, 1, _G657)
Yes
\end{verbatim}


\end{itemize}

\section{Moodul ''abi``}

Üldiste abimeetodite moodul.

\begin{itemize}
\item {\tt algseadista\_loendur(-{\it Lo})} - Uue loenduri {\it Lo} algseadistamine väärtusele 0.

\item {\tt kl\_vaartuseta\_muutujad(+{\it F}:valem, ?{\it Ci}:int, +{\it S}:olek, -{\it Ps}:list, -{\it Ns}:list)} - Valemist {\it F} klausli {\it Ci} väärtustamata positiivsena esinevate muutujate Ps
ja negatiivsena esinevate muutujate {\it Ns} leidmine.

\item {\tt m\_filtreeri\_vaartuseta(+{\it Ms}:list, +{\it S}:olek, -{\it Ms1}:list)} - Muutujate listist {\it Ms} väärtustamata muutujate väljafiltreerimine
listi {\it Ms1}.

\item {\tt algvaartustus(+{\it Fin}:list, -{\it F}:valem, -{\it S}:olek, -{\it V}:term)} - DPLL algoritmides algandmestruktuuride koostamine.
{\it Fin} - sisendvalem, {\it F} - valem sisekujul, {\it S} - tühi väärtustus,
{\it V} - väärtustuse muutujate osa.

\item {\tt indekslist(+{\it Xs}:list, -{\it Ps}:list)} - Listi {\it Xs} konverteerimine listiks Ps, mille elementideks
on paarid {\it I-E}, kus {\it I} on elemendi täisarvuline indeks
ja {\it E} on element ise. Indeksid algavad väärtusega 1 ja väljendavad
elemendi esinemispositsiooni listi algusosast alates.

\end{itemize}

\section{Moodul ''vaartustus``}

Väärtustuse lisamiseks/kontrollimiseks kasutatvad protseduurid.

\begin{itemize}
\item {\tt tyhi\_vaartustus(+{\it F}:valem, -{\it V}:olek)} - Valemi {\it F} muutujate jaoks tühja väärtustuse koostamine.

\item {\tt muutuja\_vaartuseta(+{\it V}:olek, +{\it I}:int)} - Kontrollimine, kas etteantud indeksiga muutuja
on vaartustuses {\it V} väärtustatud hetkel või mitte.

\item {\tt klausel\_vaartuseta(+{\it V}:olek, +{\it I}:int)} - Kontrollimine, kas etteantud indeksiga klausel
on vaartustuses {\it V} väärtustatud hetkel või mitte.

\item {\tt vaartusta\_muutuja(+{\it V}:olek, +{\it I}:int, +{\it Val})} - Etteantud indeksiga {\it I} muutuja väärtustamine
väärtusega {\it Val} väärtustuses {\it V}.

\item {\tt vaartusta\_klausel(+{\it V}:olek, +{\it I}:int, +{\it Val})} - Klausli indeksiga {\it I} väärtustamine
väärtusega {\it Val} väärtustuses {\it V}.

\end{itemize}

\section{Moodul ''valem``}

Valemiga tegevuste sooritamise moodul.

\begin{itemize}
\item {\tt klausli\_muutujad(+{\it F}:valem, +{\it I}:int, -{\it Ps}:list, -{\it Ns}:list)} - Leia valemist {\it F} klauslis indeksiga {\it I} positiivse
literaalina esinevate muutujate indeksid Ps
ja negatiivsena esinevate muutujate indeksid {\it Ns}.

\item {\tt klausli\_pos\_muutujad(+{\it F}:valem, +{\it I}:int, -{\it Ps}:list)} - Leia valemist {\it F} klauslis indeksiga {\it I} positiivse
literaalina esinevate muutujate indeksid {\it Ps}.

\item {\tt klausli\_neg\_muutujad(+{\it F}:valem, +{\it I}:int, -{\it Ps}:list)} - Leia valemist {\it F} klauslis indeksiga {\it I} negatiivse
literaalina esinevate muutujate indeksid Ns.

\item {\tt muutuja\_klauslid(+{\it F}:valem, +{\it I}:int, -{\it Ps}:list, -{\it Ns}:list)} - Leia valemist {\it F} klauslite indeksid Ps,
kus muutuja indeksiga {\it I} esineb positiivselt ja klauslite
indeksid Ns, kus muutuja indeksiga {\it I} esineb negatiivselt.

\item {\tt pos\_muutuja\_klauslid(+{\it F}:valem, +{\it I}:int, -{\it Ps}:list)} - Leia valemist {\it F} klauslite indeksid Ps,
kus muutuja indeksiga {\it I} esineb positiivselt.

\item {\tt neg\_muutuja\_klauslid(+{\it F}:valem, +{\it I}:int, -{\it Ps}:list)} - Leia valemist {\it F} klauslite indeksid Ns,
kus muutuja indeksiga {\it I} esineb negatiivselt.

\item {\tt teisenda(+{\it F1}:valem, -{\it F2}:valem)} - Konjuktiivsel normaalkujul listiesituses valemi {\it F1} teisendamine sisekujule {\it F2}.
Näide kasutamisest:

\begin{verbatim}
?- teisenda([[-1, 2, -3], [-2, -4], [4]], T).
T = valem(
4,
3,
vc(c([], [1]), c([1], [2]), c([], [1]), c([3], [2])),
cv(v([2], [1, 3]), v([], [2, 4]), v([4], []))
)
\end{verbatim}


\end{itemize}

\section{Moodul ''kontrolli``}

Valemi kontrollimine vahetult pärast muutuja väärtustamist.

\begin{itemize}
\item {\tt kontrolli(-{\it O}:otsus, +{\it L}:int, +{\it F}:valem, +{\it V}:olek, +{\it Ct}:int, +{\it Ys}:list)} - Kontrolli hetkel väärtustust viimati tehtud
lahendussammu järgi. {\it O} - väljund, {\it L} - viimati tõeseks
valitud literaal, {\it F} - valem, mida lahendatakse, {\it V} - hetkel
kehtiv väärtustus, {\it Ct} - seni tõeseks saadud klauslite arv,
{\it Ys} - ühikliteraalide pinu.
Väljund {\it O} võib olla üks järgmistest: \texttt{sat} - valem on tõene;
\texttt{yhik(Ls)} - võta järgmistes sammudes tõeseks ühikliteraalid \texttt{Ls};
\texttt{vali} - vali järgmises sammus suvaline muutuja.

\end{itemize}

\section{Moodul ''propageeri``}

Protseduur muutuja väärtustuse valimise propageerimiseks.

\begin{itemize}
\item {\tt propageeri(+{\it L}:int, +{\it F}:valem, +{\it V}:olek, -{\it Cf}:int, -{\it Ct}:int)} - Etteantud tõese literaali {\it L} väärtuse propageerimine
(valemi {\it F} vastava muutuja/klauslite väärtustamine).
Väljundparameetrid: {\it Cf} - saadud väärade klauslite arv;
{\it Ct} - saadud tõeste klauslite arv.

\end{itemize}

\section{Moodul ''silur``}

Programmi silumist abistavad protseduurid.

\begin{itemize}
\item {\tt sisse} - Siluri väljundi näitamise lubamine. Kui siluri väljundi näitamine
on lubatud, kirjutatakse standardväljundisse infot algoritmi sammude
kohta.

\item {\tt valja} - Siluri väljundi näitamise keelamine. Kui siluri väljundi näitamine
on lubatud, kirjutatakse standardväljundisse infot algoritmi sammude
kohta.

\end{itemize}

\section{Moodul ''knk``}

Prologi muutujatega antud valemi teisendamine listiesituses konjuktiivsele normaalkujule.

\begin{itemize}
\item {\tt knk(+{\it F}:valem, -{\it G}:valem)} - Lausearvutusvalemi {\it F} teisendamine konjuktiivsele
normaalkujule {\it G}. Protseduur aksepteerib loogiliste
operaatoritena \texttt{-} (eitus), \texttt{v} (disjunktsioon)
\texttt{\&} (konjuktsioon), \texttt{=>} (implikatsioon),
\texttt{<=>} (ekvivalents). Operaatorite prioriteedid kasvavalt:
\texttt{-}, \texttt{\&}, \texttt{v}, \texttt{=>}, \texttt{<=>}.
Protseduur ei eemalda valemist tautoloogiaid.
Näide kasutamisest:

\begin{verbatim}
?- knk((A v B) => C, W).
W = (-A v C)& (-B v C)
\end{verbatim}


\item {\tt konj\_list(+{\it F}:valem, -{\it L}:list)} - Valemi {\it F} teisendamine konjuktiivsele normaalkujule.
Protseduur aksepteerib neidsamu operaatoreid, mida \texttt{knk/2}.

\end{itemize}



\end{document}
