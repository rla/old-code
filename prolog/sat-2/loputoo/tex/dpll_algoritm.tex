\chapter{DPLL algoritm}

\textit{DP (Davis-Putnam)} algoritm \cite{davis60} oli üks esimesi
süstemaatilisi algoritme esimest järku predikaatarvutuse valemitega kirja pandud teoreemide
tõestamiseks arvuti abil. Olgugi, et algoritm on küllaltki ebaefektiivne võrreldes
paar aastat hiljem (1965) avaldatud Robinsoni algoritmiga, on tarvitatud sellest
algoritmist pärit meetodeid hiljem kehtestatavuse algoritmide arendamiseks.
\textit{DP} kasutas loogilise tõesuse kontrollimise asemel valemi eituse
mittekehtestatavuse väljaselgitamist, mis on esimesega samaväärne probleem.

Töö piiratud maht võimaldab anda ainult väga põgusa ülevaate
originaalalgoritmist. Algoritm koosnes neljast osast:

\begin{enumerate}
  \item Esialgse valemi $F$ eituse $\neg F$ leidmine.
  \item Valemi $\neg F$ kirjutamine prenex-kujule\footnotemark[2] $G_p$
  (prefikskuju leidmine).
  \item $G_p$ eksistentsikvantorite eemaldamine
  vastavate muutujate funktsionaalsümbolitega asendamise teel
  (skolemiseerimine\footnotemark[2]). Saadakse valem $G_s$. Tähtis on teada,
  et $G_p$ ja $G_s$ ei ole loogiliselt ekvivalentsed, kuid on mittekehtestatavuse suhtes ekvivalentsed, st. $G_p$ on mittekehtestatav parajasti siis, kui $G_s$ on mittekehtestatav.
  \item Valemist $G_s$ muutuja- ja kvantorivabade lausearvutusvalemite
  $G_{s,1},G_{s,2},\dots$ genereerimine kuni nendest $n$ esimese valemi
  konjuktsioon $G_{s,1}\wedge\dots\wedge G_{s,n}$ on mittekehtestatav. 
\end{enumerate}

\footnotetext[2]{Prenex-kuju/skolemiseerimine jt.
valdkonnaga seotud mõisted seletatakse lahti allikas
\cite{schoning04}.}

Algoritm töötas nii kaua, kuni sammus (4) oli genereeritud piisaval hulgal
kvantori- ja muutujavabu valemeid. On ära näidatud, et kui esialgne valem $F$
on loogiliselt tõene, siis leidub lõplik kogus valemeid $G_{s,1},G_{s,2},\dots$
ja programm lõpetab oma töö, vastasel korral jääbki ta valemeid genereerima.
Kuna predikaatloogika on poollahenduv, siis see ongi parim tulemus, mille me
võime saada.

Reeglid sammus (4) valemite konjuktsiooni $W=G_{s,1}\wedge\dots\wedge G_{s,n}$
kehtestatavuse kindlakstegemiseks olid järgmised:

\begin{enumerate}
  \item \emph{Ühikklauslite elimineerimine}
  	\begin{enumerate}
        \item Kui valem $W$ sisaldab ühikklauslit $C_i$ literaaliga $l$ ja
        samuti ühikklauslit literaaliga $\bar{l}$, siis sisaldab $W$ vastuolu
        ning on mittekehtestatav.
        \item Kui (a) ei ole rakendatav ja valem $W$ sisaldab ühikklauslit $C_i$
        literaaliga $l$, siis kustutada valemist $W$ kõik klauslid, mis sisaldavad literaali $l$ ja
        igast klauslist, mis sisaldab $l$ komplementaari $\bar{l}$,
        kustutada $\bar{l}$. Saadav valem $W'$ on mittekehtestatav parajasti
        siis, kui seda on $W$.
   \end{enumerate}
   \item \emph{Puhta literaali elimineerimine}\\
        Kui muutuja $x$ esineb valemi $W$ klauslites ainult
        positiivse või negatiivse literaalina, siis võib kõik muutujat $x$
        sisaldavad klauslid eemaldada, saades valemi $W'$, mis on mittekehtestatav parajasti siis,
        kui seda on $W$.
   \item \emph{Muutujate elimineerimine}\\
   		Kui valem $W$ on kujul $W=(A\vee l)\wedge(B\vee \bar{l})\wedge R$, kus $A$,
   		$B$ ja $R$ literaali $l$ ega tema komplementaari $\bar{l}$ ei sisalda,
   		siis võib valemi $W$ ümber kirjutada kujule $W'=(A\vee B\vee)\wedge R$.
   		Valem $W$ on mittekehtestatav parajasti siis, kui $W'$ on mittekehtestatav.
\end{enumerate}

Hiljem publitseeritud artiklis \cite{davis62} asendatakse
muutujate elimineerimisreegel teistsugusel kujul oleva nn.
\emph{jaotamisreegliga}:

Kui valem $W$ on kujul $W=(A\vee l)\wedge(B\vee \bar{l})\wedge R$, kus $A$,
$B$ ja $R$ literaali $l$ ega tema komplementaari $\bar{l}$ ei sisalda,
siis võib valemi $W$ jaotada valemiteks $W_1=A\wedge R$ ja $W_2=B\wedge R$ ning
esialgne valem on mittekehtestatav parajasti siis kui mõlemad $W_1$ ja $W_2$ on
mittekehtestatavad.

Koos selle reegliga moodustab \textit{DP} algoritm algoritmi \textit{DPLL}.

Hetkel levinud \textit{DPLL}-tüüpi kehtestatavuse algoritmid kasutavad
originaalsest algoritmist erinevat lähenemist. Eelkõige on algoritmid
realiseeritud tagurdusalgoritmidena, mis töötavad muutujate osalise
väärtustusega, samm-sammult valides muutujatele väärtused, lihtsustades
valemit pärast iga väärtustamist, kuni saadakse selline lõpptulemus, mille 
korral valemit võib lugeda seni leitud väärtustuse piires tõeseks.


\section{Boole-Shannoni dekompositsioon}

Boole-Shannoni dekompositsioon (lahutus) vastab \textit{DPLL} algoritmis
jaotamisreeglile. Sisuliselt on siin tegemist universaalse vahendiga
$n$-muutuja ($n>0$) Boole'i funktsiooni avaldamiseks kahe, või erijuhul ühe,
$n-1$ muutuja Boole'i funktsiooni kaudu.

Olgu antud antud $n$-muutuja Boole'i funktsioon $f(x_1,\dots,x_n)$. Sellest
funktsioonist võime saada kaks $n-1$-muutja funktsiooni, asendades muutuja
$x_i$ väärtusega 1 esimeses ja 0 teises. Nendeks funktsioonideks on:

$$f_{\bar{x_1}}(x_1,\dots,x_{i-1},x_{i+1}\dots,x_n)=f(x_1,\dots,x_{i-1},0,x_{i+1}\dots,x_n)$$

ja

$$f_{x_1}(x_1,\dots,x_{i-1},x_{i+1}\dots,x_n)=f(x_1,\dots,x_{i-1},1,x_{i+1}\dots,x_n).$$

Sealjuures kehtib ka samaväärsus

$$f=f_{\bar{x_1}}\vee f_{x_1},$$

mille korrektsuse tõestuse võib leida allikast \cite{tombak07}. Põhimõtteliselt
on võimalik kehtestatavuse algoritm koostada ainult dekompositsiooni kasutades,
sest rakendades seda rekursiivselt järjest igale alamvalemile, jõuame ükskord
olukorrani, kus kõik muutujad on asendatud väärtustega 1 või 0, ning saame
valemi väärtuse välja arvutada. Paraku oleks see väga ebaefektiivne algoritm,
võrreldav tõeväärtustabeli koostamisega.

Vaatleme dekompositsiooni rakendust KNK valemile $F=(x\vee\neg y\vee z) \wedge
(\neg x\vee y\vee\neg z) \wedge (x\vee\neg y\vee\neg z) \wedge (\neg x\vee\neg
y\vee\neg z)$ muutuja $y$ järgi. Saame valemid:

$$F_y=(x\vee 0\vee z) \wedge
(\neg x\vee 1\vee\neg z) \wedge (x\vee 0\vee\neg z) \wedge (\neg x\vee 0\vee\neg
z)$$
ja
$$F_{\bar{y}}=(x\vee 1\vee z) \wedge
(\neg x\vee 0\vee\neg z) \wedge (x\vee 1\vee\neg z) \wedge (\neg x\vee 1\vee\neg
z).$$

Oh lihtne märgata, et saadud valemeid oleks võimalik oluliselt lihtsustada.
Üldiselt saame lihtsustamise põhimõtted kirja panna lihtsa eeskirjana, mis
lihtsustab dekompositsiooni $F \Leftrightarrow F_x \vee F_{\bar{x}}$ valemid
$F_x$ ja $F_{\bar{x}}$:

\begin{enumerate}
  \item Valemist $F_x$ eemalda kõik klauslid, kus muutuja $x$ esineb
  positiivselt ja kõikidest $F_x$ klauslitest, kus $x$ esineb negatiivselt,
  eemalda $x$.
  \item Valemist $F_{\bar{x}}$ eemalda kõik klauslid, kus muutuja $x$ esineb
  negatiivselt ja kõikidest $F_{\bar{x}}$ klauslitest, kus $x$ esineb positiivselt,
  eemalda $x$.
\end{enumerate}

See algoritm võimaldab lihtsustada $F_x$ ja $F_{\bar{x}}$ n.ö. vahepealse
väljarvutamiseta, kuigi asendades $x$ väärtuse valemisse sisse, saame valemi
lihtsustada ka tuntud lausearvutuse samaväärsusi $x\vee 0 \equiv x$ ja $x\vee 1
\equiv 1$ kasutades.

Võttes eespool toodud näite uuesti ette, rakendame lihtsustusi leitud valemitele
$F_y$ ja $F_{\bar{y}}$, saades semantiliselt samaväärsed, kuigi süntaktiliselt
märksa lihtsamad KNK valemid:

$$F_y=(x\vee z)\wedge (x\vee\neg z)\wedge(\neg x\vee\neg z)$$
ja
$$F_{\bar{y}}=(\neg x\vee\neg z).$$

Dekompositsiooni ei saa rakendada kuitahes palju arv kordi, sest
valemi muutujate arv väheneb iga korraga ühe võrra (kuigi lihtsustuste
tõttu võivad kaduda valemist ka need muutujad, mille järgi ei ole lahutust
rakendatud). On võimalik kahe erineva olukorra tekkimine:

\begin{enumerate}
  \item[(*a)] Valemist eemaldatakse kõik klauslid - sellisel juhul ei saa valem
  enam vääraks muutuda, sest ei ole võimalik, et mõni valemi klausel oleks
  väär. Sellisel juhul on valem tõene.
  \item[(*b)] Valemisse juhtub sisse jääma klausel, millest on eemaldatud kõik
  muutujad - selline valem on väär, kuivõrd temas leidub väär klausel.
\end{enumerate}

Rakendades dekompositsiooni näites saadud valemile $F'_{\bar{y}}$ edasi muutuja
$x$ järgi, saame omakorda valemid $F_{\bar{y},x}=\neg z$ ja
$F_{\bar{y},\bar{x}}$, millest viimane ei sisalda ühtegi klauslit ja on
seetõttu tõene ning seega oleme leidnud kehtestava väärtustuse
$\{y=0,x=0\}$.

Pannes kokku nii väärtustamise kui valemite lihtsustamise, saab koostada
järgmise algoritmi (1):

\newpage

\begin{algorithm}{SAT($F$)}
\ainputoutput{konjuktiivsel normaalkujul valem $F$.}{tulemus, kas valem $F$ on
kehtestatav.}
\abody

\begin{algorithmic}[1]
\IF {$F$ ei sisalda ühtegi klauslit}
	\RETURN \TRUE
\ELSIF {$F$ sisaldab tühja klauslit}
	\RETURN \FALSE
\ELSE
	\STATE Vali valemist $F$ muutuja $x$.
	\STATE Rakenda valemile $F$ dekompositsiooni $x$ järgi, saades valemid $F_x$
	ja $F_{\bar{x}}$.
	\STATE Lihtsusta valemeid $F_x$ ja $F_{\bar{x}}$.
	\RETURN \proc{SAT($F_x$)}$\vee$\proc{SAT($F_{\bar{x}}$)}.
\ENDIF
\end{algorithmic}
\end{algorithm}

Esitatud rekursiivsest pseudokoodist on näha, et ridadel (1-2) kontrollitakse
eespool antud lõpptingimuse (*a) täidetust, kus valemi $F$ kõik klauslid on
tõeseks muutunud ja lihtsustamiste tulemusena ei ole valemisse $F$ enam
ühtegi klauslit jäänud. Sellisel juhul tagastab protseduur \texttt{true},
st. etteantud valem on kehtestatav.

Ridadel (3-4) kontrollitakse seevastu lõpptingimuse (*b) täidetust, st.
inspekteeritavasse valemisse on sattunud väär klausel. Sellisel juhul ei ole
enam mõtet midagi edasi arvutada, sest valemit ei õnnestu ühelgi viisil
tõeseks muuta. Protseduur tagastab \texttt{false}, st. etteantud valem $F$ on
mittekehtestatav.

Kui protseduuri kutsungist ei väljutud ridade (1-4) kaudu, siis rakendadakse
valemile Boole-Snannoni dekompositsiooni (6-7), vahetult mille järel toimub
lihtsustamine (8). See annab kaks valemit $F_x$ ja $F_{\bar{x}}$, milles on
vähem muutujaid, kui esialgses valemis $F$. Real (9) rakendatakse mõlemale
alamvalemile algoritmi \proc{SAT} rekursiivselt.

Nagu eespool mitu korda mainitud, realiseeritakse sellised algoritmid
tagurdusalgoritmidega. Tagurduse valikupunktiks on siin rida (9). Siin on näha,
et valik tuleb teha kahe valemi $F_x$ ja $F_{\bar{x}}$ vahel, millele algoritmi
rekursiivselt edasi rakendada. Kui me oskaks kohe valida õige valemi, oleks
sellise algoritmi keskmine keerukus lineaarne muutujate arvu suhtes.
Paraku me ei oska õiget valemit valida ja selletõttu tuleb halvimal juhul mõlemad valemid läbi
proovida, aga see muudab algoritmi tööaja juba eksponentsiaalseks. On näidatud,
et leidub isegi terve valemite klass, mille jaoks see algoritm töötab alati
eksponentsiaalses ajas \cite{tombak07}.

\subsection{Ühikklauslite elimineerimine}

Boole-Shannoni dekompositsioonil on erikuju. Olgu $F$ konjuktiivsel
normaalkujul olev valem. Kui $F$ sisaldab sellist klauslit $C$, mis koosneb
ainult ühest literaalist, st. $C=l$ ja $l\equiv x$, siis võib sooritada $F$
lahutuse ($F_x$, $F_{\bar{x}}$) muutuja $x$ järgi ja ignoreerida valemit
$F_{\bar{x}}$, sest see on samaselt väär. Seda põhjendab asjaolu, et
väärtustusel $v(x)=0$, $C=l=0$. See aga tähendab omakorda, et valem
$F_{\bar{x}}$ sisaldab väära klauslit. Täpselt sarnane olukord kehtib ka juhul
kui $l\equiv\neg x$, aga siis võime vaatluse alt välja jätta valemi $F_x$.

Ühikklauslite elimineerimine kui Boole-Shannoni lahutuse erikuju vastab
originaalses \textit{DPLL} algoritmis ühikklauslite elimineerimisreegli juhule
(b). Selle reegli saame lisada otse algoritmi (1), saades juba märksa
efektiivsema, kuigi siiski veel eksponentsiaalses ajas töötava, algoritmi:

\begin{algorithm}{DPLLSAT($F$)}
\ainputoutput{konjuktiivsel normaalkujul valem $F$.}{tulemus, kas valem $F$ on
kehtestatav.}
\abody

\begin{algorithmic}[1]
\IF {$F$ ei sisalda ühtegi klauslit}
	\RETURN \TRUE
\ELSIF {$F$ sisaldab tühja klauslit}
	\RETURN \FALSE
\ELSIF {$F$ sisaldab ühikklauslit $C_i=l$}
	\STATE Rakenda valemile $F$ dekompositsiooni $x$ järgi, saades valemid
	$F_x$ ja $F_{\bar{x}}$.
	\IF {$l\equiv x$}
		\STATE Lihtsusta valemit $F_x$.
		\RETURN \proc{DPLLSAT($F_x$)}.
	\ELSE
		\STATE Lihtsusta valemit $F_{\bar{x}}$.
		\RETURN \proc{DPLLSAT($F_{\bar{x}}$)}
	\ENDIF
\ELSE
	\STATE Vali valemist $F$ muutuja $x$.
	\STATE Rakenda valemile $F$ dekompositsiooni $x$ järgi, saades valemid $F_x$
	ja $F_{\bar{x}}$.
	\STATE Lihtsusta valemeid $F_x$ ja $F_{\bar{x}}$.
	\RETURN \proc{DPLLSAT($F_x$)}$\vee$\proc{DPLLSAT($F_{\bar{x}}$)}.
\ENDIF
\end{algorithmic}
\end{algorithm}

\subsection{Puhta literaali elimineerimine}

Puhta literaali elimineerimine on lisaks ühikklausli elimineerimisele veel üks
täiendav heuristiline reegel kehtestatavuse leidmiseks. Olgugi et see esines
originaalses \proc{DPLL} algoritmis, on temast hilisemates variantides
loobutud eelkõige selle tõttu, et praktikas on osutunud tema kasutamine
ebaefektiivse realisatsiooni tõttu vähetõhusaks.

Puhta literaali $l$ elimineerimine kujub endast valemi $F=C_1\wedge\dots\wedge
C_k$ klauslitest nende eemaldamist, kus $l$ esineb. Sealjuures ei tohi
üheski klauslis $C_1\wedge\dots\wedge C_k$ esineda $l$ komplementaari $\bar{l}$.
Reegli õigustuse võib leida allikast \cite{tombak07}. Puhta literaali reeglile
vastab Boole-Shannoni dekompositsiooni rakendamine valemile $F$ literaalile $l$
vastava muutuja $x$ järgi. Siis ütleb reegel, et me võime ignoreerida valemit
$F_{\bar{x}}$, kui $l\equiv x$ ja valemit $F_x$ juhul $l\equiv -x$.

Paraku on puhta literaali eemaldamisel mittekasulik omadus
valemi lahendite arvu suhtes. Nimelt kaotame selle kasutamisel osa lahenditest,
sest võib juhtuda, et valem võib olla tõene ka juhul, kui puhas literaal $l$ on
väär, st. esialgne valem ja pärast teisendust saadud valem ei ole loogiliselt
samaväärsed. Näitena sobib valem $F=x\vee y$, millest saaksime rangelt
puhta literaali elimineerimist tarvitades lahendi $x=1, y=1\}$, aga sealjuures
ignoreeriksime täielikult lahendeid $\{x=0, y=1\}$ ja $\{x=1, y=0\}$.

See halb omadus on põhjuseks, miks puhta literaali elimineerimise reegel on
käesolevas töös vaadeldud \textit{DPLL} algoritmi variandist välja jäetud.
Teiseks põhjuseks on valemist puhta literaali otsimise arvutuslik keerukus
valemi andmestruktuurist, mis on esitatud töö teises pooles, sest andmestruktuur on
optimeeritud ühikliteraali leidmiseks ning valemi kiireks lihtsustamiseks
vahetult pärast muutuja valikut.


\section{Lahendite loendamine}

Seni oleme töös vaadelnud ainult kehtestatavuse algoritmi. Selgub, et
lahendite loendamiseks tuleb seda (2) ainult vähe muuta.
Loendamine kasutab ära asjaolu, et ülalkirjeldatud algoritm üritab leida
kehtestatavat väärtustust üksikute muutujate väärtustamise kaupa, kusjuures
alati lõpetab protseduur ühes kahest olukorrast (*a, *b), mida on töös eespool
juba kirjeldatud.

On ilmselt selge, et valemi lahendite arv (kujutades ette protseduuri
\proc{SAT} või \proc{DPLLSAT} rakendamist valemile $F$ ja vahetult olukorra (*a)
või (*b) esinemist) on neil juhtudel järgmine:

\begin{itemize}
  \item[(*a):] Valemis $F$ oli esialgu $n$ muutujat, nendest
  on ära väärtustatud $k$ tükki. See tähendab, et ülejäänud $t=n-k$ muutujale võib anda
  suvalise väärtustuse, ilma et peaks kartma, et valem $F$ vääraks muutuks. $t$
  muutujale saab kokku anda $2^t$ erinevat väärtustust, seega on $F$
  lahendite arv $2^{n-k}$.
  \item[(*b):] Kuna valem $F$ on väär, siis on tema lahendite arv 0.
\end{itemize}

Vastavalt nendele juhtudele on vaja modifitseerida algoritmi \proc{SAT} või
\proc{DPLLSAT} ainult niipalju, et anda sisendparameetrina kaasa valemi $F$
esialgne lahendite arv $n$ ja seni väärtustatud muutjate arv $k$. Protseduuri
rekursiivses sammus (\proc{SAT}, rida 9; \proc{DPLLSAT}, rida 18) tuleb
loomulikult igal sammul väärtustatud muutujate arvule juurde liita 1 . Saadav
algoritm on esitatud järgnevalt:

\begin{algorithm}{DPLL\_LOENDA($F,N,K$)}
\ainputoutput{knk valem $F$, $F$ muutjate arv $N$, väärtustatud muutujate arv
$K$.}{valemi $F$ lahendite arv $S$.}
\abody

\begin{algorithmic}[1]
\IF {$F$ ei sisalda ühtegi klauslit}
	\RETURN $2^{N-K}$
\ELSIF {$F$ sisaldab tühja klauslit}
	\RETURN 0
\ELSIF {$F$ sisaldab ühikklauslit $C_i=l$}
	\STATE Rakenda valemile $F$ dekompositsiooni $x$ järgi, saades valemid
	$F_x$ ja $F_{\bar{x}}$.
	\IF {$l\equiv x$}
		\STATE Lihtsusta valemit $F_x$.
		\RETURN \proc{DPLL\_LOENDA($F_x$)}.
	\ELSE
		\STATE Lihtsusta valemit $F_{\bar{x}}$.
		\RETURN \proc{DPLL\_LOENDA($F_{\bar{x}}$)}
	\ENDIF
\ELSE
	\STATE Vali valemist $F$ muutuja $x$.
	\STATE Rakenda valemile $F$ dekompositsiooni $x$ järgi, saades valemid $F_x$
	ja $F_{\bar{x}}$.
	\STATE Lihtsusta valemeid $F_x$ ja $F_{\bar{x}}$.
	\RETURN \proc{DPLL\_LOENDA($F_x$)}$ + $\proc{DPLL\_LOENDA($F_{\bar{x}}$)}.
\ENDIF
\end{algorithmic}
\end{algorithm}

Sarnaselt algoritmile \proc{DPLLSAT}, saab antud protseduuri kirja panna
tagurdusalgoritmina, mida on tehtud ka töö teises pooles realiseeritud
programmis. Selleks piisab võtta kasutusele globaalne muutuja, mille
väärtust suurendada iga kord $2^{n-k}$ võrra ($n$ - muutujate koguarv, $k$ -
väärtustatud muutujate arv), kui algoritm leiab kehtestava väärtustuse.
Seejärel tuleb lasta algoritmil tagurdada kuni viimase rekursioonitasemeni,
milles pole läbi vaadatud mõlemat valemit. (rida 18 algoritmis
\proc{DPLL\_LOENDA}) ning pärast protseduuri rakendamise lõppu lugeda selle
muutuja väärtus, mis annabki lahendite arvu.

