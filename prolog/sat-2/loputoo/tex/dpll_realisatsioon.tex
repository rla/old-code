\chapter{Loendamisalgoritmi realisatsioon}

Selles töö osas anname eespool kirjeldatud algoritmi realisatsiooni
keeles Prolog. Prolog sai valitud tema mitmete heade omaduste
tõttu, mida saab edukalt ära kasutada sellist tüüpi ülesannete lahendamisel.
Prolog baseerub ise tagurdusalgoritmil ja see vähendab algoritmi
kirjapanekul tööd oluliselt, sest programmeerija ise ei pea
tegelema selle keerulise osaga, mida on tagurdamisel muutujate
väärtuste efektiivne taastamine. Lisaks on Prolog tuntud kui
sümbolmanipuleerimise keel, peale selle ei ole vaja Prologis
programmeerijal muretseda korrektse mäluhalduse pärast, sest Prolog
sisaldab automaatset mälukoristust. Need omadused teevad Prologist keele,
mis on vägagi sobiv keeruliste algoritmide
realiseerimiseks või prototüüpimiseks. Samuti on oluline mainida, et
Prolog on populaarne keel tehisintellekti valdkonda kuuluvate ülesannete
lahendamiseks. See muudab parajasti lihtsaks antud algoritmi integreerimise ja
laiendamise teiste tehisintellekti probleemide jaoks.

Käesoleva töö praktilise osa täielikuks mõistmiseks on tarvis osata lugeda
Prologi programme. Väga heaks sissejuhatavaks materjaliks (millest ka täiesti
piisab töö mõistmiseks), on allikas \cite{tamme03}.

Realisatsiooni kirjeldust alustame programmi üldise struktuuri esitamisega,
kusjuures ei ole isegi olulised kasutatavad andmestruktuurid, mille
kirjelduse anname alles mõnevõrra hiljem.

Selles töös esitatud viis \textit{DPLL} kehtestatavus- ja loendamisalgoritmi
jaoks ei ole loomulikult ainuõige võimalus programmi realiseerimiseks. Samas on
see lähenemine töö autori arvates sobiv Prolog-tüüpi keeles algoritmi
implementeerimiseks. Imperatiivses programmeerimiskeeles, näiteks C
keeles, on märksa levinum kasutada lähenemist, mille pseudokood on toodud
artiklis \cite{silva96}.

\newpage

\section{Realisatsiooni ülesehitus}

Algoritmi realiseerimisel sai koostatud lihtne tagurdusel põhinev
programm. Kasutatud andmestruktuurid seletatakse põhjalikumalt lahti selleks
pühendatud töö sektsioonis pärast algoritmi kirjeldust.

\subsection{Muutujate kodeerimine}

Programmis esitame lahendatava valemi muutujad täisarvudena 1,2,\ldots. See on
levinud viis, kuna võimaldab mugavat indekseerimist lineaarsetes andmestruktuurides,
näiteks massiivides. Samuti saame kompaktselt ja elegantselt esitada literaale.
Muutuja positiivse esinemise tähistamiseks tuleb kasutada lihtsalt
muutujale vastavat täisarvu, negatiivse esinemise tarvis aga vastandarvu.

See kodeerimismeetod pakub triviaalsed lahendused operatsioonidele nagu
literaali komplementaari leidmine jt. Olgugi et Prolog kui predikaatarvutust
põhjana kasutav keel otseselt massiive andmestruktuuridena ei paku, leidub
alternatiivne lähenemine termide inspekteerimise vahendite näol. 

\subsection{Põhiprotseduur}

Eespool esitatud algoritm \proc{DPLLSAT} on kirja pandud
kahe protseduuri järjest rakendamisel saadava rekursiivse programmi abil:

\begin{lstlisting}[numbers=left,xleftmargin=1cm,basicstyle=\tt]
dpll(L, F, S, Ct, D, Ys):-
	kontrolli(O, L, F, S, Ct, Ys),
	samm(O, F, S, Ct, D).
\end{lstlisting}

Selles protseduuris on muutujate kirjeldused järgmised (kõik muutujad on
sisendparameetrid):

\begin{itemize}
  \setlength{\itemsep}{1mm}
  \item [\texttt{L} -] viimati tõeseks valitud literaal;
  \item [\texttt{F} -] lahendatav valem;
  \item [\texttt{S} -] hetkel kehtiv muutujate ja klauslite väärtustus;
  \item [\texttt{Ct} -] hetkel tõeste klauslite arv;
  \item [\texttt{D} -] seni sooritatud lahendussammude arv, võrdne seni
  väärtustatud muutujate arvuga;
  \item [\texttt{Ys} -] ühikliteraalide vahepinu (sisaldab literaale, mis on
  programmi töö käigus ühikliteraalideks muutunud ja tuleks deskompositsiooni
  rakendades esmajärjekorras ära kasutada);
  \item [\texttt{O} -] protseduuri \texttt{kontrolli/6} rakendamisel saadav
  väärtus, mille põhjal tehakse järgmine lahendussamm.
\end{itemize}

Protseduur \texttt{kontrolli/6} sisuliselt vaatab järgi, kas hetkel kehtiva
väärtustuse piires on valem \texttt{F} juba tõeseks muutunud või mitte.
Põhimõtteliselt on võimalikud kolm olukorda:

\begin{enumerate}
  \item Valem \texttt{F} on tõeseks muutunud, st. kõik tema klauslid sisaldavad
  vähemalt ühte tõest literaali. Sellisel juhul on leitud kehtestav väärtustus
  ja programm lõpetab oma töö ära. \texttt{kontrolli/6} väljund on siis
  \texttt{O = sat}.
  \item Valemis \texttt{F} leiduvad ühikklauslid ning sooritada tuleb
  dekompositsioon nende järgi. Sellele olukorrale vastab \texttt{kontrolli/6}
  väljund \texttt{O = yhik(Ls)}, kus \texttt{Ls} on list literaalidest, mis
  ühikklauslitena esinevad. \texttt{kontrolli/6} on koostatud niimoodi, et see
  list ei saa kunagi tühi olla. Kui protseduuri \texttt{dpll/6} kutsudes
  sisaldas loend \texttt{Ys} mõnd elementi, siis täitub alati
  \texttt{kontrolli/6} praegune tulemus ja \texttt{Ls} sisaldab vähemalt neid
  elemente, mida sisaldab \texttt{Ys}.
  \item Valem \texttt{F} ei ole veel tõene, aga temas puuduvad hetkel
  ühikklauslid, mistõttu tuleb valida suvaline muutuja dekompositsiooni
  rakendamiseks. Siin on erijuhuks olukord, kus kõik muutujad on juba varem ära
  valitud ja antud sammus ei ole see enam võimalik. Programm tagurdab
  automaatselt viimase valikupunktini, milleks on eelmine
  rekursioonisamm, kus valiti küll muutuja, aga pole veel läbi proovitud tema
  mõlemat väärtustust. Kui selliseid valikupunkte ei ole, siis lõpetab programm
  töö negatiivse tulemusega. Olukorrale, kus järgmises sammus tuleks teha
  muutuja valik, vastab \texttt{kontrolli/6} väljund \texttt{O = vali}.
\end{enumerate}

Järgnevalt esitame protseduuri \texttt{samm/5} ülalkirjeldatud juhtudele. On
selge, et ei ole tähtis definitsioonide järjekorral, sest nad on üksteist
välistavad \texttt{samm/5} esimese sisendargumendi järgi (mis samal ajal on ka
\texttt{kontrolli/6} väljundiks).

\textbf{Juht 1:}

Kuivõrd on leitud kehtestav väärtustus, siis ei pea enam midagi edasi arvutama
ja samal ajal ei ole tarvis hoolida muutujate \texttt{F, S, Ct, D} väärtustest,
mida võime ignoreerida. Seega sobib järgmine definitsioon:

\begin{lstlisting}[numbers=left,xleftmargin=1cm,basicstyle=\tt]
samm(sat, _, _, _, _).
\end{lstlisting}

\textbf{Juht 2:}

Valemis esinesid ühikklauslid, võtame esimese vastava literaali tõeseks ja
rakendame väärtustusprotseduuri \texttt{propageeri/5}, mille väljundina saame
tekkinud väärade klauslite arvu \texttt{Cf} ja tõeste klauslite arvu
\texttt{Ct1}. Loomulikult on tarvis kontrollida ega vääri klausleid
ei saadud, mida tehakse real (3). Kui saadi, lõpetatakse selles harus
arvutamine ära ja tagurdatakse, muidu arvutatakse tõeste klauslite koguarv
\texttt{Ct2} ja rekursioonisügavus \texttt{D1} ning sooritatakse
päring \texttt{dpll/6}.

\begin{lstlisting}[numbers=left,xleftmargin=1cm,basicstyle=\tt]
samm(yhik([Y|Ys]), F, S, Ct, D):-
	propageeri(Y, F, S, Cf, Ct1),
	(Cf > 0 ->
		fail
		;
		Ct2 is Ct + Ct1,
		D1 is D + 1,
		dpll(Y, F, S, Ct2, D1, Ys)
	).
\end{lstlisting}

\textbf{Juht 3:}

Viimases võimalikus olukorras toimetetakse sarnaselt eelmise juhuga. Erinev on
vaid see, et siin tuleb muutuja valida eraldi päringuga \texttt{vali\_muutuja(F,
S, L)}. See protseduur teostab valiku valemi \texttt{F} ja hetkel kehtiva
väärtustuse \texttt{S} põhjal. Tema väljundiks on literaal \texttt{L}, mis
tähistab, kas vastav muutuja tuleb valida tõeseks või vääraks. Ülejäänud osas
toimitakse samamoodi kui juhul (2), ainult et ühikliteraalide vahepinu võetakse
võrdseks tühja listiga. 

\begin{lstlisting}[numbers=left,xleftmargin=1cm,basicstyle=\tt]
samm(vali, F, S, Ct, D):-
	vali_muutuja(F, S, L),
	propageeri(L, F, S, Cf, Ct1),
	(Cf > 0 ->
		fail
		;
		Ct2 is Ct + Ct1,
		D1 is D + 1,
		dpll(L, F, S, Ct2, D1, [])
	).
\end{lstlisting}

Kehtestatavusprogramm koos silumismooduli kutsetega, mis selguse huvides jäeti
välja siin esitatud kirjeldusest, on realiseeritud programmimoodulis
\texttt{dpll} ja asub failis \textit{dpll.pl}.

\section{Lahendite loendamine}

Lahendite loendamise jaoks tuleb programmi täiustada
tõeste väärtustuste loenduriga (lisa-sisendparameeter predikaatidele
\texttt{dpll} ja \texttt{samm}) ja asendada sammu sooritamise protseduuri
\texttt{samm} definitsioon ülalkirjeldatud juhule (1) järgmisega:

\begin{lstlisting}[numbers=left,xleftmargin=1cm,basicstyle=\tt]
samm(sat, F, _, _, D, Lo):-
	muutujate_arv(F, N),
	nb_getval(Lo, C),
	C1 is C + integer(2^(N-D)),
	nb_setval(Lo, C1),
	fail.
\end{lstlisting}

Protseduuris tähistab \texttt{Lo} loendurit, millest küsitakse kõigepealt seni
saadud lahendite arv \texttt{C} (rida 3), millele liidetakse juurde hetkel
leitud lahendite arv (rida 4). Saadud tulemus \texttt{C1} salvestatakse tagasi
loendurisse \texttt{Lo} (rida 5). Protseduuri lõpus teostatakse
pseudopäring \texttt{fail} (rida 6), mis programselt kutsub esile tagurdamise.
Loenduri kui globaalse muutuja kasutamiseks tarvitatakse ekstraloogilisi predikaate
\texttt{nb\_getval/2} ning \texttt{nb\_setval/2}, mille tulemus ei sõltu
üldisest Prologi virtuaalmasina käitumisest tagurdusoperatsiooni käigus.

\section{Põhiandmestruktuurid}

Eespool esitatud programmi selgitusest on välja jäetud valemi andmestruktuuri
kirjeldus (\texttt{F} struktuur protseduurides \texttt{dpll} ja \texttt{samm}),
samuti hetkel lahendatava valemi lausearvutusmuutujate väärtuse talletamine
(\texttt{S} struktuur). Programmi üldisel kirjeldamisel on nad ebaolulised,
kuid siiski vajalikud kasutatud abipredikaatide \texttt{kontrolli/6},
\texttt{propageeri/5} ja \texttt{muutujate\_arv/2} implementeerimisel.

\subsection{Valem}

Valemit käsitletakse programmis kui staatilist struktuuri. Eespool kirjeldatud
lihtsustusteisendused on lahendatud klauslite vaheväärtuste hoidmisega ning
efektiivse muutujate-klauslite indekseerimispõhimõttega.

Valemit säilitatakse termis kujul
{\tt valem(n,m,vc(C1,...,Cn),cv(V1,...,Vm))}, kus {\tt n} on valemi muutujate
arv, {\tt m} on valemi klauslite arv ning termid {\tt vc} ja {\tt cv} on kasutusel järgmisel
otstarbel:

\begin{itemize}
	\item Termis {\tt vc(C1,...,Cn)} hoitakse muutuja indeksi {\tt i} järgi
	klausleid, kuhu antud muutuja kuulub. Iga {\tt Ci} on omakorda term {\tt
	c(Cip, Cin)}, kus {\tt Cip} tähistab nende klauslite listi, kus muutuja
	indeksiga {\tt i} positiivselt esineb ja {\tt Cin} nende klauslite listi, kus
	see muutuja negatiivselt esineb.
	\item Termis {\tt cv(V1,...,Vm)} seevastu hoitakse klausli indeksi {\tt j}
	järgi muutujaid, mis sellesse klauslisse kuuluvad. Siin tuleb sammuti vahet
	teha positiivse ja negatiivse esinemise vahel. Selletõttu oleks sobiv {\tt Vj}
	struktuur {\tt v(Vjp, Vjn)}, kus list {\tt Vjp}({\tt Vjn}) tähistab muutujate
	indekseid, mis esinevad klauslis indeksiga {\tt j} positiivselt(negatiivselt).
\end{itemize}

\textbf{Näide valemi teisendamisest sisekujule:}

\begin{verbatim}
?- teisenda([[-1, 2, -3], [-2, -4], [4]], T).
T = valem(
    4,
    3,
    vc(c([], [1]), c([1], [2]), c([], [1]), c([3], [2])),
    cv(v([2], [1, 3]), v([], [2, 4]), v([4], []))
)
\end{verbatim}

Selline andmestruktuur võimaldab leida konstantse ajaga kõik klauslid, mis
sisaldavad etteantud literaali, samuti mingi kindla klausli (teades klausli
indeksit) kõik literaalid, kuna termi argumendi inspekteerimine protseduuri
\texttt{arg/3} abil on konstantse keerukusega \cite{swi}.

\subsection{Väärtustus}

Programmis hetkel kehtivat väärtustust hoiame struktuuris \\
{\tt olek(v(X1,...,Xn), c(Y1,...,Ym))}, kus
{\tt X1,...,Xn} on vastavate muutujate väärtused ja {\tt Y1,...,Ym} klauslite
väärtused. Tõest väärtust tähistame täisarvuga 1 ja väära täisarvuga 0. Kui
muutujal väärtust ei ole, siis on vastav koht termis lihtsalt tühi. Kuna
korraga saab ühel kohal asuda ainult üks väärtus, siis garanteerib selline
andmestruktuur muutujate oleku ühesuse. Klauslite väärtustamist kasutatakse ära
programmi efektiivsuse tõstmiseks, sest siis ei ole tarvis pidevalt uuesti
välja arvutada klauslite väärtuseid.

\textbf{Näide tühja väärtustuse koostamisest valemi jaoks, milles on 4 muutujat
ja 3 klauslit:}

\begin{verbatim}
?- tyhi_olek(4, 3, S).
S = olek(v(_G256, _G259, _G262, _G265), v(_G276, _G279, _G282))
\end{verbatim}

Kasutades algoritmis ära teadmist viimati väärtustatud muutuja kohta
(\texttt{L} - viimati tõeseks valitud literaal protseduurides
\texttt{kontrolli/6} ja \texttt{propageeri/5}), on selge, et saab praktiliselt
konstantses ajas teha järgmisi operatsioone:

\begin{enumerate}
  \item Ühikklauslite leidmine - vaadata läbi ainult need klauslid, mis
  sisaldasid literaali \texttt{L} komplementaari \texttt{-L}, sest ainult
  sellistes klauslites vähenes esinevate muutujate arv.
  \item Tõeste klauslite leidmine - tõeseks märkida kõik
  klauslid, milles esines \texttt{L}.
  \item Väärade klauslite leidmine - vääraks saavad muutuda ainult klauslid,
  mis sisaldasid \texttt{L} komplementaari \texttt{-L}.
\end{enumerate}
