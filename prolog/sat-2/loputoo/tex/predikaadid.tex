\section{Moodul ''dpllp``}

Moodul DPLL algoritmi kasutamiseks läbi Prolog muutujate.

\begin{itemize}
\item {\tt dpllp(+{\it F})} - Kehtestatavuse lahendaja, mille sisendiks
on konjuktiivsel normaalkujul listiesituses
antud valem. Valemi muutujad, mis on vajalikud
valemi tõesuseks, väärtustatakse väärtusega 1 (tõene)
või 0 (väär).
Näide kasutamisest:

\begin{verbatim}
?- dpllp((A v B) => C).
C = 1 ;
A = 0,
B = 0,
C = 0
?-
\end{verbatim}


\end{itemize}

\section{Moodul ''dpll``}

DPLL kehtestatavusalgoritmi peamoodul.

\begin{itemize}
\item {\tt dpll(+{\it F}:valem, -{\it V})} - DPLL algoritmi realisatsioon. Valem {\it F} on konjuktiivsel
normaalkujul valem, mis on antud listiesituses ja mille
muutujad on kodeeritud täisarvudena. Väljund {\it V} on muutujate
väärtustus, mis on antud andmestruktuuriga {\tt olek}.
Näide kasutamisest:

\begin{verbatim}
?- dpll([[1, 2, 3], [-1]], V1).
V1 = v(0, 1, _G657)
Yes
\end{verbatim}


\end{itemize}

\section{Moodul ''abi``}

Üldiste abimeetodite moodul.

\begin{itemize}
\item {\tt algseadista\_loendur(-{\it Lo})} - Uue loenduri {\it Lo} algseadistamine väärtusele 0.

\item {\tt kl\_vaartuseta\_muutujad(+{\it F}:valem, ?{\it Ci}:int, +{\it S}:olek, -{\it Ps}:list, -{\it Ns}:list)} - Valemist {\it F} klausli {\it Ci} väärtustamata positiivsena esinevate muutujate Ps
ja negatiivsena esinevate muutujate {\it Ns} leidmine.

\item {\tt m\_filtreeri\_vaartuseta(+{\it Ms}:list, +{\it S}:olek, -{\it Ms1}:list)} - Muutujate listist {\it Ms} väärtustamata muutujate väljafiltreerimine
listi {\it Ms1}.

\item {\tt algvaartustus(+{\it Fin}:list, -{\it F}:valem, -{\it S}:olek, -{\it V}:term)} - DPLL algoritmides algandmestruktuuride koostamine.
{\it Fin} - sisendvalem, {\it F} - valem sisekujul, {\it S} - tühi väärtustus,
{\it V} - väärtustuse muutujate osa.

\item {\tt indekslist(+{\it Xs}:list, -{\it Ps}:list)} - Listi {\it Xs} konverteerimine listiks Ps, mille elementideks
on paarid {\it I-E}, kus {\it I} on elemendi täisarvuline indeks
ja {\it E} on element ise. Indeksid algavad väärtusega 1 ja väljendavad
elemendi esinemispositsiooni listi algusosast alates.

\end{itemize}

\section{Moodul ''vaartustus``}

Väärtustuse lisamiseks/kontrollimiseks kasutatvad protseduurid.

\begin{itemize}
\item {\tt tyhi\_vaartustus(+{\it F}:valem, -{\it V}:olek)} - Valemi {\it F} muutujate jaoks tühja väärtustuse koostamine.

\item {\tt muutuja\_vaartuseta(+{\it V}:olek, +{\it I}:int)} - Kontrollimine, kas etteantud indeksiga muutuja
on vaartustuses {\it V} väärtustatud hetkel või mitte.

\item {\tt klausel\_vaartuseta(+{\it V}:olek, +{\it I}:int)} - Kontrollimine, kas etteantud indeksiga klausel
on vaartustuses {\it V} väärtustatud hetkel või mitte.

\item {\tt vaartusta\_muutuja(+{\it V}:olek, +{\it I}:int, +{\it Val})} - Etteantud indeksiga {\it I} muutuja väärtustamine
väärtusega {\it Val} väärtustuses {\it V}.

\item {\tt vaartusta\_klausel(+{\it V}:olek, +{\it I}:int, +{\it Val})} - Klausli indeksiga {\it I} väärtustamine
väärtusega {\it Val} väärtustuses {\it V}.

\end{itemize}

\section{Moodul ''valem``}

Valemiga tegevuste sooritamise moodul.

\begin{itemize}
\item {\tt klausli\_muutujad(+{\it F}:valem, +{\it I}:int, -{\it Ps}:list, -{\it Ns}:list)} - Leia valemist {\it F} klauslis indeksiga {\it I} positiivse
literaalina esinevate muutujate indeksid Ps
ja negatiivsena esinevate muutujate indeksid {\it Ns}.

\item {\tt klausli\_pos\_muutujad(+{\it F}:valem, +{\it I}:int, -{\it Ps}:list)} - Leia valemist {\it F} klauslis indeksiga {\it I} positiivse
literaalina esinevate muutujate indeksid {\it Ps}.

\item {\tt klausli\_neg\_muutujad(+{\it F}:valem, +{\it I}:int, -{\it Ps}:list)} - Leia valemist {\it F} klauslis indeksiga {\it I} negatiivse
literaalina esinevate muutujate indeksid Ns.

\item {\tt muutuja\_klauslid(+{\it F}:valem, +{\it I}:int, -{\it Ps}:list, -{\it Ns}:list)} - Leia valemist {\it F} klauslite indeksid Ps,
kus muutuja indeksiga {\it I} esineb positiivselt ja klauslite
indeksid Ns, kus muutuja indeksiga {\it I} esineb negatiivselt.

\item {\tt pos\_muutuja\_klauslid(+{\it F}:valem, +{\it I}:int, -{\it Ps}:list)} - Leia valemist {\it F} klauslite indeksid Ps,
kus muutuja indeksiga {\it I} esineb positiivselt.

\item {\tt neg\_muutuja\_klauslid(+{\it F}:valem, +{\it I}:int, -{\it Ps}:list)} - Leia valemist {\it F} klauslite indeksid Ns,
kus muutuja indeksiga {\it I} esineb negatiivselt.

\item {\tt teisenda(+{\it F1}:valem, -{\it F2}:valem)} - Konjuktiivsel normaalkujul listiesituses valemi {\it F1} teisendamine sisekujule {\it F2}.
Näide kasutamisest:

\begin{verbatim}
?- teisenda([[-1, 2, -3], [-2, -4], [4]], T).
T = valem(
4,
3,
vc(c([], [1]), c([1], [2]), c([], [1]), c([3], [2])),
cv(v([2], [1, 3]), v([], [2, 4]), v([4], []))
)
\end{verbatim}


\end{itemize}

\section{Moodul ''kontrolli``}

Valemi kontrollimine vahetult pärast muutuja väärtustamist.

\begin{itemize}
\item {\tt kontrolli(-{\it O}:otsus, +{\it L}:int, +{\it F}:valem, +{\it V}:olek, +{\it Ct}:int, +{\it Ys}:list)} - Kontrolli hetkel väärtustust viimati tehtud
lahendussammu järgi. {\it O} - väljund, {\it L} - viimati tõeseks
valitud literaal, {\it F} - valem, mida lahendatakse, {\it V} - hetkel
kehtiv väärtustus, {\it Ct} - seni tõeseks saadud klauslite arv,
{\it Ys} - ühikliteraalide pinu.
Väljund {\it O} võib olla üks järgmistest: \texttt{sat} - valem on tõene;
\texttt{yhik(Ls)} - võta järgmistes sammudes tõeseks ühikliteraalid \texttt{Ls};
\texttt{vali} - vali järgmises sammus suvaline muutuja.

\end{itemize}

\section{Moodul ''propageeri``}

Protseduur muutuja väärtustuse valimise propageerimiseks.

\begin{itemize}
\item {\tt propageeri(+{\it L}:int, +{\it F}:valem, +{\it V}:olek, -{\it Cf}:int, -{\it Ct}:int)} - Etteantud tõese literaali {\it L} väärtuse propageerimine
(valemi {\it F} vastava muutuja/klauslite väärtustamine).
Väljundparameetrid: {\it Cf} - saadud väärade klauslite arv;
{\it Ct} - saadud tõeste klauslite arv.

\end{itemize}

\section{Moodul ''silur``}

Programmi silumist abistavad protseduurid.

\begin{itemize}
\item {\tt sisse} - Siluri väljundi näitamise lubamine. Kui siluri väljundi näitamine
on lubatud, kirjutatakse standardväljundisse infot algoritmi sammude
kohta.

\item {\tt valja} - Siluri väljundi näitamise keelamine. Kui siluri väljundi näitamine
on lubatud, kirjutatakse standardväljundisse infot algoritmi sammude
kohta.

\end{itemize}

\section{Moodul ''knk``}

Prologi muutujatega antud valemi teisendamine listiesituses konjuktiivsele normaalkujule.

\begin{itemize}
\item {\tt knk(+{\it F}:valem, -{\it G}:valem)} - Lausearvutusvalemi {\it F} teisendamine konjuktiivsele
normaalkujule {\it G}. Protseduur aksepteerib loogiliste
operaatoritena \texttt{-} (eitus), \texttt{v} (disjunktsioon)
\texttt{\&} (konjuktsioon), \texttt{=>} (implikatsioon),
\texttt{<=>} (ekvivalents). Operaatorite prioriteedid kasvavalt:
\texttt{-}, \texttt{\&}, \texttt{v}, \texttt{=>}, \texttt{<=>}.
Protseduur ei eemalda valemist tautoloogiaid.
Näide kasutamisest:

\begin{verbatim}
?- knk((A v B) => C, W).
W = (-A v C)& (-B v C)
\end{verbatim}


\item {\tt konj\_list(+{\it F}:valem, -{\it L}:list)} - Valemi {\it F} teisendamine konjuktiivsele normaalkujule.
Protseduur aksepteerib neidsamu operaatoreid, mida \texttt{knk/2}.

\end{itemize}

